% arara: pdflatex: {shell: yes, files: [latexindent]}
\section{The -r and -rr switches}\label{sec:replacements}

 You can instruct \texttt{latexindent.pl} to perform replacements/substitutions on your
 file by using either the \texttt{-r} or \texttt{-rr} switches; the
 only difference between the two switches is that the \texttt{-rr} instructs
 \texttt{latexindent.pl} to skip all of the other indentation operations.

 The default value of the \texttt{replacements} field is shown in
 \cref{lst:replacements}; as with all of the other fields, you are encouraged to customise
 and change this as you see fit. The options in this field will \emph{only} be
 considered if either the \texttt{-r} or \texttt{-rr} switch are
 active; when discussing YAML settings related to either of the replacement-mode switches,
 we will use the style given in \cref{lst:replacements}.

 \cmhlistingsfromfile*[style=replacements]*{../defaultSettings.yaml}[width=0.95\linewidth,before=\centering,replace-TCB]{\texttt{replacements}}{lst:replacements}

 You'll notice that, by default, there is only \emph{one} entry in the
 \texttt{replacements} field, but it can take as many entries as you would like; each
 one needs to begin with a \texttt{-} on its own line.

\subsection{Introduction to replacements}
	Let's explore the action of the default settings, and then we'll demonstrate the feature
	with further examples. with reference to \cref{lst:replacements}, the default action will
	replace every instance of the text \texttt{latexindent.pl} with \texttt{pl.latexindent}.

	Beginning with the code in \cref{lst:replace1} and running the command
	\begin{commandshell}
latexindent.pl -r replace1.tex
\end{commandshell}
	gives the output given in \cref{lst:replace1-r1}.

	\begin{cmhtcbraster}[raster column skip=.01\linewidth]
		\cmhlistingsfromfile*{demonstrations/replace1.tex}{\texttt{replace1.tex}}{lst:replace1}
		\cmhlistingsfromfile*{demonstrations/replace1-r1.tex}{\texttt{replace1.tex} default}{lst:replace1-r1}
	\end{cmhtcbraster}

	If we don't wish to perform this replacement, then we can tweak the default settings of
	\vref{lst:replacements} by changing \texttt{lookForThis} to 0; we perform this action
	in \cref{lst:replace1-yaml}, and run the command
	\begin{commandshell}
latexindent.pl -r replace1.tex -l=replace1.yaml
\end{commandshell}
	which gives the output in \cref{lst:replace1-mod1}.

	\begin{cmhtcbraster}[raster column skip=.01\linewidth]
		\cmhlistingsfromfile*{demonstrations/replace1-mod1.tex}{\texttt{replace1.tex} using \cref{lst:replace1}}{lst:replace1-mod1}
		\cmhlistingsfromfile*[style=yaml-LST]*{demonstrations/replace1.yaml}[replace-TCB]{\texttt{replace1.yaml}}{lst:replace1-yaml}
	\end{cmhtcbraster}
	We haven't yet discussed the \texttt{when} field; don't worry, we'll get to it
	as part of the discussion in what follows.

\subsection{The two types of replacements}
	There are two types of replacements:
	\begin{enumerate}
		\item \emph{string}-based replacements, which replace the string in
		      \emph{this} with the string in \emph{that}.
		      If you specify \texttt{this} and you do not specify \texttt{that}, then
		      the \texttt{that} field will be assumed to be empty.
		\item \emph{regex}-based replacements, which use the \texttt{substitution} field.
	\end{enumerate}
	We will demonstrate both in the examples that follow.

	\texttt{latexindent.pl} chooses which type of replacement to make based on which fields
	have been specified; if the \texttt{this} field is specified, then it will make
	\emph{string}-based replacements, regardless of if \texttt{substitution} is
	present or not.

\subsection{Examples of replacements}
	\begin{example}
		We begin with code given in \cref{lst:colsep}

		\cmhlistingsfromfile*{demonstrations/colsep.tex}{\texttt{colsep.tex}}{lst:colsep}

		Let's assume that our goal is to remove both of the \texttt{arraycolsep} statements; we can achieve this in
		a few different ways.

		Using the YAML in \cref{lst:colsep-yaml}, and running the command
		\begin{commandshell}
latexindent.pl -r colsep.tex -l=colsep.yaml
\end{commandshell}
		then we achieve the output in \cref{lst:colsep-mod0}.
		\begin{cmhtcbraster}[raster column skip=.01\linewidth]
			\cmhlistingsfromfile*{demonstrations/colsep-mod0.tex}{\texttt{colsep.tex} using \cref{lst:colsep}}{lst:colsep-mod0}
			\cmhlistingsfromfile*[style=yaml-LST]*{demonstrations/colsep.yaml}[replace-TCB]{\texttt{colsep.yaml}}{lst:colsep-yaml}
		\end{cmhtcbraster}
		Note that in \cref{lst:colsep}, we have specified \emph{two} separate fields, each with their own `\emph{this}' field;
		furthermore, for both of the separate fields, we have not specified `\texttt{that}', so the \texttt{that} field
		is assumed to be blank by \texttt{latexindent.pl};

		We can make the YAML in \cref{lst:colsep} more consise by exploring the \texttt{substitution} field. Using
		the settings in \cref{lst:colsep1} and running the command
		\begin{commandshell}
latexindent.pl -r colsep.tex -l=colsep1.yaml
\end{commandshell}
		then we achieve the output in \cref{lst:colsep-mod1}.
		\begin{cmhtcbraster}[raster column skip=.01\linewidth,
				raster force size=false,
				raster column 1/.style={add to width=-.1\textwidth}]
			]
			\cmhlistingsfromfile*{demonstrations/colsep-mod1.tex}{\texttt{colsep.tex} using \cref{lst:colsep1}}{lst:colsep-mod1}
			\cmhlistingsfromfile*[style=yaml-LST]*{demonstrations/colsep1.yaml}[replace-TCB,width=0.6\textwidth]{\texttt{colsep1.yaml}}{lst:colsep1}
		\end{cmhtcbraster}

		The code given in \cref{lst:colsep1} is an example of a \emph{regular expression}.
		This manual is not intended to be
		a tutorial on regular expressions; you might like to read, for example, \cite{masteringregexp} for a detailed
		covering of the topic. With reference to \cref{lst:colsep1}, we do note the following:
		\begin{itemize}
			\item the general form of the \texttt{substitution} field is \lstinline!s/regex/replacement/modifiers!. You can
			      place any regular expression you like within this;
			\item we have `escaped' the backslash by using \lstinline!\\!
			\item we have used \lstinline!\d+! to represent \emph{at least} one digit
			\item the \texttt{s} \emph{modifier} (in the \texttt{sg} at the end of the line) instructs \texttt{latexindent.pl} to
			      treat your file as one single line;
			\item the \texttt{g} \emph{modifier} (in the \texttt{sg} at the end of the line) instructs \texttt{latexindent.pl} to
			      make the substitution \emph{globally} throughout your file; you might try removing
			      the \texttt{g} modifier from \cref{lst:colsep1} and observing the
			      difference in output.
		\end{itemize}
		You might like to see \href{https://perldoc.perl.org/perlre.html#Modifiers}{https://perldoc.perl.org/perlre.html\#Modifiers}
		for details of modifiers; in general, I recommend starting with the \texttt{sg} modifiers for this feature.
	\end{example}

	\begin{example}
		We'll keep working with the file in \vref{lst:colsep} for this example.

		Using the YAML in \cref{lst:multi-line}, and running the command
		\begin{commandshell}
latexindent.pl -r colsep.tex -l=multi-line.yaml
\end{commandshell}
		then we achieve the output in \cref{lst:colsep-mod2}.
		\begin{cmhtcbraster}[raster column skip=.01\linewidth]
			\cmhlistingsfromfile*{demonstrations/colsep-mod2.tex}{\texttt{colsep.tex} using \cref{lst:multi-line}}{lst:colsep-mod2}
			\cmhlistingsfromfile*[style=yaml-LST]*{demonstrations/multi-line.yaml}[replace-TCB]{\texttt{multi-line.yaml}}{lst:multi-line}
		\end{cmhtcbraster}
		With reference to \cref{lst:multi-line}, we have specified a \emph{multi-line} version of \texttt{this} by employing the \emph{literal}
		style \lstinline!|-!. See, for example, \href{https://stackoverflow.com/questions/3790454/in-yaml-how-do-i-break-a-string-over-multiple-lines}{https://stackoverflow.com/questions/3790454/in-yaml-how-do-i-break-a-string-over-multiple-lines}
		for further options, all of which can be used in your YAML file.

		This is a natural point to explore the \texttt{when} field, specified in \vref{lst:replacements}. This field can take two values: \emph{before}
		and \emph{after}, which respectively instruct \texttt{latexindent.pl} to perform the replacements \emph{before} indentation or \emph{after} it.
		The default value is \texttt{before}.

		Using the YAML in \cref{lst:multi-line1}, and running the command
		\begin{commandshell}
latexindent.pl -r colsep.tex -l=multi-line1.yaml
\end{commandshell}
		then we achieve the output in \cref{lst:colsep-mod3}.
		\begin{cmhtcbraster}[raster column skip=.01\linewidth]
			\cmhlistingsfromfile*{demonstrations/colsep-mod3.tex}{\texttt{colsep.tex} using \cref{lst:multi-line1}}{lst:colsep-mod3}
			\cmhlistingsfromfile*[style=yaml-LST]*{demonstrations/multi-line1.yaml}[replace-TCB]{\texttt{multi-line1.yaml}}{lst:multi-line1}
		\end{cmhtcbraster}
		We note that, because we have specified \texttt{when: after}, that \texttt{latexindent.pl} has not found the string specified
		in \cref{lst:multi-line1} within the file in \vref{lst:colsep}. As it has looked for the string within \cref{lst:multi-line1} \emph{after} the indentation has been performed. After
		indentation, the string as written in \cref{lst:multi-line1} is no longer part of the file, and has therefore not been replaced.

		As a final note on this example, if you use the \texttt{-rr} switch, as follows,
		\begin{commandshell}
latexindent.pl -rr colsep.tex -l=multi-line1.yaml
\end{commandshell}
		then the \texttt{when} field is ignored, no indentation is done, and the output is as in \cref{lst:colsep-mod2}.
	\end{example}

	\begin{example}
		An important part of the substitution routine is in \emph{capture groups}.

		Assuming that we start with
		the code in \cref{lst:displaymath}, let's assume that our goal is to replace each occurrence of \lstinline!$$...$$!
		with \lstinline!\begin{equation*}...\end{equation*}!. This example is partly motivated by \href{https://tex.stackexchange.com/questions/242150/good-looking-latex-code}{tex stackexchange question 242150}.

		\cmhlistingsfromfile*{demonstrations/displaymath.tex}{\texttt{displaymath.tex}}{lst:displaymath}

		We use the settings in \cref{lst:displaymath1} and run the command
		\begin{commandshell}
latexindent.pl -r displaymath.tex -l=displaymath1.yaml
\end{commandshell}
		to receive the output given in \cref{lst:displaymath-mod1}.

		\begin{cmhtcbraster}[raster left skip=-3.75cm,
				raster right skip=-2cm,]
			\cmhlistingsfromfile*{demonstrations/displaymath-mod1.tex}{\texttt{displaymath.tex} using \cref{lst:displaymath1}}{lst:displaymath-mod1}
			\cmhlistingsfromfile*[style=yaml-LST]*{demonstrations/displaymath1.yaml}[replace-TCB]{\texttt{displaymath1.yaml}}{lst:displaymath1}
		\end{cmhtcbraster}

		A few notes about \cref{lst:displaymath1}:
		\begin{enumerate}
			\item we have used the \texttt{x} modifier, which allows us to have white space
			      within the regex;
			\item we have used a capture group, \lstinline!(.*?)! which captures the content between
			      the \lstinline!$$...$$! into the special variable, \lstinline!$1!;
			\item we have used the content of the capture group, \lstinline!$1!, in the
			      replacement text.
		\end{enumerate}
		See \href{https://perldoc.perl.org/perlre.html#Capture-groups}{https://perldoc.perl.org/perlre.html\#Capture-groups} for a discussion
		of capture groups.

		The features of the replacement switches can, of course, be combined with others from the toolkit of \texttt{latexindent.pl}. For example,
		we can combine the poly-switches  of \vref{sec:poly-switches}, which we do in \cref{lst:equation}; upon running the command
		\begin{commandshell}
latexindent.pl -r -m displaymath.tex -l=displaymath1.yaml,equation.yaml
\end{commandshell}
		then we receive the output in \cref{lst:displaymath-mod2}.

		\begin{cmhtcbraster}[
				raster force size=false,
				raster column 1/.style={add to width=-.1\textwidth},
				raster column skip=.06\linewidth]
			\cmhlistingsfromfile*{demonstrations/displaymath-mod2.tex}{\texttt{displaymath.tex} using \cref{lst:displaymath1,lst:equation}}{lst:displaymath-mod2}
			\cmhlistingsfromfile*[style=yaml-LST]*{demonstrations/equation.yaml}[MLB-TCB,width=0.55\textwidth]{\texttt{equation.yaml}}{lst:equation}
		\end{cmhtcbraster}
	\end{example}

	\begin{example}
		This example is motivated by \href{https://tex.stackexchange.com/questions/490086/bring-several-lines-together-to-fill-blank-spaces-in-texmaker}{tex stackexchange question 490086}.
		We begin with the code in \cref{lst:phrase}.

		\cmhlistingsfromfile*{demonstrations/phrase.tex}{\texttt{phrase.tex}}{lst:phrase}

		Our goal is to make the spacing uniform between the phrases. To achieve this, we emply the settings in \cref{lst:hspace},
		and run the command
		\begin{commandshell}
latexindent.pl -r phrase.tex -l=hspace.yaml
\end{commandshell}
		which gives the output in \cref{lst:phrase-mod1}.

		\begin{cmhtcbraster}
			\cmhlistingsfromfile*{demonstrations/phrase-mod1.tex}{\texttt{phrase.tex} using \cref{lst:hspace}}{lst:phrase-mod1}
			\cmhlistingsfromfile*[style=yaml-LST]*{demonstrations/hspace.yaml}[replace-TCB]{\texttt{hspace.yaml}}{lst:hspace}
		\end{cmhtcbraster}

		The \lstinline!\h+! setting in \cref{lst:hspace} say to replace \emph{at least one horizontal space} with a single space.
	\end{example}

	\begin{example}
		We begin with the code in \cref{lst:references}.

		\cmhlistingsfromfile*{demonstrations/references.tex}{\texttt{references.tex}}{lst:references}

		Our goal is to change each reference so that both the text and the reference are contained within one hyperlink. We
		achieve this by employing \cref{lst:reference} and running the command
		\begin{commandshell}
latexindent.pl -r references.tex -l=reference.yaml
\end{commandshell}
		which gives the output in \cref{lst:references-mod1}.

		\cmhlistingsfromfile*{demonstrations/references-mod1.tex}{\texttt{references.tex} using \cref{lst:reference}}{lst:references-mod1}

		\cmhlistingsfromfile*[style=yaml-LST]*{demonstrations/reference.yaml}[replace-TCB]{\texttt{reference.yaml}}{lst:reference}

		Referencing \cref{lst:reference}, the \lstinline!|! means \emph{or}, we have used \emph{capture groups}, together with an example
		of an \emph{optional} pattern, \lstinline!(?:eq)?!.
	\end{example}
