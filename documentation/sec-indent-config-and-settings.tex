% arara: pdflatex: {shell: yes, files: [latexindent]}
\section{User, local settings, \texttt{indentconfig.yaml} and \texttt{.indentconfig.yaml}}\label{sec:indentconfig}
 Editing \texttt{defaultSettings.yaml} is not ideal as it may be overwritten when
 updating your distribution--a better way to customize the settings to your liking
 is to set up your own settings file,
 \texttt{mysettings.yaml} (or any name you like, provided it ends with \texttt{.yaml}).
 The only thing you have to do is tell \texttt{latexindent.pl} where to find it.

 \texttt{latexindent.pl} will always check your home directory for \texttt{indentconfig.yaml}
 and  \texttt{.indentconfig.yaml} (unless
 it is called with the \texttt{-d} switch),
 which is a plain text file you can create that contains the \emph{absolute}
 paths for any settings files that you wish \texttt{latexindent.pl} to load. There is no difference
 between \texttt{indentconfig.yaml} and \texttt{.indentconfig.yaml}, other than the
 fact that \texttt{.indentconfig.yaml} is a `hidden' file; thank you to \cite{jacobo-diaz-hidden-config}
 for providing this feature. In what follows, we will use \texttt{indentconfig.yaml}, but it
 is understood that this equally represents \texttt{.indentconfig.yaml} as well. If you
 have both files in existence,  \texttt{indentconfig.yaml} takes priority.

 For Mac and Linux users, their home directory is \texttt{~/username} while
 Windows (Vista onwards) is \lstinline!C:\Users\username!\footnote{If you're not sure
	 where to put \texttt{indentconfig.yaml}, don't
	 worry \texttt{latexindent.pl} will tell you in the log file exactly where to
	 put it assuming it doesn't exist already.}
 \Cref{lst:indentconfig} shows a sample \texttt{indentconfig.yaml} file.

 \begin{yaml}{\texttt{indentconfig.yaml} (sample)}{lst:indentconfig}
	# Paths to user settings for latexindent.pl
	#
	# Note that the settings will be read in the order you
	# specify here- each successive settings file will overwrite
	# the variables that you specify

	paths:
	- /home/cmhughes/Documents/yamlfiles/mysettings.yaml
	- /home/cmhughes/folder/othersettings.yaml
	- /some/other/folder/anynameyouwant.yaml
	- C:\Users\chughes\Documents\mysettings.yaml
	- C:\Users\chughes\Desktop\test spaces\more spaces.yaml
\end{yaml}

 Note that the \texttt{.yaml} files you specify in \texttt{indentconfig.yaml}
 will be loaded in the order that you write them in. Each file doesn't have
 to have every switch from \texttt{defaultSettings.yaml}; in fact, I recommend
 that you only keep the switches that you want to \emph{change} in these
 settings files.

 To get started with your own settings file, you might like to save a copy of
 \texttt{defaultSettings.yaml} in another directory and call it, for
 example, \texttt{mysettings.yaml}. Once you have added the path to \texttt{indentconfig.yaml}
 you can change the switches and add more code-block names to it
 as you see fit -- have a look at \cref{lst:mysettings} for an example
 that uses four tabs for the default indent, adds the \texttt{tabbing}
 environment/command to the list of environments that contains alignment delimiters; you might also like to
 refer to the many YAML files detailed throughout the rest of this documentation.

 \begin{yaml}{\texttt{mysettings.yaml} (example)}{lst:mysettings}
# Default value of indentation
defaultIndent: "\t\t\t\t"

# environments that have tab delimiters, add more
# as needed
lookForAlignDelims:
    tabbing: 1
\end{yaml}

 You can make sure that your settings are loaded by checking \texttt{indent.log}
 for details -- if you have specified a path that \texttt{latexindent.pl} doesn't
 recognize then you'll get a warning, otherwise you'll get confirmation that
 \texttt{latexindent.pl} has read your settings file \footnote{Windows users
	 may find that they have to end \texttt{.yaml} files with a blank line}.

 \begin{warning}
	 When editing \texttt{.yaml} files it is \emph{extremely} important
	 to remember how sensitive they are to spaces. I highly recommend copying
	 and pasting from \texttt{defaultSettings.yaml} when you create your
	 first \texttt{whatevernameyoulike.yaml} file.

	 If \texttt{latexindent.pl} can not read your \texttt{.yaml} file it
	 will tell you so in \texttt{indent.log}.
 \end{warning}

\subsection{\texttt{localSettings.yaml}}\label{sec:localsettings}
	The \texttt{-l} switch tells \texttt{latexindent.pl} to look for \texttt{localSettings.yaml} in the
	\emph{same directory} as \texttt{myfile.tex}.  If you'd prefer to name your \texttt{localSettings.yaml} file something 
    different, (say, \texttt{myyaml.yaml}) then
	you can call \texttt{latexindent.pl} using, for example,
	\begin{commandshell}
latexindent.pl -l=myyaml.yaml myfile.tex
\end{commandshell}

Any settings file(s) specified using the \texttt{-l} switch will be read \emph{after} \texttt{defaultSettings.yaml} and, assuming they exist,
	user settings from \texttt{indentconfig.yaml}.

	Your settings file can contain any switches that you'd
	like to change; a sample is shown in \cref{lst:localSettings}, and you'll find plenty of further examples throughout this manual.

	\begin{yaml}{\texttt{localSettings.yaml} (example)}{lst:localSettings}
#  verbatim environments- environments specified
#  in this hash table will not be changed at all!
verbatimEnvironments:
    cmhenvironment: 0
\end{yaml}

	You can make sure that your settings file has been loaded by checking \texttt{indent.log}
	for details; if it can not be read then you receive a warning, otherwise you'll get confirmation that
	\texttt{latexindent.pl} has read your settings file.

\subsection{Settings load order}\label{sec:loadorder}
	\texttt{latexindent.pl} loads the settings files in the following order:
	\begin{enumerate}
		\item \texttt{defaultSettings.yaml} is always loaded, and can not be renamed;
		\item \texttt{anyUserSettings.yaml} and any other arbitrarily-named files specified in \texttt{indentconfig.yaml};
		\item \texttt{localSettings.yaml} but only if found in the same directory as \texttt{myfile.tex} and called
		      with \texttt{-l} switch; this file can be renamed, provided that the call to \texttt{latexindent.pl} is adjusted
		      accordingly (see \cref{sec:localsettings}). You may specify relative  paths to other
		      YAML files using the \texttt{-l} switch, separating multiple files using commas.
	\end{enumerate}
	A visual representation of this is given in \cref{fig:loadorder}.

	\begin{figure}
		\centering
		\begin{tikzpicture}[
				needed/.style={very thick, draw=blue,fill=blue!20,
						text centered, minimum height=2.5em,rounded corners=1ex},
				optional/.style={draw=black, very thick,scale=0.8,
						text centered, minimum height=2.5em,rounded corners=1ex},
				optionalfill/.style={fill=black!10},
				connections/.style={draw=black!30,dotted,line width=3pt,text=red},
			]
			% Draw diagram elements
			\node (latexindent) [needed,circle]  {\texttt{latexindent.pl}};
			\node (default) [needed,above right=.5cm of latexindent]  {\texttt{defaultSettings.yaml}};
			\node (indentconfig) [optional,right=of latexindent]  {\texttt{indentconfig.yaml}};
			\node (any) [optional,optionalfill,above right=of indentconfig]  {\texttt{any.yaml}};
			\node (name) [optional,optionalfill,right=of indentconfig]  {\texttt{name.yaml}};
			\node (you) [optional,optionalfill,below right=of indentconfig]  {\texttt{you.yaml}};
			\node (want) [optional,optionalfill,below=of indentconfig]  {\texttt{want.yaml}};
			\node (local) [optional,below=of latexindent]  {\texttt{localSettings.yaml}};
			% Draw arrows between elements
			\draw[connections,solid] (latexindent) to[in=-90]node[pos=0.5,anchor=north]{1} (default.south) ;
			\draw[connections,optional] (latexindent) -- node[pos=0.5,anchor=north]{2} (indentconfig) ;
			\draw[connections,optional] (indentconfig) to[in=-90] (any.south) ;
			\draw[connections,optional] (indentconfig) -- (name) ;
			\draw[connections,optional] (indentconfig) to[out=-45,in=90] (you) ;
			\draw[connections,optional] (indentconfig) -- (want) ;
			\draw[connections,optional] (latexindent) -- node[pos=0.5,anchor=west]{3} (local) ;
		\end{tikzpicture}
		\caption{Schematic of the load order described in \cref{sec:loadorder}; solid lines represent
			mandatory files, dotted lines represent optional files. \texttt{indentconfig.yaml} can
			contain as many files as you like. The files will be loaded in order; if you specify
			settings for the same field in more than one file, the most recent takes priority. }
		\label{fig:loadorder}
	\end{figure}
