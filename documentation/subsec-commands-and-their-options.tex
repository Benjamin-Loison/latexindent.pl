% arara: pdflatex: {shell: yes, files: [latexindent]}
\subsection{Commands and the strings between their arguments}\label{subsec:commands-string-between}
	The \texttt{command} code blocks will always look for optional (square bracketed)
	and mandatory (curly braced) arguments which can contain comments, line breaks and
	`beamer' commands \lstinline!<.*?>!  between them. There are switches that can allow
	them to contain other strings, which we discuss next.

\yamltitle{commandCodeBlocks}*{fields}

	The \texttt{commandCodeBlocks} field%
	\announce{2018-04-27}*{commandCodeBlocks}
	contains a few switches detailed in \cref{lst:commandCodeBlocks}.

	\cmhlistingsfromfile[style=commandCodeBlocks]*{../defaultSettings.yaml}[width=.8\linewidth,before=\centering,yaml-TCB]{\texttt{commandCodeBlocks}}{lst:commandCodeBlocks}

\yamltitle{roundParenthesesAllowed}{0|1}

	The need for this field was mostly motivated by commands found in code used to generate
	images in \texttt{PSTricks} and \texttt{tikz}; for example, let's consider
	the code given in \cref{lst:pstricks1}.

	\begin{minipage}{.45\textwidth}
		\cmhlistingsfromfile{demonstrations/pstricks1.tex}{\texttt{pstricks1.tex}}{lst:pstricks1}
	\end{minipage}
	\hfill
	\begin{minipage}{.45\textwidth}
		\cmhlistingsfromfile{demonstrations/pstricks1-default.tex}{\texttt{pstricks1} default output}{lst:pstricks1-default}
	\end{minipage}

	Notice that the \lstinline!\defFunction! command has an optional argument, followed by a
	mandatory argument, followed by a round-parenthesis argument, $(u,v)$.

	By default, because \texttt{roundParenthesesAllowed} is set to $1$ in
	\cref{lst:commandCodeBlocks}, then \texttt{latexindent.pl} will allow round parenthesis between
	optional and mandatory arguments. In the case of the code in \cref{lst:pstricks1},
	\texttt{latexindent.pl} finds \emph{all} the arguments of
	\lstinline!defFunction!, both before and after \lstinline!(u,v)!.

	The default output from running \texttt{latexindent.pl} on \cref{lst:pstricks1} actually
	leaves it unchanged (see \cref{lst:pstricks1-default}); note in particular, this is because of
	\texttt{noAdditionalIndentGlobal} as discussed on \cpageref{page:command:noAddGlobal}.

	Upon using the YAML settings in \cref{lst:noRoundParentheses}, and running the command
	\begin{commandshell}
latexindent.pl pstricks1.tex -l noRoundParentheses.yaml
        \end{commandshell}
	we obtain the output given in  \cref{lst:pstricks1-nrp}.

	\begin{cmhtcbraster}[raster column skip=.1\linewidth]
		\cmhlistingsfromfile{demonstrations/pstricks1-nrp.tex}{\texttt{pstricks1.tex} using \cref{lst:noRoundParentheses}}{lst:pstricks1-nrp}
		\cmhlistingsfromfile[style=yaml-LST]*{demonstrations/noRoundParentheses.yaml}[yaml-TCB]{\texttt{noRoundParentheses.yaml}}{lst:noRoundParentheses}
	\end{cmhtcbraster}

	Notice the difference between \cref{lst:pstricks1-default} and \cref{lst:pstricks1-nrp}; in
	particular, in \cref{lst:pstricks1-nrp}, because round parentheses are
	\emph{not} allowed, \texttt{latexindent.pl} finds that the
	\lstinline!\defFunction! command finishes at the first opening round parenthesis. As such,
	the remaining braced, mandatory, arguments are found to be \texttt{UnNamedGroupingBracesBrackets} (see \vref{tab:code-blocks}) which,
	by default, assume indentation for their body, and hence the tabbed indentation in
	\cref{lst:pstricks1-nrp}.

	Let's explore this using the YAML given in \cref{lst:defFunction} and run the command
	\begin{commandshell}
latexindent.pl pstricks1.tex -l defFunction.yaml
        \end{commandshell}
	then the output is as in \cref{lst:pstricks1-indent-rules}.

	\begin{cmhtcbraster}[raster column skip=.1\linewidth]
		\cmhlistingsfromfile[showspaces=true]{demonstrations/pstricks1-indent-rules.tex}{\texttt{pstricks1.tex} using \cref{lst:defFunction}}{lst:pstricks1-indent-rules}
		\cmhlistingsfromfile[style=yaml-LST]*{demonstrations/defFunction.yaml}[yaml-TCB]{\texttt{defFunction.yaml}}{lst:defFunction}
	\end{cmhtcbraster}

	Notice in \cref{lst:pstricks1-indent-rules} that the \emph{body} of the
	\lstinline!defFunction! command i.e, the subsequent lines containing arguments after the
	command name, have received the single space of indentation specified by
	\cref{lst:defFunction}.

\yamltitle{stringsAllowedBetweenArguments}*{fields}
	\texttt{tikz} users may well specify code such as that given in
	\cref{lst:tikz-node1}; processing this code using
	\texttt{latexindent.pl} gives the default output in \cref{lst:tikz-node1-default}.

	\begin{minipage}{.45\textwidth}
		\cmhlistingsfromfile[columns=fixed]*{demonstrations/tikz-node1.tex}{\texttt{tikz-node1.tex}}{lst:tikz-node1}
	\end{minipage}
	\hfill
	\begin{minipage}{.45\textwidth}
		\cmhlistingsfromfile[columns=fixed]*{demonstrations/tikz-node1-default.tex}{\texttt{tikz-node1} default output}{lst:tikz-node1-default}
	\end{minipage}

	With reference to \vref{lst:commandCodeBlocks}, we see that the strings
	\begin{quote}
		to, node, ++
	\end{quote}
	are all allowed to appear between arguments; importantly, you are encouraged to add
	further names to this field as necessary. This means that when \texttt{latexindent.pl}
	processes \cref{lst:tikz-node1}, it consumes:
	\begin{itemize}
		\item the optional argument \lstinline![thin]!
		\item the round-bracketed argument \lstinline!(c)! because \texttt{roundParenthesesAllowed} is
		      $1$ by default
		\item the string \lstinline!to! (specified in \texttt{stringsAllowedBetweenArguments})
		\item the optional argument \lstinline![in=110,out=-90]!
		\item the string \lstinline!++! (specified in \texttt{stringsAllowedBetweenArguments})
		\item the round-bracketed argument \lstinline!(0,-0.5cm)! because \texttt{roundParenthesesAllowed} is
		      $1$ by default
		\item the string \lstinline!node! (specified in \texttt{stringsAllowedBetweenArguments})
		\item the optional argument \lstinline![below,align=left,scale=0.5]!
	\end{itemize}

	We can explore this further, for example using \cref{lst:draw} and running the
	command
	\begin{commandshell}
latexindent.pl tikz-node1.tex -l draw.yaml  
\end{commandshell}
	we receive the output given in \cref{lst:tikz-node1-draw}.

	\begin{cmhtcbraster}[raster column skip=.1\linewidth]
		\cmhlistingsfromfile[showspaces=true]{demonstrations/tikz-node1-draw.tex}{\texttt{tikz-node1.tex} using \cref{lst:draw}}{lst:tikz-node1-draw}
		\cmhlistingsfromfile[style=yaml-LST]*{demonstrations/draw.yaml}[yaml-TCB]{\texttt{draw.yaml}}{lst:draw}
	\end{cmhtcbraster}

	Notice that each line after the \lstinline!\draw! command (its `body') in
	\cref{lst:tikz-node1-draw} has been given the appropriate two-spaces worth of indentation
	specified in \cref{lst:draw}.

	Let's compare this with the output from using the YAML settings in
	\cref{lst:no-strings}, and running the command
	\begin{commandshell}
latexindent.pl tikz-node1.tex -l no-strings.yaml  
\end{commandshell}
	given in \cref{lst:tikz-node1-no-strings}.

	\begin{cmhtcbraster}[raster column skip=.1\linewidth]
		\cmhlistingsfromfile{demonstrations/tikz-node1-no-strings.tex}{\texttt{tikz-node1.tex} using \cref{lst:no-strings}}{lst:tikz-node1-no-strings}
		\cmhlistingsfromfile[style=yaml-LST]*{demonstrations/no-strings.yaml}[yaml-TCB]{\texttt{no-strings.yaml}}{lst:no-strings}
	\end{cmhtcbraster}

	In this case, \texttt{latexindent.pl} sees that:
	\begin{itemize}
		\item the \lstinline!\draw! command finishes after the \lstinline!(c)!, as
		      \texttt{stringsAllowedBetweenArguments} has been set to $0$ so there are no strings
		      allowed between arguments;
		\item it finds a \texttt{namedGroupingBracesBrackets} called \texttt{to} (see \vref{tab:code-blocks}) \emph{with}
		      argument \lstinline![in=110,out=-90]!
		\item it finds another \texttt{namedGroupingBracesBrackets} but this time called \texttt{node} with
		      argument \lstinline![below,align=left,scale=0.5]!
	\end{itemize}

	Referencing \vref{lst:commandCodeBlocks},%
	\announce{2018-04-27}*{amalgamate feature in commandCodeBlocks}, we see that the first field in the \texttt{stringsAllowedBetweenArguments} is
	\texttt{amalgamate} and is set to \texttt{1} by default. This is for
	users who wish to specify their settings in multiple YAML files. For example, by using
	the settings in either \cref{lst:amalgamate-demo} or\cref{lst:amalgamate-demo1} is equivalent to
	using the settings in \cref{lst:amalgamate-demo2}.

	\begin{cmhtcbraster}[raster columns=3,
			raster left skip=-3.5cm,
			raster right skip=-2cm,
			raster column skip=.03\linewidth]
		\cmhlistingsfromfile[style=yaml-LST]*{demonstrations/amalgamate-demo.yaml}[yaml-TCB]{\texttt{amalgamate-demo.yaml}}{lst:amalgamate-demo}
		\cmhlistingsfromfile[style=yaml-LST]*{demonstrations/amalgamate-demo1.yaml}[yaml-TCB]{\texttt{amalgamate-demo1.yaml}}{lst:amalgamate-demo1}
		\cmhlistingsfromfile[style=yaml-LST]*{demonstrations/amalgamate-demo2.yaml}[yaml-TCB]{\texttt{amalgamate-demo2.yaml}}{lst:amalgamate-demo2}
	\end{cmhtcbraster}

	We specify \texttt{amalgamate} to be set to \texttt{0} and in which case
	any settings loaded prior to those specified, including the default, will be overwritten.
	For example, using the settings in \cref{lst:amalgamate-demo3} means that only the strings
	specified in that field will be used.

	\cmhlistingsfromfile[style=yaml-LST]*{demonstrations/amalgamate-demo3.yaml}[yaml-TCB]{\texttt{amalgamate-demo3.yaml}}{lst:amalgamate-demo3}

	It is important to note that the \texttt{amalgamate} field, if used, must be in the
	first field, and specified using the syntax given in \cref{lst:amalgamate-demo1,lst:amalgamate-demo2,lst:amalgamate-demo3}.

	We may explore this feature further with the code in \cref{lst:for-each}, whose
	default output is given in \cref{lst:for-each-default}.

	\begin{cmhtcbraster}[raster column skip=.1\linewidth]
		\cmhlistingsfromfile{demonstrations/for-each.tex}{\texttt{for-each.tex}}{lst:for-each}
		\cmhlistingsfromfile{demonstrations/for-each-default.tex}{\texttt{for-each} default output}{lst:for-each-default}
	\end{cmhtcbraster}

	Let's compare this with the output from using the YAML settings in
	\cref{lst:foreach}, and running the command
	\begin{commandshell}
latexindent.pl for-each.tex -l foreach.yaml  
\end{commandshell}
	given in \cref{lst:for-each-mod1}.

	\begin{cmhtcbraster}[raster column skip=.1\linewidth]
		\cmhlistingsfromfile{demonstrations/for-each-mod1.tex}{\texttt{for-each.tex} using \cref{lst:foreach}}{lst:for-each-mod1}
		\cmhlistingsfromfile[style=yaml-LST]*{demonstrations/foreach.yaml}[yaml-TCB]{\texttt{foreach.yaml}}{lst:foreach}
	\end{cmhtcbraster}

	You might like to compare the output given in \cref{lst:for-each-default} and
	\cref{lst:for-each-mod1}. Note,in particular, in \cref{lst:for-each-default} that the
	\texttt{foreach} command has not included any of the subsequent strings, and that
	the braces have been treated as a \texttt{namedGroupingBracesBrackets}. In \cref{lst:for-each-mod1} the
	\texttt{foreach} command has been allowed to have \lstinline!\x/\y! and
	\texttt{in} between arguments because of the settings given in
	\cref{lst:foreach}.

\yamltitle{commandNameSpecial}*{fields}
	There are some special command names%
	\announce{2018-04-27}*{commandNameSpecial} that do not fit within the names recognised by
	\texttt{latexindent.pl}, the first one of which is \lstinline!\@ifnextchar[!. From the
	perspective of \texttt{latexindent.pl}, the whole of the text \lstinline!\@ifnextchar[! is
	a command, because it is immediately followed by sets of mandatory arguments. However,
	without the \texttt{commandNameSpecial} field, \texttt{latexindent.pl} would not be able to
	label it as such, because the \lstinline![! is, necessarily, not matched by a
	closing \lstinline!]!.

	For example, consider the sample file in \cref{lst:ifnextchar}, which has default
	output in \cref{lst:ifnextchar-default}.

	\begin{cmhtcbraster}[raster column skip=.1\linewidth]
		\cmhlistingsfromfile{demonstrations/ifnextchar.tex}{\texttt{ifnextchar.tex}}{lst:ifnextchar}
		\cmhlistingsfromfile{demonstrations/ifnextchar-default.tex}{\texttt{ifnextchar.tex} default output}{lst:ifnextchar-default}
	\end{cmhtcbraster}

	Notice that in \cref{lst:ifnextchar-default} the \texttt{parbox} command has been able
	to indent its body, because \texttt{latexindent.pl} has successfully found the command
	\lstinline!\@ifnextchar! first; the pattern-matching of \texttt{latexindent.pl} starts
	from \emph{the inner most <thing> and works outwards}, discussed in more detail on \cpageref{page:phases}.

	For demonstration, we can compare this output with that given in \cref{lst:ifnextchar-off}
	in which the settings from \cref{lst:no-ifnextchar} have dictated that no special command
	names, including the \lstinline!\@ifnextchar[! command, should not be searched for
	specially; as such, the \texttt{parbox} command has been \emph{unable}
	to indent its body successfully, because the \lstinline!\@ifnextchar[! command has not been
	found.

	\begin{cmhtcbraster}[raster column skip=.1\linewidth]
		\cmhlistingsfromfile{demonstrations/ifnextchar-off.tex}{\texttt{ifnextchar.tex} using \cref{lst:no-ifnextchar}}{lst:ifnextchar-off}
		\cmhlistingsfromfile[style=yaml-LST]*{demonstrations/no-ifnextchar.yaml}[yaml-TCB]{\texttt{no-ifnextchar.yaml}}{lst:no-ifnextchar}
	\end{cmhtcbraster}

	The \texttt{amalgamate} field can be used for \texttt{commandNameSpecial}, just as for
	\texttt{stringsAllowedBetweenArguments}. The same condition holds as stated previously, which we state
	again here:
	\begin{warning}
		It is important to note that the \texttt{amalgamate} field, if used, in either \texttt{commandNameSpecial} or \texttt{stringsAllowedBetweenArguments} must be in the first field,
		and specified using the syntax given in \cref{lst:amalgamate-demo1,lst:amalgamate-demo2,lst:amalgamate-demo3}.
	\end{warning}
