% arara: pdflatex: { files: [latexindent]}
\section{Introduction}
\subsection{Thanks}
 I first created \texttt{latexindent.pl} to help me format chapter files in a big project.
 After I blogged about it on the \TeX{} stack exchange \cite{cmhblog} I received some
 positive feedback and follow-up feature requests. A big thank you to Harish Kumar
 \cite{harish} who helped to develop and test the initial versions of the script.

 The \texttt{YAML}-based interface of \texttt{latexindent.pl} was inspired by the
 wonderful \texttt{arara} tool; any similarities are deliberate, and I hope that it is
 perceived as the compliment that it is. Thank you to Paulo Cereda and the team for
 releasing this awesome tool; I initially worried that I was going to have to make a GUI
 for \texttt{latexindent.pl}, but the release of \texttt{arara} has meant there is no
 need.

 There have been several contributors to the project so far (and hopefully more in the
 future!); thank you very much to the people detailed in \vref{sec:contributors} for
 their valued contributions, and thank you to those who report bugs and request features
 at \cite{latexindent-home}.

\subsection{License}
 \texttt{latexindent.pl} is free and open source, and it always will be; it
 is released under the GNU General Public License v3.0.

 Before you start using it on any important files, bear in mind that
 \texttt{latexindent.pl} has the option to overwrite your \texttt{.tex} files. It will
 always make at least one backup (you can choose how many it makes, see
 \cpageref{page:onlyonebackup}) but you should still be careful when using it. The script
 has been tested on many files, but there are some known limitations (see
 \cref{sec:knownlimitations}). You, the user, are responsible for ensuring that you
 maintain backups of your files before running \texttt{latexindent.pl} on them. I think
 it is important at this stage to restate an important part of the license here:
 \begin{quote}\itshape
  This program is distributed in the hope that it will be useful,
  but WITHOUT ANY WARRANTY; without even the implied warranty of
  MERCHANTABILITY or FITNESS FOR A PARTICULAR PURPOSE. See the
  GNU General Public License for more details.
 \end{quote}
 There is certainly no malicious intent in releasing this script, and I do hope that it
 works as you expect it to; if it does not, please first of all make sure that you have
 the correct settings, and then feel free to let me know at \cite{latexindent-home} with
 a complete minimum working example as I would like to improve the code as much as
 possible.

 \begin{warning}
  Before you try the script on anything important (like your thesis), test it out on the
  sample files in the \texttt{test-case} directory \cite{latexindent-home}.
  \index{warning!be sure to test before use}
 \end{warning}

 \emph{If you have used any version 2.* of \texttt{latexindent.pl}, there
 are a few changes to the interface; see \vref{app:differences} and the comments
 throughout this document for details}.

\subsection{Quick start}\label{sec:quickstart}
 If you'd like to get started with \texttt{latexindent.pl} then simply type

 \begin{commandshell}
latexindent.pl myfile.tex
\end{commandshell}

 from the command line.

 We give an introduction to \texttt{latexindent.pl} using \cref{lst:quick-start}; there
 is no explanation in this section, which is deliberate for a quick start. The rest of
 the manual is more verbose.

 \cmhlistingsfromfile{demonstrations/quick-start.tex}{\texttt{quick-start.tex}}{lst:quick-start}

 Running

 \begin{commandshell}
latexindent.pl quick-start.tex
\end{commandshell}

 gives \cref{lst:quick-start-default}.

 \cmhlistingsfromfile{demonstrations/quick-start-default.tex}{\texttt{quick-start-default.tex}}{lst:quick-start-default}

 \begin{example}
 Running

 \begin{commandshell}
latexindent.pl -l quick-start1.yaml quick-start.tex
\end{commandshell}

 gives \cref{lst:quick-start-mod1}.

 \begin{cmhtcbraster}
  \cmhlistingsfromfile[style=yaml-LST]{demonstrations/quick-start1.yaml}[yaml-TCB]{\texttt{quick-start1.yaml}}{lst:quick-start1yaml}
  \cmhlistingsfromfile[showspaces=true]{demonstrations/quick-start-mod1.tex}{\texttt{quick-start-mod1.tex}}{lst:quick-start-mod1}
 \end{cmhtcbraster}
 \end{example}

 \begin{example}
 Running

 \begin{commandshell}
latexindent.pl -l quick-start2.yaml quick-start.tex
\end{commandshell}

 gives \cref{lst:quick-start-mod2}.

 \begin{cmhtcbraster}
  \cmhlistingsfromfile[style=yaml-LST]{demonstrations/quick-start2.yaml}[yaml-TCB]{\texttt{quick-start2.yaml}}{lst:quick-start2yaml}
  \cmhlistingsfromfile[showspaces=true]{demonstrations/quick-start-mod2.tex}{\texttt{quick-start-mod2.tex}}{lst:quick-start-mod2}
 \end{cmhtcbraster}
 \end{example}

 \begin{example}
 Running

 \begin{commandshell}
latexindent.pl -l quick-start3.yaml quick-start.tex
\end{commandshell}

 gives \cref{lst:quick-start-mod3}.

 \begin{cmhtcbraster}
  \cmhlistingsfromfile[style=yaml-LST]{demonstrations/quick-start3.yaml}[yaml-TCB]{\texttt{quick-start3.yaml}}{lst:quick-start3yaml}
  \cmhlistingsfromfile{demonstrations/quick-start-mod3.tex}{\texttt{quick-start-mod3.tex}}{lst:quick-start-mod3}
 \end{cmhtcbraster}
 \end{example}

 \begin{example}
 Running

 \begin{commandshell}
latexindent.pl -m -l quick-start4.yaml quick-start.tex
\end{commandshell}

 gives \cref{lst:quick-start-mod4}.

 \begin{cmhtcbraster}
  \cmhlistingsfromfile[style=yaml-LST]{demonstrations/quick-start4.yaml}[MLB-TCB]{\texttt{quick-start4.yaml}}{lst:quick-start4yaml}
  \cmhlistingsfromfile{demonstrations/quick-start-mod4.tex}{\texttt{quick-start-mod4.tex}}{lst:quick-start-mod4}
 \end{cmhtcbraster}
 Full details of text wrapping in \cref{subsec:textwrapping}.
 \end{example}

 \begin{example}
 Running

 \begin{commandshell}
latexindent.pl -m -l quick-start5.yaml quick-start.tex
\end{commandshell}

 gives \cref{lst:quick-start-mod5}.

 \begin{cmhtcbraster}
  \cmhlistingsfromfile[style=yaml-LST]{demonstrations/quick-start5.yaml}[MLB-TCB]{\texttt{quick-start5.yaml}}{lst:quick-start5yaml}
  \cmhlistingsfromfile{demonstrations/quick-start-mod5.tex}{\texttt{quick-start-mod5.tex}}{lst:quick-start-mod5}
 \end{cmhtcbraster}
 Full details of text wrapping in \cref{subsec:textwrapping}.
 \end{example}

 \begin{example}
 Running

 \begin{commandshell}
latexindent.pl -m -l quick-start6.yaml quick-start.tex
\end{commandshell}

 gives \cref{lst:quick-start-mod6}.

 \begin{cmhtcbraster}
  \cmhlistingsfromfile[style=yaml-LST]{demonstrations/quick-start6.yaml}[MLB-TCB]{\texttt{quick-start6.yaml}}{lst:quick-start6yaml}
  \cmhlistingsfromfile{demonstrations/quick-start-mod6.tex}{\texttt{quick-start-mod6.tex}}{lst:quick-start-mod6}
 \end{cmhtcbraster}
 This is an example of a \emph{poly-switch}; full details of \emph{poly-switches} are
 covered in \cref{sec:poly-switches}.
 \end{example}

 \begin{example}
 Running

 \begin{commandshell}
latexindent.pl -m -l quick-start7.yaml quick-start.tex
\end{commandshell}

 gives \cref{lst:quick-start-mod7}.

 \begin{cmhtcbraster}
  \cmhlistingsfromfile[style=yaml-LST]{demonstrations/quick-start7.yaml}[MLB-TCB]{\texttt{quick-start7.yaml}}{lst:quick-start7yaml}
  \cmhlistingsfromfile{demonstrations/quick-start-mod7.tex}{\texttt{quick-start-mod7.tex}}{lst:quick-start-mod7}
 \end{cmhtcbraster}
 Full details of \emph{poly-switches} are covered in \cref{sec:poly-switches}.
 \end{example}

\subsection{Required perl modules}
 If you receive an error message such as that given in
 \cref{lst:poss-errors}, then you need to install the missing perl modules.
 \begin{cmhlistings}[style=tcblatex,language=Perl]{Possible error messages}{lst:poss-errors}
Can't locate File/HomeDir.pm in @INC (@INC contains: /Library/Perl/5.12/darwin-thread-multi-2level /Library/Perl/5.12 /Network/Library/Perl/5.12/darwin-thread-multi-2level /Network/Library/Perl/5.12 /Library/Perl/Updates/5.12.4/darwin-thread-multi-2level /Library/Perl/Updates/5.12.4 /System/Library/Perl/5.12/darwin-thread-multi-2level /System/Library/Perl/5.12 /System/Library/Perl/Extras/5.12/darwin-thread-multi-2level /System/Library/Perl/Extras/5.12 .) at helloworld.pl line 10.
BEGIN failed--compilation aborted at helloworld.pl line 10.
\end{cmhlistings}
 \texttt{latexindent.pl} ships with a script to help with this process; if you run the
 following script, you should be prompted to install the appropriate modules.

 \begin{commandshell}
perl latexindent-module-installer.pl
\end{commandshell}

 You might also like to see
 \href{https://stackoverflow.com/questions/19590042/error-cant-locate-file-homedir-pm-in-inc}{https://stackoverflow.com/questions/19590042/error-cant-locate-file-homedir-pm-in-inc},
 for example, as well as \vref{sec:requiredmodules}.

\subsection{About this documentation}
 As you read through this documentation, you will see many listings; in this version of
 the documentation, there are a total of \totallstlistings. This may seem a lot, but I
 deem it necessary in presenting the various different options of \texttt{latexindent.pl}
 and the associated output that they are capable of producing.

 The different listings are presented using different styles:

 \begin{minipage}{.4\textwidth}
  \cmhlistingsfromfile{demonstrations/demo-tex.tex}{\texttt{demo-tex.tex}}{lst:demo-tex}
 \end{minipage}%
 \hfill
 \begin{minipage}{.4\textwidth}
  This type of listing is a \texttt{.tex} file.
 \end{minipage}%

 \begin{minipage}{.4\textwidth}
  \cmhlistingsfromfile[style=fileExtensionPreference]{../defaultSettings.yaml}[width=.8\linewidth,before=\centering,yaml-TCB]{\texttt{fileExtensionPreference}}{lst:fileExtensionPreference-demo}
 \end{minipage}%
 \hfill
 \begin{minipage}{.4\textwidth}
  This type of listing is a \texttt{.yaml} file; when you see line numbers given (as here)
  it means that the snippet is taken directly from \texttt{defaultSettings.yaml}, discussed in
  detail in \vref{sec:defuseloc}.
 \end{minipage}%

 \begin{minipage}{.55\textwidth}
  \cmhlistingsfromfile[style=modifylinebreaks]{../defaultSettings.yaml}[MLB-TCB,width=.85\linewidth,before=\centering]{\texttt{modifyLineBreaks}}{lst:modifylinebreaks-demo}
 \end{minipage}%
 \hfill
 \begin{minipage}{.4\textwidth}
  This type of listing is a \texttt{.yaml} file, but it will only
  be relevant when the \texttt{-m} switch is active; see \vref{sec:modifylinebreaks}
  for more details.
 \end{minipage}%

 \begin{minipage}{.55\textwidth}
  \cmhlistingsfromfile[style=replacements]{../defaultSettings.yaml}[replace-TCB,width=.85\linewidth,before=\centering]{\texttt{replacements}}{lst:replacements-demo}
 \end{minipage}%
 \hfill
 \begin{minipage}{.4\textwidth}
  This type of listing is a \texttt{.yaml} file, but it will only
  be relevant when the \texttt{-r} switch is active; see \vref{sec:replacements}
  for more details.
 \end{minipage}%

 % \begin{latexonly}
 You will occasionally see dates shown in the margin (for example, next to this
 paragraph!) \announce{2017-06-25}{announce} which detail the date of the version in which
 the feature was implemented; the `N' stands for `new as of the date shown' and `U' stands
 for `updated as of the date shown'. If you see \stardemo, it means that the feature is
 either new (N) or updated (U) as of the release of the current version; if you see
 \stardemo\, attached to a listing, then it means that listing is new (N) or updated (U)
 as of the current version. If you have not read this document before (and even if you
 have!), then you can ignore every occurrence of the \stardemo; they are simply there to
 highlight new and updated features. The new and updated features in this documentation
 (\gitRel) are on the following pages: \listOfNewFeatures% % \end{latexonly}

\subsection{A word about regular expressions}
 \index{regular expressions!a word about}
 As you read this documentation, you may encounter the term \emph{regular expressions}.
 I've tried to write this documentation in such a way so as to allow you to engage with
 them or not, as you prefer. This documentation is not designed to be a guide to regular
 expressions, and if you'd like to read about them, I recommend \cite{masteringregexp}.
