% arara: pdflatex: {shell: yes, files: [latexindent]}
\subsection{Conflicting poly-switches: sequential code blocks}
	It is very easy to have conflicting poly-switches; if we use the example from
	\vref{lst:mycommand1}, and consider the YAML settings given in \cref{lst:mycom-mlb4}.
	The output from running
	\begin{commandshell}
latexindent.pl -m -l=mycom-mlb4.yaml mycommand1.tex
\end{commandshell}
	is given in \cref{lst:mycom-mlb4}.

	\begin{cmhtcbraster}
		\cmhlistingsfromfile{demonstrations/mycommand1-mlb4.tex}{\texttt{mycommand1.tex} using \cref{lst:mycom-mlb4}}{lst:mycommand1-mlb4}
		\cmhlistingsfromfile[style=yaml-LST]*{demonstrations/mycom-mlb4.yaml}[MLB-TCB,width=\linewidth]{\texttt{mycom-mlb4.yaml}}{lst:mycom-mlb4}
	\end{cmhtcbraster}

	Studying \cref{lst:mycom-mlb4}, we see that the two poly-switches are at opposition with
	one another:
	\begin{itemize}
		\item on the one hand, \texttt{LCuBStartsOnOwnLine} should \emph{not} start on its own line
		      (as poly-switch is set to $-1$);
		\item on the other hand, \texttt{RCuBFinishesWithLineBreak} \emph{should} finish with a line break.
	\end{itemize}
	So, which should win the conflict? As demonstrated in \cref{lst:mycommand1-mlb4}, it is clear
	that \texttt{LCuBStartsOnOwnLine} won this conflict, and the reason is that
	\emph{the second argument was processed after the first} -- in general, the most recently-processed code block and
	associated poly-switch takes priority.

	We can explore this further by considering the YAML settings in \cref{lst:mycom-mlb5};
	upon running the command
	\begin{commandshell}
latexindent.pl -m -l=mycom-mlb5.yaml mycommand1.tex
\end{commandshell}
	we obtain the output given in \cref{lst:mycommand1-mlb5}.

	\begin{cmhtcbraster}[raster column skip=.1\linewidth]
		\cmhlistingsfromfile{demonstrations/mycommand1-mlb5.tex}{\texttt{mycommand1.tex} using \cref{lst:mycom-mlb5}}{lst:mycommand1-mlb5}
		\cmhlistingsfromfile[style=yaml-LST]*{demonstrations/mycom-mlb5.yaml}[MLB-TCB,width=\linewidth]{\texttt{mycom-mlb5.yaml}}{lst:mycom-mlb5}
	\end{cmhtcbraster}

	As previously, the most-recently-processed code block takes priority -- as before, the
	second (i.e, \emph{last}) argument. Exploring this further, we consider the
	YAML settings in \cref{lst:mycom-mlb6}, which give associated output in
	\cref{lst:mycommand1-mlb6}.

	\begin{cmhtcbraster}[raster column skip=.1\linewidth]
		\cmhlistingsfromfile{demonstrations/mycommand1-mlb6.tex}{\texttt{mycommand1.tex} using \cref{lst:mycom-mlb6}}{lst:mycommand1-mlb6}
		\cmhlistingsfromfile[style=yaml-LST]*{demonstrations/mycom-mlb6.yaml}[MLB-TCB,width=\linewidth]{\texttt{mycom-mlb6.yaml}}{lst:mycom-mlb6}
	\end{cmhtcbraster}

	Note that a \lstinline!%! \emph{has} been added to the trailing
	first \lstinline!}!; this is because:
	\begin{itemize}
		\item while processing the \emph{first} argument, the trailing line break has been
		      removed (\texttt{RCuBFinishesWithLineBreak} set to $-1$);
		\item while processing the \emph{second} argument, \texttt{latexindent.pl} finds that
		      it does \emph{not} begin on its own line, and so because
		      \texttt{LCuBStartsOnOwnLine} is set to $2$, it adds a comment, followed by a
		      line break.
	\end{itemize}

\subsection{Conflicting poly-switches: nested code blocks}
	Now let's consider an example when nested code blocks have conflicting poly-switches;
	we'll use the code in \cref{lst:nested-env}, noting that it contains nested
	environments.

	\cmhlistingsfromfile{demonstrations/nested-env.tex}{\texttt{nested-env.tex}}{lst:nested-env}

	Let's use the YAML settings given in \cref{lst:nested-env-mlb1-yaml}, which upon running the
	command
	\begin{commandshell}
latexindent.pl -m -l=nested-env-mlb1.yaml nested-env.tex
\end{commandshell}
	gives the output in \cref{lst:nested-env-mlb1}.

	\begin{cmhtcbraster}[raster column skip=.05\linewidth]
		\cmhlistingsfromfile{demonstrations/nested-env-mlb1.tex}{\texttt{nested-env.tex} using \cref{lst:nested-env-mlb1-yaml}}{lst:nested-env-mlb1}
		\cmhlistingsfromfile[style=yaml-LST]*{demonstrations/nested-env-mlb1.yaml}[MLB-TCB,width=\linewidth]{\texttt{nested-env-mlb1.yaml}}{lst:nested-env-mlb1-yaml}
	\end{cmhtcbraster}

	In \cref{lst:nested-env-mlb1}, let's first of all note that both environments have received
	the appropriate (default) indentation; secondly, note that the poly-switch
	\texttt{EndStartsOnOwnLine} appears to have won the conflict, as \lstinline!\end{one}! has
	had its leading line break removed.

	To understand it, let's talk about the three basic phases \label{page:phases}of
	\texttt{latexindent.pl}:
	\begin{enumerate}
		\item Phase 1: packing, in which code blocks are replaced with unique ids, working from
		      \emph{the inside to the outside}, and then sequentially -- for example, in \cref{lst:nested-env},
		      the \texttt{two} environment is found \emph{before} the
		      \texttt{one} environment; if the -m switch is active, then during this phase:
		      \begin{itemize}
			      \item line breaks at the beginning of the \texttt{body} can be added (if
			            \texttt{BodyStartsOnOwnLine} is $1$ or $2$) or removed (if
			            \texttt{BodyStartsOnOwnLine} is $-1$);
			      \item line breaks at the end of the body can be added (if \texttt{EndStartsOnOwnLine} is
			            $1$ or $2$) or removed (if \texttt{EndStartsOnOwnLine} is
			            $-1$);
			      \item line breaks after the end statement can be added (if \texttt{EndFinishesWithLineBreak} is
			            $1$ or $2$).
		      \end{itemize}
		\item Phase 2: indentation, in which white space is added to the begin, body, and end
		      statements;
		\item Phase 3: unpacking, in which unique ids are replaced by their \emph{indented} code
		      blocks; if the -m switch is active, then during this phase,
		      \begin{itemize}
			      \item line breaks before \texttt{begin} statements can be added or removed (depending
			            upon \texttt{BeginStartsOnOwnLine});
			      \item line breaks after \emph{end} statements can be removed but
			            \emph{NOT} added (see \texttt{EndFinishesWithLineBreak}).
		      \end{itemize}
	\end{enumerate}

	With reference to \cref{lst:nested-env-mlb1}, this means that during Phase 1:
	\begin{itemize}
		\item the \texttt{two} environment is found first, and the line break ahead of the
		      \lstinline!\end{two}! statement is removed because \texttt{EndStartsOnOwnLine} is set to
		      $-1$. Importantly, because, \emph{at this stage},
		      \lstinline!\end{two}! \emph{does} finish with a line break,
		      \texttt{EndFinishesWithLineBreak} causes no action.
		\item next, the \texttt{one} environment is found; the line break ahead of
		      \lstinline!\end{one}! is removed because \texttt{EndStartsOnOwnLine} is set to
		      $-1$.
	\end{itemize}
	The indentation is done in Phase 2; in Phase 3 \emph{there is no option to add a line break after the \lstinline!end! statements}. We can justify
	this by remembering that during Phase 3, the \texttt{one} environment will be
	found and processed first, followed by the \texttt{two} environment. If the
	\texttt{two} environment were to add a line break after the
	\lstinline!\end{two}! statement, then \texttt{latexindent.pl} would have no way of
	knowing how much indentation to add to the subsequent text (in this case,
	\lstinline!\end{one}!).

	We can explore this further using the poly-switches in \cref{lst:nested-env-mlb2}; upon
	running the command
	\begin{commandshell}
latexindent.pl -m -l=nested-env-mlb2.yaml nested-env.tex
\end{commandshell}
	we obtain the output given in \cref{lst:nested-env-mlb2-output}.

	\begin{cmhtcbraster}
		\cmhlistingsfromfile{demonstrations/nested-env-mlb2.tex}{\texttt{nested-env.tex} using \cref{lst:nested-env-mlb2}}{lst:nested-env-mlb2-output}
		\cmhlistingsfromfile[style=yaml-LST]*{demonstrations/nested-env-mlb2.yaml}[MLB-TCB,width=\linewidth]{\texttt{nested-env-mlb2.yaml}}{lst:nested-env-mlb2}
	\end{cmhtcbraster}

	During Phase 1:
	\begin{itemize}
		\item the \texttt{two} environment is found first, and the line break ahead of the
		      \lstinline!\end{two}! statement is not changed because \texttt{EndStartsOnOwnLine} is set to
		      $1$. Importantly, because, \emph{at this stage},
		      \lstinline!\end{two}! \emph{does} finish with a line break,
		      \texttt{EndFinishesWithLineBreak} causes no action.
		\item next, the \texttt{one} environment is found; the line break ahead of
		      \lstinline!\end{one}! is already present, and no action is needed.
	\end{itemize}
	The indentation is done in Phase 2, and then in Phase 3, the \texttt{one}
	environment is found and processed first, followed by the \texttt{two}
	environment. \emph{At this stage}, the \texttt{two} environment finds
	\texttt{EndFinishesWithLineBreak} is $-1$, so it removes the trailing line break;
	remember, at this point, \texttt{latexindent.pl} has completely finished with the
	\texttt{one} environment.
