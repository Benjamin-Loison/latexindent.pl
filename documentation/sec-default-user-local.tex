% arara: pdflatex: {shell: yes, files: [latexindent]}
\section{default}\label{sec:defuseloc}
 \texttt{latexindent.pl} loads its settings from \texttt{defaultSettings.yaml}
 (rhymes with camel). The idea is to separate the behaviour of the script
 from the internal working -- this is very similar to the way that we separate content
 from form when writing our documents in \LaTeX.

\subsection{defaultSettings.yaml}
	If you look in \texttt{defaultSettings.yaml} you'll find the switches
	that govern the behaviour of \texttt{latexindent.pl}. If you're not sure where
	\texttt{defaultSettings.yaml} resides on your computer, don't worry as \texttt{indent.log}
	will tell you where to find it.
	\texttt{defaultSettings.yaml} is commented,
	but here is a description of what each switch is designed to do. The default
	value is given in each case; whenever you see \emph{integer} in \emph{this}
	section, assume that it must be greater than or equal to \texttt{0} unless
	otherwise stated.

	You can certainly feel free to edit \texttt{defaultSettings.yaml}, but
	this is not ideal as it may be overwritten when you update your \TeX{} distribution --
	all of your hard work tweaking the script would be undone! Don't worry,
	there's a solution, feel free to peek ahead to \cref{sec:indentconfig} if you like.

\yamltitle{fileExtensionPreference}*{fields}
	\texttt{latexindent.pl} can be called to
	act on a file without
	specifying the file extension.  For example we can call
	\begin{commandshell}
latexindent.pl myfile
\end{commandshell}
	\begin{wrapfigure}[8]{r}[0pt]{6cm}
		\cmhlistingsfromfile[firstnumber=22,linerange={22-26},style=yaml-LST,numbers=left]{../defaultSettings.yaml}[width=.8\linewidth,before=\centering,yaml-TCB]{\texttt{fileExtensionPreference}}{lst:fileExtensionPreference}
	\end{wrapfigure}

	in which case the script will look for \texttt{myfile} with the extensions
	specified in \texttt{fileExtensionPreference} in their numeric order. If
	no match is found, the script will exit. As with all of the fields, you should
	change and/or add to this as necessary.

	Calling \texttt{latexindent.pl myfile} with the (default) settings specified in \cref{lst:fileExtensionPreference}
	means that the script will first look for \texttt{myfile.tex}, then \texttt{myfile.sty}, \texttt{myfile.cls},
	and finally \texttt{myfile.bib} in order.

\yamltitle{backupExtension}*{extension name}

	If you call \texttt{latexindent.pl} with the \texttt{-w} switch (to overwrite
	\texttt{myfile.tex}) then it will create a backup file before doing
	any indentation; the default extension is \texttt{.bak}, so, for example, \texttt{myfile.bak0}
	would be created when calling \texttt{latexindent.pl myfile.tex}.

	By default, every time you subsequently call \texttt{latexindent.pl} with
	the \texttt{-w} to act upon \texttt{myfile.tex}, it will create successive back up files: \texttt{myfile.bak1}, \texttt{myfile.bak2},
	etc.

\yamltitle{onlyOneBackUp}*{integer}
	\label{page:onlyonebackup}
	If you don't want a backup for every time that you call \texttt{latexindent.pl} (so
	you don't want \texttt{myfile.bak1}, \texttt{myfile.bak2}, etc) and you simply
	want \texttt{myfile.bak} (or whatever you chose \texttt{backupExtension} to be)
	then change \texttt{onlyOneBackUp} to \texttt{1}; the default value of
	\texttt{onlyOneBackUp} is \texttt{0}.

\yamltitle{maxNumberOfBackUps}*{integer}
	Some users may only want a finite number of backup files,
	say at most $3$, in which case, they can change this switch.
	The smallest value of \texttt{maxNumberOfBackUps} is $0$ which will \emph{not}
	prevent backup files being made; in this case, the behaviour will be dictated
	entirely by \texttt{onlyOneBackUp}. The default value of \texttt{maxNumberOfBackUps}
	is \texttt{0}.

\yamltitle{cycleThroughBackUps}*{integer}
	Some users may wish to cycle through backup files, by deleting the
	oldest backup file and keeping only the most recent; for example,
	with \texttt{maxNumberOfBackUps: 4}, and \texttt{cycleThroughBackUps}
	set to \texttt{1}  then the \texttt{copy} procedure given below
	would be obeyed.

	\begin{commandshell}
copy myfile.bak1 to myfile.bak0
copy myfile.bak2 to myfile.bak1
copy myfile.bak3 to myfile.bak2
copy myfile.bak4 to myfile.bak3
	\end{commandshell}
	The default value of \texttt{cycleThroughBackUps} is \texttt{0}.

\yamltitle{logFilePreferences}*{fields}
	\begin{wrapfigure}[10]{r}[0pt]{9cm}
		\cmhlistingsfromfile[firstnumber=63,linerange={63-67},style=yaml-LST,numbers=left]{../defaultSettings.yaml}[width=.85\linewidth,before=\centering,yaml-TCB]{\texttt{logFilePreferences}}{lst:logFilePreferences}
	\end{wrapfigure}
	\texttt{latexindent.pl} writes information to \texttt{indent.log}, some
	of which can be customised by changing \texttt{logFilePreferences}; see \cref{lst:logFilePreferences}.
	If you load your own user settings (see \vref{sec:indentconfig}) then \texttt{latexindent.pl} will
	detail them in \texttt{indent.log}; you can choose not to have the details logged by switching
	\texttt{showEveryYamlRead} to \texttt{0}. Once all of your settings have
	been loaded, you can see the amalgamated settings in the log file by switching \texttt{showAmalgamatedSettings}
	to \texttt{1}, if you wish. The log file will end with the characters
	given in \texttt{endLogFileWith}, and will report the \texttt{GitHub} address
	of \texttt{latexindent.pl} to the log file if \texttt{showGitHubInfoFooter} is set to \texttt{1}.

\yamltitle{verbatimEnvironments}*{fields}

	\begin{wrapfigure}[14]{r}[0pt]{6cm}
		\cmhlistingsfromfile[firstnumber=71,linerange={71-73},style=yaml-LST,numbers=left]{../defaultSettings.yaml}[width=.8\linewidth,before=\centering,yaml-TCB]{\texttt{verbatimEnvironments}}{lst:verbatimEnvironments}

		\vspace{.2cm}
		\cmhlistingsfromfile[firstnumber=76,linerange={76-78},style=yaml-LST,numbers=left]{../defaultSettings.yaml}[width=.8\linewidth,before=\centering,yaml-TCB]{\texttt{verbatimCommands}}{lst:verbatimCommands}
	\end{wrapfigure}
	A field that contains a list of environments
	that you would like left completely alone -- no indentation will be performed
	on environments that you have specified in this field, see \cref{lst:verbatimEnvironments}.

	Note that if  you put an environment in \texttt{verbatimEnvironments}
	and in other fields such as \texttt{lookForAlignDelims} or \texttt{noAdditionalIndent}
	then \texttt{latexindent.pl} will \emph{always} prioritize \texttt{verbatimEnvironments}.

\yamltitle{verbatimCommands}*{fields}
	A field that contains a list of commands that are verbatim commands, for example
	\lstinline|\lstinline|; any commands populated in this field are protected from line breaking
	routines (only relevant if the \texttt{-m} is active, see \vref{sec:modifylinebreaks}).

\yamltitle{noIndentBlock}*{fields}

	\begin{wrapfigure}[8]{r}[0pt]{6cm}
		\cmhlistingsfromfile[firstnumber=84,linerange={84-86},style=yaml-LST,numbers=left]{../defaultSettings.yaml}[width=.8\linewidth,before=\centering,yaml-TCB]{\texttt{noIndentBlock}}{lst:noIndentBlock}
	\end{wrapfigure}
	If you have a block of code that you don't want \texttt{latexindent.pl} to touch (even if it is \emph{not} a verbatim-like
	environment) then you can wrap it in an environment from \texttt{noIndentBlock};
	you can use any name you like for this, provided you populate it as demonstrate in
	\cref{lst:noIndentBlock}.

	Of course, you don't want to have to specify these as null environments
	in your code, so you use them with a comment symbol, \lstinline!%!, followed
	by as many spaces (possibly none) as you like; see \cref{lst:noIndentBlockdemo} for
	example.

	\begin{cmhlistings}[style=demo,escapeinside={(*@}{@*)}]{\texttt{noIndentBlock} demonstration}{lst:noIndentBlockdemo}
%(*@@*) \begin{noindent}
        this code
                won't
     be touched
                    by
             latexindent.pl!
%(*@@*)\end{noindent}
	\end{cmhlistings}

\yamltitle{removeTrailingWhitespace}*{fields}\label{yaml:removeTrailingWhitespace}

	\begin{wrapfigure}[12]{r}[0pt]{6cm}
		\cmhlistingsfromfile[firstnumber=89,linerange={89-91},style=yaml-LST,numbers=left]{../defaultSettings.yaml}[width=.8\linewidth,before=\centering,yaml-TCB]{removeTrailingWhitespace}{lst:removeTrailingWhitespace}

		\vspace{.2cm}
		\cmhlistingsfromfile[firstnumber=95,linerange={95-98},style=yaml-LST,numbers=left]{../defaultSettings.yaml}[width=.8\linewidth,before=\centering,yaml-TCB]{\texttt{fileContentsEnvironments}}{lst:fileContentsEnvironments}
	\end{wrapfigure}
	Trailing white space can be removed both \emph{before} and \emph{after} processing
	the document, as detailed in \cref{lst:removeTrailingWhitespace}; each of the fields
	can take the values \texttt{0} or \texttt{1}. See \vref{lst:removeTWS-before,lst:env-mlb5-modAll,lst:env-mlb5-modAll-remove-WS}
	for before and after results.  Thanks to \cite{vosskuhle} for providing this feature.

\yamltitle{fileContentsEnvironments}*{field}

	Before \texttt{latexindent.pl} determines the difference between preamble (if any) and the main document,
	it first searches for any of the environments specified in \texttt{fileContentsEnvironments}, see
	\cref{lst:fileContentsEnvironments}.
	The behaviour of \texttt{latexindent.pl} on these environments is determined by their location (preamble or not), and
	the value \texttt{indentPreamble}, discussed next.

\yamltitle{indentPreamble}{0|1}

	The preamble of a document can sometimes contain some trickier code
	for \texttt{latexindent.pl} to operate upon. By default, \texttt{latexindent.pl}
	won't try to operate on the preamble (as \texttt{indentPreamble} is set to \texttt{0},
	by default), but if you'd like \texttt{latexindent.pl} to try then change \texttt{indentPreamble} to \texttt{1}.

\yamltitle{lookForPreamble}*{fields}

	\begin{wrapfigure}[8]{r}[0pt]{5cm}
		\cmhlistingsfromfile[firstnumber=103,linerange={103-107},style=yaml-LST,numbers=left]{../defaultSettings.yaml}[width=.8\linewidth,before=\centering,yaml-TCB]{lookForPreamble}{lst:lookForPreamble}
	\end{wrapfigure}
	Not all files contain preamble; for example, \texttt{sty}, \texttt{cls} and \texttt{bib} files typically do \emph{not}. Referencing
	\cref{lst:lookForPreamble}, if you set, for example, \texttt{.tex} to \texttt{0}, then regardless of the setting of the value of \texttt{indentPreamble}, preamble
	will not be assumed when operating upon \texttt{.tex} files.
\yamltitle{preambleCommandsBeforeEnvironments}{0|1}
	Assuming that \texttt{latexindent.pl} is asked to operate upon the preamble of a document,
	when this switch is set to \texttt{0} then environment code blocks will be sought first,
	and then command code blocks. When this switch is set to \texttt{1}, commands
	will be sought first. The example that first motivated this switch contained the code given in \cref{lst:motivatepreambleCommandsBeforeEnvironments}.

	\begin{cmhlistings}{Motivating \texttt{preambleCommandsBeforeEnvironments}}{lst:motivatepreambleCommandsBeforeEnvironments}
...
preheadhook={\begin{mdframed}[style=myframedstyle]},
postfoothook=\end{mdframed},
...
\end{cmhlistings}

\yamltitle{defaultIndent}*{horizontal space}
	This is the default indentation (\lstinline!\t! means a tab, and is the default value) used in the absence of other details
	for the command or environment we are working with; see \texttt{indentRules} in \vref{sec:noadd-indent-rules}
	for more details.

	If you're interested in experimenting with \texttt{latexindent.pl} then you
	can \emph{remove} all indentation by setting \texttt{defaultIndent: ""}

\yamltitle{lookForAlignDelims}*{fields}
	\begin{wrapfigure}[12]{r}[0pt]{5cm}
		\begin{yaml}[numbers=none]{\texttt{lookForAlignDelims} (basic)}[width=.8\linewidth,before=\centering]{lst:aligndelims:basic}
lookForAlignDelims:
   tabular: 1
   tabularx: 1
   longtable: 1
   array: 1
   matrix: 1
   ...
	\end{yaml}
	\end{wrapfigure}
	This contains a list of environments and/or commands that
	are operated upon in a special way by \texttt{latexindent.pl} (see \cref{lst:aligndelims:basic}).
	In fact, the fields in \texttt{lookForAlignDelims} can actually
	take two different forms: the \emph{basic} version is shown in \cref{lst:aligndelims:basic}
	and the \emph{advanced} version in \cref{lst:aligndelims:advanced}; we will discuss each in turn.

	The environments specified in this field will be operated on in a special way  by \texttt{latexindent.pl}. In particular, it will try and align each column by its alignment
	tabs. It does have some limitations (discussed further in \cref{sec:knownlimitations}),
	but in many cases it will produce results such as those in \cref{lst:tabularbefore:basic,lst:tabularafter:basic}.

	If you find that \texttt{latexindent.pl} does not perform satisfactorily on such
	environments then you can set the relevant key to \texttt{0}, for example \texttt{tabular: 0}; alternatively, if you just want to ignore \emph{specific}
	instances of the environment, you could wrap them in something from \texttt{noIndentBlock} (see \cref{lst:noIndentBlock}).

	\begin{minipage}{.45\textwidth}
		\cmhlistingsfromfile[columns=fixed]{demonstrations/tabular1.tex}{\texttt{tabular} before}{lst:tabularbefore:basic}
	\end{minipage}%
	\hfill
	\begin{minipage}{.45\textwidth}
		\cmhlistingsfromfile[columns=fixed]{demonstrations/tabular1-default.tex}{\texttt{tabular} after (basic)}{lst:tabularafter:basic}
	\end{minipage}%

	If you wish to remove the alignment of the \lstinline!\\! within a delimiter-aligned block, then the
	advanced form of \texttt{lookForAlignDelims} shown in \cref{lst:aligndelims:advanced} is for you.

	\cmhlistingsfromfile[firstnumber=121,linerange={121-128},style=yaml-LST,numbers=left]{../defaultSettings.yaml}[yaml-TCB]{\texttt{lookForAlignDelims} (advanced)}{lst:aligndelims:advanced}

	Note that you can use a mixture of the basic and advanced form: in \cref{lst:aligndelims:advanced} \texttt{tabular} and \texttt{tabularx}
	are advanced and \texttt{longtable} is basic. When using the advanced form, each field should receive at least 1 sub-field, and \emph{can} (but does not have to) receive up to 3 fields:
	\begin{itemize}
		\item \texttt{delims}: switch equivalent to simply specifying, for example, \texttt{tabular: 1} in
		      the basic version shown in \cref{lst:aligndelims:basic} (default: 1);
		\item \texttt{alignDoubleBackSlash}: switch to determine if \lstinline!\\! should be aligned (default: 1);
		\item \texttt{spacesBeforeDoubleBackSlash}: optionally, specifies the number of spaces to be inserted
		      before (non-aligned) \lstinline!\\!. In order to use this field, \texttt{alignDoubleBackSlash} needs
		      to be set to 0 (default: 0).
	\end{itemize}

	Assuming that you have the settings in \cref{lst:aligndelims:advanced} saved in \texttt{mysettings.yaml}, and the code
	from \cref{lst:tabularbefore:basic} in \texttt{myfile.tex} and you run
	\begin{commandshell}
latexindent.pl -l mysettings.yaml myfile.tex 
\end{commandshell}
	then you should receive the before-and-after results shown in
	\cref{lst:tabularbefore:advanced,lst:tabularafter:advanced}; note that the ampersands have been aligned, but
	the \lstinline!\\! have not (compare the alignment of \lstinline!\\! in \cref{lst:tabularafter:basic,lst:tabularafter:advanced}).

	\begin{minipage}{.45\textwidth}
		\cmhlistingsfromfile[columns=fixed]{demonstrations/tabular1.tex}{\texttt{tabular} before }{lst:tabularbefore:advanced}
	\end{minipage}%
	\hfill
	\begin{minipage}{.45\textwidth}
		\cmhlistingsfromfile[columns=fixed]{demonstrations/tabular1-advanced.tex}{\texttt{tabular} after (advanced)}{lst:tabularafter:advanced}
	\end{minipage}%

	Using  \texttt{spacesBeforeDoubleBackSlash: 3} gives \cref{lst:tabularbefore:spacing,lst:tabularafter:spacing},
	note the spacing before the \lstinline!\\! in \cref{lst:tabularafter:spacing}.

	\begin{minipage}{.45\textwidth}
		\cmhlistingsfromfile[columns=fixed]{demonstrations/tabular1.tex}{\texttt{tabular} before}{lst:tabularbefore:spacing}
	\end{minipage}%
	\hfill
	\begin{minipage}{.45\textwidth}
		\cmhlistingsfromfile[columns=fixed]{demonstrations/tabular1-advanced-3spaces.tex}{\texttt{tabular} after (spacing)}{lst:tabularafter:spacing}
	\end{minipage}%

	As of Version 3.0, the alignment routine works on mandatory and optional arguments within commands, and also within `special' code blocks
	(see \fixthis{need a reference to special section!}); for example, assuming that you have a command called \lstinline!\matrix!
	and that it is populated within \texttt{lookForAlignDelims} (which it is, by default), then the before-and-after results
	shown in \cref{lst:matrixbefore,lst:matrixafter} are achievable by default.

	\begin{minipage}{.45\textwidth}
		\cmhlistingsfromfile[columns=fixed]{demonstrations/matrix1.tex}{\texttt{matrix} before}{lst:matrixbefore}
	\end{minipage}%
	\hfill
	\begin{minipage}{.45\textwidth}
		\cmhlistingsfromfile[columns=fixed]{demonstrations/matrix1-default.tex}{\texttt{matrix} after}{lst:matrixafter}
	\end{minipage}%

	\fixthis{need to resolve alignment outside of environments}
	%If you have blocks of code that you wish to align at the \&  character that
	%are \emph{not} wrapped in, for example, \texttt{\begin{tabular}\ldots\end{tabular}}, then you use the mark up
	%  illustrated in \cref{lst:alignmentmarkup}. Note that the \texttt{%*} must be next to
	%  each other, but that there can be any number of spaces (possibly none) between the
	%  \texttt{*} and \texttt{\begin{tabular}}; note also that you may use any
	%    environment name that you have specified in \texttt{lookForAlignDelims}.
	%    \begin{cmhlistings}[style=demo,columns=fixed]{Mark up for aligning delimiters outside of environments}{lst:alignmentmarkup}
	%      \matrix{%
	%      %* \begin{tabular}
	%         1 & 2 & 3 & 4 \\
	%         5 &   & 6 &   \\
	%        %* \end{tabular}
	%        }

	\fixthis{final check: line numbers in yaml}

\yamltitle{indentAfterItems}*{fields}
	\begin{wrapfigure}[5]{r}[0pt]{7cm}
		\cmhlistingsfromfile[firstnumber=155,linerange={155-158},style=yaml-LST,numbers=left]{../defaultSettings.yaml}[width=.8\linewidth,before=\centering,yaml-TCB]{\texttt{indentAfterItems}}{lst:indentafteritems}
	\end{wrapfigure}
	The environments specified in \texttt{indentAfterItems}  tell
	\texttt{latexindent.pl} to look for \lstinline!\item! commands; if these switches are set to \texttt{1}
	then indentation will be performed so as indent the code after each \texttt{item}.
	A demonstration is given in \cref{lst:itemsbefore,lst:itemsafter}

	\begin{minipage}{.45\textwidth}
		\cmhlistingsfromfile{demonstrations/items1.tex}{\texttt{items} before}{lst:itemsbefore}
	\end{minipage}%
	\hfill
	\begin{minipage}{.45\textwidth}
		\cmhlistingsfromfile{demonstrations/items1-default.tex}{\texttt{items} after}{lst:itemsafter}
	\end{minipage}

\yamltitle{itemNames}*{fields}
	\begin{wrapfigure}[5]{r}[0pt]{5cm}
		\cmhlistingsfromfile[firstnumber=164,linerange={164-166},style=yaml-LST,numbers=left]{../defaultSettings.yaml}[width=.8\linewidth,before=\centering,yaml-TCB]{\texttt{itemNames}}{lst:itemNames}
	\end{wrapfigure}
	If you have your own \texttt{item} commands (perhaps you
	prefer to use \texttt{myitem}, for example)
	then you can put populate them in \texttt{itemNames}.
	For example, users of the \texttt{exam} document class might like to add
	\texttt{parts} to \texttt{indentAfterItems} and \texttt{part} to \texttt{itemNames}
	to their user settings--see \vref{sec:indentconfig} for details of how to configure user settings,
	and \vref{lst:mysettings} in particular.\label{page:examsettings}

\yamltitle{specialBeginEnd}*{fields}
	The fields specified in \texttt{specialBeginEnd} are, in their default state, focused on math mode begin and end statements, but
	there is no requirement for this to be the case; \cref{lst:specialBeginEnd} shows the
	default settings of \texttt{specialBeginEnd}.

	\cmhlistingsfromfile[firstnumber=170,linerange={170-182},style=yaml-LST,numbers=left]{../defaultSettings.yaml}[width=.8\linewidth,before=\centering,yaml-TCB]{\texttt{specialBeginEnd}}{lst:specialBeginEnd}

	The field \texttt{displayMath} represents \lstinline!\[...\]!, \texttt{inlineMath} represents
	\lstinline!$...$! and \texttt{displayMathTex} represents \lstinline!$$...$$!. You can, of course,
	rename these in your own YAML files (see \vref{sec:localsettings}); indeed, you
	might like to set up your own specil begin and end statements.

	A demonstration of the before-and-after results are shown in \cref{lst:specialbefore,lst:specialafter}.

	\begin{minipage}{.45\textwidth}
		\cmhlistingsfromfile{demonstrations/special1.tex}{\texttt{special1.tex} before}{lst:specialbefore}
	\end{minipage}%
	\hfill
	\begin{minipage}{.45\textwidth}
		\cmhlistingsfromfile{demonstrations/special1-default.tex}{\texttt{special1.tex} after}{lst:specialafter}
	\end{minipage}

	For each field, the \texttt{lookForThis} is set to \texttt{1} by default, which means that \texttt{latexindent.pl}
	will look for this pattern; you can tell \texttt{latexindent.pl} not to look for the pattern, by setting
	\texttt{lookForThis} to \texttt{0}.

\yamltitle{indentAfterHeadings}*{fields}
	\begin{wrapfigure}[17]{r}[0pt]{8cm}
		\cmhlistingsfromfile[firstnumber=192,linerange={192-201},style=yaml-LST,numbers=left]{../defaultSettings.yaml}[width=.8\linewidth,before=\centering,yaml-TCB]{\texttt{indentAfterHeadings}}{lst:indentAfterHeadings}
	\end{wrapfigure}
	This field enables the user to specify
	indentation rules that take effect after heading commands such as \lstinline!\part!, \lstinline!\chapter!,
	\lstinline!\section!, \lstinline!\subsection*!, or indeed any user-specified command written in this field.\footnote{There is a slight
		difference in interface for this field when comparing Version 2.2 to Version 3.0; see \vref{app:differences} for details.}

	The default settings do \emph{not} place indentation after a heading, but you
	can easily switch them on by changing \texttt{indentAfterThisHeading: 0} to \texttt{indentAfterThisHeading: 1}.
	The \texttt{level} field tells \texttt{latexindent.pl} the hierarchy of the heading
	structure in your document. You might, for example, like to have both \texttt{section}
	and \texttt{subsection} set with \texttt{level: 3} because you do not want the indentation to go too deep.

	You can add any of your own custom heading commands to this field, specifying the \texttt{level}
	as appropriate.  You can also specify your own indentation in \texttt{indentRules} \fixthis{need a reference to indentRules};
	you will find the default \texttt{indentRules} contains \lstinline!chapter: " "! which
	tells \texttt{latexindent.pl} simply to use a space character after \texttt{\chapter} headings
	(once \texttt{indent} is set to \texttt{1} for \texttt{chapter}).

	For example, assuming that you have read \vref{sec:localsettings}, say that
	you have the code in \cref{lst:headings1yaml} saved into \texttt{headings1.yaml},
	and that you have the text from \cref{lst:headings1} saved into \texttt{headings1.tex}.

	\begin{minipage}{.45\textwidth}
		\cmhlistingsfromfile[style=yaml-LST]{demonstrations/headings1.yaml}[yaml-TCB]{\texttt{headings1.yaml}}{lst:headings1yaml}
	\end{minipage}%
	\hfill
	\begin{minipage}{.45\textwidth}
		\cmhlistingsfromfile{demonstrations/headings1.tex}{\texttt{headings1.tex}}{lst:headings1}
	\end{minipage}

	If you run the command
	\begin{commandshell}
latexindent.pl headings1.tex -l=headings1.yaml
\end{commandshell}
	then you should receive the output given in \cref{lst:headings1-mod1}.

	\begin{minipage}{.45\textwidth}
		\cmhlistingsfromfile{demonstrations/headings1-mod1.tex}{\texttt{headings1.tex} using \cref{lst:headings1yaml}}{lst:headings1-mod1}
	\end{minipage}%
	\hfill
	\begin{minipage}{.45\textwidth}
		\cmhlistingsfromfile{demonstrations/headings1-mod2.tex}{\texttt{headings1.tex} second modification}{lst:headings1-mod2}
	\end{minipage}

	Now say that you modify the \texttt{YAML} from \cref{lst:headings1yaml} so that the \texttt{paragraph} \texttt{level} is \texttt{1}; after
	running
	\begin{commandshell}
latexindent.pl headings1.tex -l=headings1.yaml
\end{commandshell}
	you should now receive the code given in \cref{lst:headings1-mod2}; notice that
	the \texttt{paragraph} and \texttt{subsection} are at the same indentation level.

    \subsubsection{The code blocks known \texttt{latexindent.pl}}\label{subsubsec:code-blocks}
	As of Version 3.0, \texttt{latexindent.pl} processes documents using code blocks; each
	of these are shown in \cref{tab:code-blocks}.

	\begin{longtable}{m{.3\linewidth}@{\hspace{.25cm}}m{.4\linewidth}@{}m{.2\linewidth}}
		\caption{Code blocks known to \texttt{latexindent.pl}}\label{tab:code-blocks}\\
		\toprule
		Code block                    & characters allowed in name                                                                                     & example                                                                                                                                                    \\
		\midrule
		environments                  & \lstinline!a-zA-Z@\*0-9_\\!                                                                                        &
		\begin{lstlisting}[,nolol=true,]
\begin{myenv}
body of myenv
\end{myenv}
  \end{lstlisting}
		\\\cmidrule{2-3}
		optionalArguments             & \emph{inherits} name from parent (e.g environment name)                                                        &
		\begin{lstlisting}[,nolol=true,]
[
opt arg text
]
  \end{lstlisting}
		\\\cmidrule{2-3}
		mandatoryArguments            & \emph{inherits} name from parent (e.g environment name)                                                        &
		\begin{lstlisting}[,nolol=true,]
{
mand arg text
}
  \end{lstlisting}
		\\\cmidrule{2-3}
		commands                      & \lstinline!+a-zA-Z@\*0-9_\:!                                                                                        & \lstinline!\mycommand!$\langle$\itshape{arguments}$\rangle$                                                                                               \\\cmidrule{2-3}
		keyEqualsValuesBraces         & \lstinline!a-zA-Z@\*0-9_\/.\h\{\}:\#-!                                                                                        & \lstinline!my key/.style=!$\langle$\itshape{arguments}$\rangle$                                                                                               \\\cmidrule{2-3}
		namedGroupingBracesBrackets   & \lstinline!a-zA-Z@\*><!                                                                                        & \lstinline!in!$\langle$\itshape{arguments}$\rangle$                                                                                               \\\cmidrule{2-3}
		UnNamedGroupingBracesBrackets & \centering\emph{No name!}                                                                                      & \lstinline!{! or \lstinline![! or \lstinline!,! or \lstinline!&! or \lstinline!)! or \lstinline!(! or \lstinline!$! followed by $\langle$\itshape{arguments}$\rangle$ \\\cmidrule{2-3}
		ifElseFi                      & \lstinline!@a-zA-Z! but must begin with either \newline \lstinline!\if! of \lstinline!\@if! &
		\begin{lstlisting}[,nolol=true,]
\ifnum...
...
\else
...
\fi
  \end{lstlisting}\\\cmidrule{2-3}
		items                         & User specified, see \vref{lst:indentafteritems,lst:itemNames}                                                  &
		\begin{lstlisting}[,nolol=true,]
\begin{enumerate}
  \item ...
\end{enumerate}
  \end{lstlisting}\\\cmidrule{2-3}
		specialBeginEnd               & User specified, see \vref{lst:specialBeginEnd}                                                                 &
		\begin{lstlisting}[,nolol=true,]
\[
  ...
\]
  \end{lstlisting}\\\cmidrule{2-3}
		afterHeading                  & User specified, see \vref{lst:indentAfterHeadings}                                                             &
		\begin{lstlisting}[,morekeywords={chapter},nolol=true,]
\chapter{title}
  ...
\section{title}
  \end{lstlisting}\\\cmidrule{2-3}
		filecontents                  & User specified, see \vref{lst:fileContentsEnvironments}                                                        &
		\begin{lstlisting}[,nolol=true,]
\begin{filecontents}
...
\end{filecontents}
  \end{lstlisting}\\
		\bottomrule
	\end{longtable}

    We will refer to these code blocks in what follows.
