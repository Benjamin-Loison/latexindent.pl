% arara: pdflatex: {shell: yes, files: [latexindent]}
\subsection{\texttt{noAdditionalIndent} and \texttt{indentRules}}
\fixthis{need an introduction}
\subsubsection{Environments and their arguments}
There are a few different YAML switches governing the indentation of environments; let's start 
with the simple sample code shown in \cref{lst:myenvtex}.

\cmhlistingsfromfile{demonstrations/myenvironment-simple.tex}{\texttt{myenv.tex}}{lst:myenvtex}

\yamltitle{noAdditionalIndent}*{0|1 {\bfseries OR} fields} 
If we do not wish \texttt{myenv} to receive any additional indentation, we have a few choices available to us, 
as demonstrated in \cref{lst:myenv-noAdd1,lst:myenv-noAdd2}.

\begin{minipage}{.45\textwidth}
\cmhlistingsfromfile[style=yaml-LST]{demonstrations/myenv-noAdd1.yaml}[width=.8\linewidth,before=\centering,yaml-TCB]{\texttt{myenv-noAdd1.yaml}}{lst:myenv-noAdd1}
\end{minipage}
\hfill
\begin{minipage}{.45\textwidth}
\cmhlistingsfromfile[style=yaml-LST]{demonstrations/myenv-noAdd2.yaml}[width=.8\linewidth,before=\centering,yaml-TCB]{\texttt{myenv-noAdd2.yaml}}{lst:myenv-noAdd2}
\end{minipage}

On applying either of the following commands,
\begin{commandshell}
latexindent.pl myenv.tex -l myenv-noAdd1.yaml  
latexindent.pl myenv.tex -l myenv-noAdd2.yaml  
\end{commandshell}
we obtain the output given in \cref{lst:myenv-output}; note in particular that the environment \texttt{myenv} 
has not received any \emph{additional} indentation, but that the \texttt{outer} environment \emph{has} still 
received indentation.

\cmhlistingsfromfile{demonstrations/myenvironment-simple-noAdd-body1.tex}{\texttt{myenv.tex output (using either \cref{lst:myenv-noAdd1,lst:myenv-noAdd2})}}{lst:myenv-output}

Upon changing the YAML files to those shown in \cref{lst:myenv-noAdd3,lst:myenv-noAdd4}, and running either
\begin{commandshell}
latexindent.pl myenv.tex -l myenv-noAdd3.yaml  
latexindent.pl myenv.tex -l myenv-noAdd4.yaml  
\end{commandshell}
we obtain the output given in \cref{lst:myenv-output-4}. 

\begin{minipage}{.45\textwidth}
\cmhlistingsfromfile[style=yaml-LST]{demonstrations/myenv-noAdd3.yaml}[width=.8\linewidth,before=\centering,yaml-TCB]{\texttt{myenv-noAdd3.yaml}}{lst:myenv-noAdd3}
\end{minipage}
\hfill
\begin{minipage}{.45\textwidth}
\cmhlistingsfromfile[style=yaml-LST]{demonstrations/myenv-noAdd4.yaml}[width=.8\linewidth,before=\centering,yaml-TCB]{\texttt{myenv-noAdd4.yaml}}{lst:myenv-noAdd4}
\end{minipage}

\cmhlistingsfromfile{demonstrations/myenvironment-simple-noAdd-body4.tex}{\texttt{myenv.tex output} (using either \cref{lst:myenv-noAdd3,lst:myenv-noAdd4})}{lst:myenv-output-4}

If we now allow \texttt{myenv} to have some optional and mandatory arguments, as in \cref{lst:myenv-args},
\cmhlistingsfromfile{demonstrations/myenvironment-args.tex}{\texttt{myenv-args.tex}}{lst:myenv-args}
then running
\begin{commandshell}
latexindent.pl -l=myenv-noAdd1.yaml myenv-args.tex  
\end{commandshell}
gives the output shown in \cref{lst:myenv-args-noAdd1}; note that the optional argument, mandatory argument and body \emph{all} 
have received no additional indent. This is because, when \texttt{noAdditionalIndent} is specified in `scalar' form (as in \cref{lst:myenv-noAdd1}), 
then \emph{all} parts of the environment (body, optional and mandatory arguments) are assumed to want no additional indent.
\cmhlistingsfromfile{demonstrations/myenvironment-args-noAdd-body1.tex}{\texttt{myenv-args.tex} using \cref{lst:myenv-noAdd1}}{lst:myenv-args-noAdd1}

We may customise \texttt{noAdditionalIndent} for optional and mandatory arguments of the \texttt{myenv} environment, as shown in, for example, \cref{lst:myenv-noAdd5,lst:myenv-noAdd6}.

\begin{minipage}{.49\textwidth}
\cmhlistingsfromfile[style=yaml-LST]{demonstrations/myenv-noAdd5.yaml}[width=.8\linewidth,before=\centering,yaml-TCB]{\texttt{myenv-noAdd5.yaml}}{lst:myenv-noAdd5}
\end{minipage}
\hfill
\begin{minipage}{.49\textwidth}
\cmhlistingsfromfile[style=yaml-LST]{demonstrations/myenv-noAdd6.yaml}[width=.8\linewidth,before=\centering,yaml-TCB]{\texttt{myenv-noAdd6.yaml}}{lst:myenv-noAdd6}
\end{minipage}

Upon running
\begin{commandshell}
latexindent.pl myenv.tex -l myenv-noAdd5.yaml  
latexindent.pl myenv.tex -l myenv-noAdd6.yaml  
\end{commandshell}
we obtain the respective outputs given in \cref{lst:myenv-args-noAdd5,lst:myenv-args-noAdd6}. Note that in \cref{lst:myenv-args-noAdd5} 
the text for the \emph{optional} argument has not received any additional indentation, and that in \cref{lst:myenv-args-noAdd6} the 
\emph{mandatory} argument has not received any additional indentation; in both cases, the \emph{body} has not received any additional indentation.

\begin{minipage}{.45\textwidth}
\cmhlistingsfromfile{demonstrations/myenvironment-args-noAdd5.tex}{\texttt{myenv-args.tex} using \cref{lst:myenv-noAdd5}}{lst:myenv-args-noAdd5}
\end{minipage}
\hfill
\begin{minipage}{.45\textwidth}
\cmhlistingsfromfile{demonstrations/myenvironment-args-noAdd6.tex}{\texttt{myenv-args.tex} using \cref{lst:myenv-noAdd6}}{lst:myenv-args-noAdd6}
\end{minipage}

\yamltitle{indentRules}*{horizontal space {\bfseries OR} fields} 
We may also specify indentation rules for environment code blocks using the \texttt{indentRules} field; see, for example,
\cref{lst:myenv-rules1,lst:myenv-rules2}.

\begin{minipage}{.45\textwidth}
\cmhlistingsfromfile[style=yaml-LST]{demonstrations/myenv-rules1.yaml}[width=.8\linewidth,before=\centering,yaml-TCB]{\texttt{myenv-rules1.yaml}}{lst:myenv-rules1}
\end{minipage}
\hfill
\begin{minipage}{.45\textwidth}
\cmhlistingsfromfile[style=yaml-LST]{demonstrations/myenv-rules2.yaml}[width=.8\linewidth,before=\centering,yaml-TCB]{\texttt{myenv-rules2.yaml}}{lst:myenv-rules2}
\end{minipage}

On applying either of the following commands,
\begin{commandshell}
latexindent.pl myenv.tex -l myenv-rules1.yaml  
latexindent.pl myenv.tex -l myenv-rules2.yaml  
\end{commandshell}
we obtain the output given in \cref{lst:myenv-rules-output}; note in particular that the environment \texttt{myenv} 
has not received any \emph{additional} indentation, but that the \texttt{outer} environment \emph{has} still 
received indentation.

\cmhlistingsfromfile{demonstrations/myenv-rules1.tex}{\texttt{myenv.tex output (using either \cref{lst:myenv-rules1,lst:myenv-rules2})}}{lst:myenv-rules-output}

If you specify a field in \texttt{indentRules} using anything other than horizontal space, it will be ignored.

Let's now return to the example in \cref{lst:myenv-args} that contains optional and mandatory arguments. Upon using \cref{lst:myenv-rules1} as in
\begin{commandshell}
latexindent.pl myenv-args.tex -l=myenv-rules1.yaml  
\end{commandshell}
we obtain the output in \cref{lst:myenv-args-rules1}; note that the body, optional argument and mandatory argument have \emph{all} 
received the same customised indentation.
\cmhlistingsfromfile{demonstrations/myenvironment-args-rules1.tex}{\texttt{myenv-args.tex} using \cref{lst:myenv-rules1}}{lst:myenv-args-rules1}

You can specify different indentation rules for the different features using, for example, \cref{lst:myenv-rules3,lst:myenv-rules4}

\begin{minipage}{.49\textwidth}
\cmhlistingsfromfile[style=yaml-LST]{demonstrations/myenv-rules3.yaml}[width=.8\linewidth,before=\centering,yaml-TCB]{\texttt{myenv-rules3.yaml}}{lst:myenv-rules3}
\end{minipage}
\hfill
\begin{minipage}{.49\textwidth}
\cmhlistingsfromfile[style=yaml-LST]{demonstrations/myenv-rules4.yaml}[width=.9\linewidth,before=\centering,yaml-TCB]{\texttt{myenv-rules4.yaml}}{lst:myenv-rules4}
\end{minipage}

After running
\begin{commandshell}
latexindent.pl myenv-args.tex -l myenv-rules3.yaml  
latexindent.pl myenv-args.tex -l myenv-rules4.yaml  
\end{commandshell}
then we obtain the respective outputs given in \cref{lst:myenv-args-rules3,lst:myenv-args-rules4}.

\begin{minipage}{.45\textwidth}
\cmhlistingsfromfile{demonstrations/myenvironment-args-rules3.tex}{\texttt{myenv-args.tex} using \cref{lst:myenv-rules3}}{lst:myenv-args-rules3}
\end{minipage}
\hfill
\begin{minipage}{.45\textwidth}
\cmhlistingsfromfile{demonstrations/myenvironment-args-rules4.tex}{\texttt{myenv-args.tex} using \cref{lst:myenv-rules4}}{lst:myenv-args-rules4}
\end{minipage}

Note that in \cref{lst:myenv-args-rules3}, the optional argument has only received a single space of indentation, while the mandatory argument 
has received the default (tab) indentation; the environment body has received three spaces of indentation.

In \cref{lst:myenv-args-rules4}, the optional argument has received the default (tab) indentation, the mandatory argument has received two tabs
of indentation, and the body has received three spaces of indentation.
