% arara: pdflatex: { files: [latexindent]}
\subsubsection{Commands with arguments}\label{subsubsec:commands-arguments}
	Let's begin with the simple example in \cref{lst:mycommand}; when \texttt{latexindent.pl}
	operates on this file, the default output is shown in \cref{lst:mycommand-default}.
	\footnote{The command code blocks have quite a few subtleties, described in
	\vref{subsec:commands-string-between}.}

	\begin{cmhtcbraster}[raster column skip=.1\linewidth]
		\cmhlistingsfromfile{demonstrations/mycommand.tex}{\texttt{mycommand.tex}}{lst:mycommand}
		\cmhlistingsfromfile{demonstrations/mycommand-default.tex}{\texttt{mycommand.tex} default output}{lst:mycommand-default}
	\end{cmhtcbraster}

	As in the environment-based case (see \vref{lst:myenv-noAdd1,lst:myenv-noAdd2}) we may
	specify \texttt{noAdditionalIndent} either in `scalar' form, or in `field' form, as shown
	in \cref{lst:mycommand-noAdd1,lst:mycommand-noAdd2}

	\begin{minipage}{.45\textwidth}
		\cmhlistingsfromfile[style=yaml-LST]{demonstrations/mycommand-noAdd1.yaml}[width=.8\linewidth,before=\centering,yaml-TCB]{\texttt{mycommand-noAdd1.yaml}}{lst:mycommand-noAdd1}
	\end{minipage}
	\hfill
	\begin{minipage}{.45\textwidth}
		\cmhlistingsfromfile[style=yaml-LST]{demonstrations/mycommand-noAdd2.yaml}[width=.8\linewidth,before=\centering,yaml-TCB]{\texttt{mycommand-noAdd2.yaml}}{lst:mycommand-noAdd2}
	\end{minipage}

	After running the following commands, \index{switches!-l demonstration}
	\begin{commandshell}
latexindent.pl mycommand.tex -l mycommand-noAdd1.yaml  
latexindent.pl mycommand.tex -l mycommand-noAdd2.yaml  
\end{commandshell}
	we receive the respective output given in
	\cref{lst:mycommand-output-noAdd1,lst:mycommand-output-noAdd2}

	\begin{minipage}{.45\textwidth}
		\cmhlistingsfromfile{demonstrations/mycommand-noAdd1.tex}{\texttt{mycommand.tex} using \cref{lst:mycommand-noAdd1}}{lst:mycommand-output-noAdd1}
	\end{minipage}
	\hfill
	\begin{minipage}{.45\textwidth}
		\cmhlistingsfromfile{demonstrations/mycommand-noAdd2.tex}{\texttt{mycommand.tex} using \cref{lst:mycommand-noAdd2}}{lst:mycommand-output-noAdd2}
	\end{minipage}

	Note that in \cref{lst:mycommand-output-noAdd1} that the `body', optional argument
	\emph{and} mandatory argument have \emph{all} received no additional indentation, while
	in \cref{lst:mycommand-output-noAdd2}, only the `body' has not received any additional
	indentation. We define the `body' of a command as any lines following the command name
	that include its optional or mandatory arguments.

	We may further customise \texttt{noAdditionalIndent} for \texttt{mycommand} as we did in
	\vref{lst:myenv-noAdd5,lst:myenv-noAdd6}; explicit examples are given in
	\cref{lst:mycommand-noAdd3,lst:mycommand-noAdd4}.

	\begin{minipage}{.45\textwidth}
		\cmhlistingsfromfile[style=yaml-LST]{demonstrations/mycommand-noAdd3.yaml}[width=.9\linewidth,before=\centering,yaml-TCB]{\texttt{mycommand-noAdd3.yaml}}{lst:mycommand-noAdd3}
	\end{minipage}
	\hfill
	\begin{minipage}{.45\textwidth}
		\cmhlistingsfromfile[style=yaml-LST]{demonstrations/mycommand-noAdd4.yaml}[width=.9\linewidth,before=\centering,yaml-TCB]{\texttt{mycommand-noAdd4.yaml}}{lst:mycommand-noAdd4}
	\end{minipage}

	After running the following commands, \index{switches!-l demonstration}
	\begin{commandshell}
latexindent.pl mycommand.tex -l mycommand-noAdd3.yaml  
latexindent.pl mycommand.tex -l mycommand-noAdd4.yaml  
\end{commandshell}
	we receive the respective output given in
	\cref{lst:mycommand-output-noAdd3,lst:mycommand-output-noAdd4}.

	\begin{minipage}{.45\textwidth}
		\cmhlistingsfromfile{demonstrations/mycommand-noAdd3.tex}{\texttt{mycommand.tex} using \cref{lst:mycommand-noAdd3}}{lst:mycommand-output-noAdd3}
	\end{minipage}
	\hfill
	\begin{minipage}{.45\textwidth}
		\cmhlistingsfromfile{demonstrations/mycommand-noAdd4.tex}{\texttt{mycommand.tex} using \cref{lst:mycommand-noAdd4}}{lst:mycommand-output-noAdd4}
	\end{minipage}

	Attentive readers will note that the body of \texttt{mycommand} in both
	\cref{lst:mycommand-output-noAdd3,lst:mycommand-output-noAdd4} has received no additional
	indent, even though \texttt{body} is explicitly set to \texttt{0} in both
	\cref{lst:mycommand-noAdd3,lst:mycommand-noAdd4}. This is because, by default,
	\texttt{noAdditionalIndentGlobal} for \texttt{commands} is set to \texttt{1} by default;
	this can be easily fixed as in
	\cref{lst:mycommand-noAdd5,lst:mycommand-noAdd6}.\label{page:command:noAddGlobal}

	\begin{minipage}{.45\textwidth}
		\cmhlistingsfromfile[style=yaml-LST]{demonstrations/mycommand-noAdd5.yaml}[width=.9\linewidth,before=\centering,yaml-TCB]{\texttt{mycommand-noAdd5.yaml}}{lst:mycommand-noAdd5}
	\end{minipage}
	\hfill
	\begin{minipage}{.45\textwidth}
		\cmhlistingsfromfile[style=yaml-LST]{demonstrations/mycommand-noAdd6.yaml}[width=.9\linewidth,before=\centering,yaml-TCB]{\texttt{mycommand-noAdd6.yaml}}{lst:mycommand-noAdd6}
	\end{minipage}

	After running the following commands, \index{switches!-l demonstration}
	\begin{commandshell}
latexindent.pl mycommand.tex -l mycommand-noAdd5.yaml  
latexindent.pl mycommand.tex -l mycommand-noAdd6.yaml  
\end{commandshell}
	we receive the respective output given in
	\cref{lst:mycommand-output-noAdd5,lst:mycommand-output-noAdd6}.

	\begin{minipage}{.45\textwidth}
		\cmhlistingsfromfile{demonstrations/mycommand-noAdd5.tex}{\texttt{mycommand.tex} using \cref{lst:mycommand-noAdd5}}{lst:mycommand-output-noAdd5}
	\end{minipage}
	\hfill
	\begin{minipage}{.45\textwidth}
		\cmhlistingsfromfile{demonstrations/mycommand-noAdd6.tex}{\texttt{mycommand.tex} using \cref{lst:mycommand-noAdd6}}{lst:mycommand-output-noAdd6}
	\end{minipage}

	Both \texttt{indentRules} and \texttt{indentRulesGlobal} can be adjusted as they were for
	\emph{environment} code blocks, as in \vref{lst:myenv-rules3,lst:myenv-rules4} and
	\vref{lst:indentRulesGlobal:environments,lst:opt-args-indent-rules-glob,lst:mand-args-indent-rules-glob}.
