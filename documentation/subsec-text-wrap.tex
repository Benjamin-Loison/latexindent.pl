% arara: pdflatex: { files: [latexindent]}
\subsection{Text Wrapping}\label{subsec:textwrapping}
 \announce{2022-03-13}{text wrap overhaul}\emph{The text wrapping routine has been over-hauled as
 of V3.16; I hope that the interface is simpler, and most importantly, the results are
 better}.

 The complete settings for this feature are given in \cref{lst:textWrapOptionsAll}.

 \cmhlistingsfromfile[style=textWrapOptionsAll]{../defaultSettings.yaml}[MLB-TCB,width=.95\linewidth,before=\centering]{\texttt{textWrapOptions}}{lst:textWrapOptionsAll}

\subsubsection{Text wrap: overview}
 An overview of how the text wrapping feature works:
 \begin{enumerate}
  \item the default value of \texttt{columns} is 0, which means that text wrapping will
        \emph{not} happen by default;
  \item it happens \emph{after} verbatim blocks have been found;
  \item it happens \emph{after} the oneSentencePerLine routine (see
        \cref{sec:onesentenceperline});
  \item it can happen \emph{before} or \emph{after} \announce*{2023-01-01}{text wrap:
        before/after} all of the other code blocks are found and does \emph{not} operate
        on a per-code-block basis; when using \texttt{before} this means that, including
        indentation, you may receive a column width wider than that which you specify in
        \texttt{columns}, and in which case you probably wish to explore \texttt{after}
        in \cref{subsubsec:tw:before:after};
  \item code blocks to be text wrapped will:
        \begin{enumerate}
         \item \emph{follow} the fields specified in \texttt{blocksFollow}
         \item \emph{begin} with the fields specified in \texttt{blocksBeginWith}
         \item \emph{end} before the fields specified in \texttt{blocksEndBefore}
        \end{enumerate}
  \item setting \texttt{columns} to a value $>0$ will text wrap blocks by first removing
        line breaks, and then wrapping according to the specified value of
        \texttt{columns};
  \item setting \texttt{columns} to $-1$ will \emph{only} remove line breaks within the
        text wrap block;
  \item by default, the text wrapping routine will remove line breaks within text blocks
        because \texttt{removeBlockLineBreaks} is set to 1; switch it to 0 if you wish to
        change this;
  \item about trailing comments within text wrap blocks:
        \begin{enumerate}
         \item trailing comments that do \emph{not} have leading space instruct the text
               wrap routine to connect the lines \emph{without} space (see
               \cref{lst:tw-tc2});
         \item multiple trailing comments will be connected at the end of the text wrap
               block (see \cref{lst:tw-tc4});
         \item the number of spaces between the end of the text wrap block and the
               (possibly combined) trailing comments is determined by the spaces (if any)
               at the end of the text wrap block (see \cref{lst:tw-tc5});
        \end{enumerate}
  \item trailing comments can receive text wrapping \announce*{2023-01-01}{text wrap
        trailing comments}; examples are shown in \cref{subsubsec:tw:comments} and
        \cref{subsubsec:ospl:tw:comments}.
 \end{enumerate}

 We demonstrate this feature using a series of examples.

\subsubsection{Text wrap: simple examples}\label{subsec:textwrapping-quick-start}

 \begin{example}
 Let's use the sample text given in \cref{lst:textwrap1}. \index{text wrap!quick start}

 \cmhlistingsfromfile{demonstrations/textwrap1.tex}{\texttt{textwrap1.tex}}{lst:textwrap1}

 We will change the value of \texttt{columns} in \cref{lst:textwrap1-yaml} and then run
 the command

 \begin{commandshell}
latexindent.pl -m -l textwrap1.yaml textwrap1.tex
\end{commandshell}

 then we receive the output given in \cref{lst:textwrap1-mod1}.

 \begin{cmhtcbraster}[raster column skip=.1\linewidth]
  \cmhlistingsfromfile{demonstrations/textwrap1-mod1.tex}{\texttt{textwrap1-mod1.tex}}{lst:textwrap1-mod1}
  \cmhlistingsfromfile[style=yaml-LST]{demonstrations/textwrap1.yaml}[MLB-TCB]{\texttt{textwrap1.yaml}}{lst:textwrap1-yaml}
 \end{cmhtcbraster}
 \end{example}

 \begin{example}
 If we set \texttt{columns} to $-1$ then \texttt{latexindent.pl} remove line breaks
 within the text wrap block, and will \emph{not} perform text wrapping. We can use this
 to undo text wrapping. \index{text wrap!setting columns to -1}

 Starting from the file in \cref{lst:textwrap1-mod1} and using the settings in
 \cref{lst:textwrap1A-yaml}

 \cmhlistingsfromfile[style=yaml-LST]{demonstrations/textwrap1A.yaml}[MLB-TCB]{\texttt{textwrap1A.yaml}}{lst:textwrap1A-yaml}

 and running

 \begin{commandshell}
latexindent.pl -m -l textwrap1A.yaml textwrap1-mod1.tex
\end{commandshell}

 gives the output in \cref{lst:textwrap1-mod1A}.

 \cmhlistingsfromfile{demonstrations/textwrap1-mod1A.tex}{\texttt{textwrap1-mod1A.tex}}{lst:textwrap1-mod1A}
 \end{example}

 \begin{example}
 By default, the text wrapping routine will convert multiple spaces into single spaces.
 You can change this behaviour by flicking the switch \texttt{multipleSpacesToSingle}
 which we have done in \cref{lst:textwrap1B-yaml}

 Using the settings in \cref{lst:textwrap1B-yaml} and running

 \begin{commandshell}
latexindent.pl -m -l textwrap1B.yaml textwrap1-mod1.tex
\end{commandshell}

 gives the output in \cref{lst:textwrap1-mod1B}.
 \begin{cmhtcbraster}[raster column skip=.1\linewidth]
  \cmhlistingsfromfile[showspaces=true]{demonstrations/textwrap1-mod1B.tex}{\texttt{textwrap1-mod1B.tex}}{lst:textwrap1-mod1B}
  \cmhlistingsfromfile[style=yaml-LST]{demonstrations/textwrap1B.yaml}[MLB-TCB]{\texttt{textwrap1B.yaml}}{lst:textwrap1B-yaml}
 \end{cmhtcbraster}
 We note that in \cref{lst:textwrap1-mod1B} the multiple spaces have \emph{not} been
 condensed into single spaces.
 \end{example}

\subsubsection{Text wrap: \texttt{blocksFollow} examples}
 We examine the \texttt{blocksFollow} field of \cref{lst:textWrapOptionsAll}. \index{text
 wrap!blocksFollow}

 \begin{example}
 Let's use the sample text given in \cref{lst:tw-headings1}. \index{text
 wrap!blocksFollow!headings}

 \cmhlistingsfromfile{demonstrations/tw-headings1.tex}{\texttt{tw-headings1.tex}}{lst:tw-headings1}

 We note that \cref{lst:tw-headings1} contains the heading commands \texttt{section} and
 \texttt{subsection}. Upon running the command

 \begin{commandshell}
latexindent.pl -m -l textwrap1.yaml tw-headings1.tex
\end{commandshell}

 then we receive the output given in \cref{lst:tw-headings1-mod1}.

 \cmhlistingsfromfile{demonstrations/tw-headings1-mod1.tex}{\texttt{tw-headings1-mod1.tex}}{lst:tw-headings1-mod1}

 We reference \vref{lst:textWrapOptionsAll} and also \vref{lst:indentAfterHeadings}:
 \begin{itemize}
  \item in \cref{lst:textWrapOptionsAll} the \texttt{headings} field is set to
        \texttt{1}, which instructs \texttt{latexindent.pl} to read the fields from
        \vref{lst:indentAfterHeadings}, \emph{regardless of the value of
        indentAfterThisHeading or level};
  \item the default is to assume that the heading command can, optionally, be followed by
        a \texttt{label} command.
 \end{itemize}
 If you find scenarios in which the default value of \texttt{headings} does not work,
 then you can explore the \texttt{other} field.

 We can turn off \texttt{headings} as in \cref{lst:bf-no-headings-yaml} and then run

 \begin{commandshell}
latexindent.pl -m -l textwrap1.yaml,bf-no-headings.yaml tw-headings1.tex
\end{commandshell}

 gives the output in \cref{lst:tw-headings1-mod2}, in which text wrapping has been
 instructed \emph{not to happen} following headings.
 \begin{cmhtcbraster}
  \cmhlistingsfromfile{demonstrations/tw-headings1-mod2.tex}{\texttt{tw-headings1-mod2.tex}}{lst:tw-headings1-mod2}
  \cmhlistingsfromfile[style=yaml-LST]{demonstrations/bf-no-headings.yaml}[MLB-TCB]{\texttt{bf-no-headings.yaml}}{lst:bf-no-headings-yaml}
 \end{cmhtcbraster}
 \end{example}

 \begin{example}
 Let's use the sample text given in \cref{lst:tw-comments1}. \index{text
 wrap!blocksFollow!comments}

 \cmhlistingsfromfile{demonstrations/tw-comments1.tex}{\texttt{tw-comments1.tex}}{lst:tw-comments1}

 We note that \cref{lst:tw-comments1} contains trailing comments. Upon running the
 command

 \begin{commandshell}
latexindent.pl -m -l textwrap1.yaml tw-comments1.tex
\end{commandshell}

 then we receive the output given in \cref{lst:tw-comments1-mod1}.

 \cmhlistingsfromfile{demonstrations/tw-comments1-mod1.tex}{\texttt{tw-comments1-mod1.tex}}{lst:tw-comments1-mod1}

 With reference to \vref{lst:textWrapOptionsAll} the \texttt{commentOnPreviousLine} field
 is set to \texttt{1}, which instructs \texttt{latexindent.pl} to find text wrap blocks
 after a comment on its own line.

 We can turn off \texttt{comments} as in \cref{lst:bf-no-comments-yaml} and then run

 \begin{commandshell}
latexindent.pl -m -l textwrap1.yaml,bf-no-comments.yaml tw-comments1.tex
\end{commandshell}

 gives the output in \cref{lst:tw-comments1-mod2}, in which text wrapping has been
 instructed \emph{not to happen} following comments on their own line.
 \begin{cmhtcbraster}[raster column skip=.1\linewidth,
   raster left skip=-3.5cm,
   raster right skip=-2cm,
  ]
  \cmhlistingsfromfile{demonstrations/tw-comments1-mod2.tex}{\texttt{tw-comments1-mod2.tex}}{lst:tw-comments1-mod2}
  \cmhlistingsfromfile[style=yaml-LST]{demonstrations/bf-no-comments.yaml}[MLB-TCB]{\texttt{bf-no-comments.yaml}}{lst:bf-no-comments-yaml}
 \end{cmhtcbraster}
 \end{example}

 Referencing \vref{lst:textWrapOptionsAll} the \texttt{blocksFollow} fields \texttt{par},
 \texttt{blankline}, \texttt{verbatim} and \texttt{filecontents} fields operate in
 analogous ways to those demonstrated in the above.

 The \texttt{other} field of the \texttt{blocksFollow} can either be \texttt{0} (turned
 off) or set as a regular expression. The default value is set to
 \lstinline!\\\]|\\item(?:\h|\[)! which can be translated to \emph{backslash followed by
 a square bracket} or \emph{backslash item followed by horizontal space or a square
 bracket}, or in other words, \emph{end of display math} or an item command.

 \begin{example}
 Let's use the sample text given in \cref{lst:tw-disp-math1}. \index{text
 wrap!blocksFollow!other} \index{regular expressions!text wrap!blocksFollow}

 \cmhlistingsfromfile{demonstrations/tw-disp-math1.tex}{\texttt{tw-disp-math1.tex}}{lst:tw-disp-math1}

 We note that \cref{lst:tw-disp-math1} contains display math. Upon running the command

 \begin{commandshell}
latexindent.pl -m -l textwrap1.yaml tw-disp-math1.tex
\end{commandshell}

 then we receive the output given in \cref{lst:tw-disp-math1-mod1}.

 \cmhlistingsfromfile{demonstrations/tw-disp-math1-mod1.tex}{\texttt{tw-disp-math1-mod1.tex}}{lst:tw-disp-math1-mod1}

 With reference to \vref{lst:textWrapOptionsAll} the \texttt{other} field is set to
 \lstinline!\\\]!, which instructs \texttt{latexindent.pl} to find text wrap blocks after
 the end of display math.

 We can turn off this switch as in \cref{lst:bf-no-disp-math-yaml} and then run
 \begin{widepage}

  \begin{commandshell}
latexindent.pl -m -l textwrap1.yaml,bf-no-disp-math.yaml tw-disp-math1.tex
\end{commandshell}

 \end{widepage}
 gives the output in \cref{lst:tw-disp-math1-mod2}, in which text wrapping has been
 instructed \emph{not to happen} following display math.
 \begin{cmhtcbraster}[raster column skip=.1\linewidth]
  \cmhlistingsfromfile{demonstrations/tw-disp-math1-mod2.tex}{\texttt{tw-disp-math1-mod2.tex}}{lst:tw-disp-math1-mod2}
  \cmhlistingsfromfile[style=yaml-LST]{demonstrations/bf-no-disp-math.yaml}[MLB-TCB]{\texttt{bf-no-disp-math.yaml}}{lst:bf-no-disp-math-yaml}
 \end{cmhtcbraster}

 Naturally, you should feel encouraged to customise this as you see fit.
 \end{example}

 The \texttt{blocksFollow} field \emph{deliberately} does not default to allowing text
 wrapping to occur after \texttt{begin environment} statements. You are encouraged to
 customize the \texttt{other} field to accomodate the environments that you would like to
 text wrap individually, as in the next example.

 \begin{example}
 Let's use the sample text given in \cref{lst:tw-bf-myenv1}. \index{text
 wrap!blocksFollow!other} \index{regular expressions!text wrap!blocksFollow}

 \cmhlistingsfromfile{demonstrations/tw-bf-myenv1.tex}{\texttt{tw-bf-myenv1.tex}}{lst:tw-bf-myenv1}

 We note that \cref{lst:tw-bf-myenv1} contains \texttt{myenv} environment. Upon running
 the command

 \begin{commandshell}
latexindent.pl -m -l textwrap1.yaml tw-bf-myenv1.tex
\end{commandshell}

 then we receive the output given in \cref{lst:tw-bf-myenv1-mod1}.

 \cmhlistingsfromfile{demonstrations/tw-bf-myenv1-mod1.tex}{\texttt{tw-bf-myenv1-mod1.tex}}{lst:tw-bf-myenv1-mod1}

 We note that we have \emph{not} received much text wrapping. We can turn do better by
 employing \cref{lst:tw-bf-myenv-yaml} and then run

 \begin{commandshell}
latexindent.pl -m -l textwrap1.yaml,tw-bf-myenv.yaml tw-bf-myenv1.tex
\end{commandshell}

 which gives the output in \cref{lst:tw-bf-myenv1-mod2}, in which text wrapping has been
 implemented across the file.
 \begin{cmhtcbraster}[raster column skip=.01\linewidth,
   raster left skip=-1.5cm,
   raster force size=false,
   raster column 1/.style={add to width=-.1\textwidth},
   raster column 2/.style={add to width=.1\textwidth},
  ]
  \cmhlistingsfromfile{demonstrations/tw-bf-myenv1-mod2.tex}{\texttt{tw-bf-myenv1-mod2.tex}}{lst:tw-bf-myenv1-mod2}
  \cmhlistingsfromfile[style=yaml-LST]{demonstrations/tw-bf-myenv.yaml}[MLB-TCB,width=0.6\linewidth]{\texttt{tw-bf-myenv.yaml}}{lst:tw-bf-myenv-yaml}
 \end{cmhtcbraster}

 \end{example}

\subsubsection{Text wrap: \texttt{blocksBeginWith} examples}
 We examine the \texttt{blocksBeginWith} field of \cref{lst:textWrapOptionsAll} with a
 series of examples. \index{text wrap!blocksBeginWith}

 \begin{example}
 By default, text wrap blocks can begin with the characters \texttt{a-z} and
 \texttt{A-Z}.

 If we start with the file given in \cref{lst:tw-0-9}
 \cmhlistingsfromfile{demonstrations/tw-0-9.tex}{\texttt{tw-0-9.tex}}{lst:tw-0-9}
 and run the command

 \begin{commandshell}
latexindent.pl -m -l textwrap1.yaml tw-0-9.tex
\end{commandshell}

 then we receive the output given in \cref{lst:tw-0-9-mod1} in which text wrapping has
 \emph{not} occured.
 \cmhlistingsfromfile{demonstrations/tw-0-9-mod1.tex}{\texttt{tw-0-9-mod1.tex}}{lst:tw-0-9-mod1}

 We can allow paragraphs to begin with \texttt{0-9} characters by using the settings in
 \cref{lst:bb-0-9-yaml} and running

 \begin{commandshell}
latexindent.pl -m -l textwrap1.yaml,bb-0-9-yaml tw-0-9.tex
\end{commandshell}

 gives the output in \cref{lst:tw-0-9-mod2}, in which text wrapping \emph{has} happened.
 \begin{cmhtcbraster}[raster column skip=.1\linewidth,]
  \cmhlistingsfromfile{demonstrations/tw-0-9-mod2.tex}{\texttt{tw-0-9-mod2.tex}}{lst:tw-0-9-mod2}
  \cmhlistingsfromfile[style=yaml-LST]{demonstrations/bb-0-9.yaml}[MLB-TCB]{\texttt{bb-0-9.yaml.yaml}}{lst:bb-0-9-yaml}
 \end{cmhtcbraster}
 \end{example}

 \begin{example}
 Let's now use the file given in \cref{lst:tw-bb-announce1}
 \cmhlistingsfromfile{demonstrations/tw-bb-announce1.tex}{\texttt{tw-bb-announce1.tex}}{lst:tw-bb-announce1}
 and run the command

 \begin{commandshell}
latexindent.pl -m -l textwrap1.yaml tw-bb-announce1.tex
\end{commandshell}

 then we receive the output given in \cref{lst:tw-bb-announce1-mod1} in which text
 wrapping has \emph{not} occured.

 \cmhlistingsfromfile{demonstrations/tw-bb-announce1-mod1.tex}{\texttt{tw-bb-announce1-mod1.tex}}{lst:tw-bb-announce1-mod1}

 We can allow \lstinline!\announce! to be at the beginning of paragraphs by using the
 settings in \cref{lst:tw-bb-announce-yaml} and running
 \begin{widepage}

  \begin{commandshell}
latexindent.pl -m -l textwrap1.yaml,tw-bb-announce.yaml tw-bb-announce1.tex
\end{commandshell}

 \end{widepage}
 gives the output in \cref{lst:tw-bb-announce1-mod2}, in which text wrapping \emph{has}
 happened.
 \begin{cmhtcbraster}[raster column skip=.1\linewidth,]
  \cmhlistingsfromfile{demonstrations/tw-bb-announce1-mod2.tex}{\texttt{tw-bb-announce1-mod2.tex}}{lst:tw-bb-announce1-mod2}
  \cmhlistingsfromfile[style=yaml-LST]{demonstrations/tw-bb-announce.yaml}[MLB-TCB]{\texttt{tw-bb-announce.yaml}}{lst:tw-bb-announce-yaml}
 \end{cmhtcbraster}

 \end{example}

\subsubsection{Text wrap: \texttt{blocksEndBefore} examples}
 We examine the \texttt{blocksEndBefore} field of \cref{lst:textWrapOptionsAll} with a
 series of examples. \index{text wrap!blocksEndBefore}

 \begin{example}
 Let's use the sample text given in \cref{lst:tw-be-equation}. \index{text
 wrap!blocksFollow!other} \index{regular expressions!text wrap!blocksFollow}

 \cmhlistingsfromfile{demonstrations/tw-be-equation.tex}{\texttt{tw-be-equation.tex}}{lst:tw-be-equation}

 We note that \cref{lst:tw-be-equation} contains an environment. Upon running the command

 \begin{commandshell}
latexindent.pl -m -l textwrap1A.yaml tw-be-equation.tex
\end{commandshell}

 then we receive the output given in \cref{lst:tw-be-equation-mod1}.

 \cmhlistingsfromfile{demonstrations/tw-be-equation-mod1.tex}{\texttt{tw-be-equation-mod1.tex}}{lst:tw-be-equation-mod1}

 With reference to \vref{lst:textWrapOptionsAll} the \texttt{other} field is set to
 \lstinline!\\begin\{|\\\[|\\end\{!, which instructs \texttt{latexindent.pl} to
 \emph{stop} text wrap blocks before \texttt{begin} statements, display math, and
 \texttt{end} statements.

 We can turn off this switch as in \cref{lst:tw-be-equation-yaml} and then run
 \begin{widepage}

  \begin{commandshell}
latexindent.pl -m -l textwrap1A.yaml,tw-be-equation.yaml tw-be-equation.tex
\end{commandshell}

 \end{widepage}
 gives the output in \cref{lst:tw-be-equation-mod2}, in which text wrapping has been
 instructed \emph{not} to stop at these statements.

 \cmhlistingsfromfile[style=yaml-LST]{demonstrations/tw-be-equation.yaml}[MLB-TCB]{\texttt{tw-be-equation.yaml}}{lst:tw-be-equation-yaml}

 \begin{widepage}
  \cmhlistingsfromfile{demonstrations/tw-be-equation-mod2.tex}{\texttt{tw-be-equation-mod2.tex}}{lst:tw-be-equation-mod2}
 \end{widepage}

 Naturally, you should feel encouraged to customise this as you see fit.
 \end{example}

\subsubsection{Text wrap: trailing comments and spaces}
 We explore the behaviour of the text wrap routine in relation to trailing comments using
 the following examples.

 \begin{example}
 The file in \cref{lst:tw-tc1} contains a trailing comment which \emph{does} have a space
 infront of it.

 Running the command

 \begin{commandshell}
latexindent.pl -m tw-tc1.tex -l textwrap1A.yaml -o=+-mod1 
\end{commandshell}

 gives the output given in \cref{lst:tw-tc1-mod1}.

 \begin{cmhtcbraster}[raster column skip=.1\linewidth]
  \cmhlistingsfromfile[showspaces=true]{demonstrations/tw-tc1.tex}{\texttt{tw-tc1.tex}}{lst:tw-tc1}
  \cmhlistingsfromfile{demonstrations/tw-tc1-mod1.tex}{\texttt{tw-tc1-mod1.tex}}{lst:tw-tc1-mod1}
 \end{cmhtcbraster}
 \end{example}

 \begin{example}
 The file in \cref{lst:tw-tc2} contains a trailing comment which does \emph{not} have a
 space infront of it.

 Running the command

 \begin{commandshell}
latexindent.pl -m tw-tc2.tex -l textwrap1A.yaml -o=+-mod1 
\end{commandshell}

 gives the output in \cref{lst:tw-tc2-mod1}.
 \begin{cmhtcbraster}[raster column skip=.1\linewidth]
  \cmhlistingsfromfile{demonstrations/tw-tc2.tex}{\texttt{tw-tc2.tex}}{lst:tw-tc2}
  \cmhlistingsfromfile{demonstrations/tw-tc2-mod1.tex}{\texttt{tw-tc2-mod1.tex}}{lst:tw-tc2-mod1}
 \end{cmhtcbraster}
 We note that, because there is \emph{not} a space before the trailing comment, that the
 lines have been joined \emph{without} a space.
 \end{example}

 \begin{example}
 The file in \cref{lst:tw-tc3} contains multiple trailing comments.

 Running the command

 \begin{commandshell}
latexindent.pl -m tw-tc3.tex -l textwrap1A.yaml -o=+-mod1 
\end{commandshell}

 gives the output in \cref{lst:tw-tc3-mod1}.
 \begin{cmhtcbraster}[raster column skip=.1\linewidth]
  \cmhlistingsfromfile{demonstrations/tw-tc3.tex}{\texttt{tw-tc3.tex}}{lst:tw-tc3}
  \cmhlistingsfromfile{demonstrations/tw-tc3-mod1.tex}{\texttt{tw-tc3-mod1.tex}}{lst:tw-tc3-mod1}
 \end{cmhtcbraster}
 \end{example}

 \begin{example}
 The file in \cref{lst:tw-tc4} contains multiple trailing comments.

 Running the command

 \begin{commandshell}
latexindent.pl -m tw-tc4.tex -l textwrap1A.yaml -o=+-mod1 
\end{commandshell}

 gives the output in \cref{lst:tw-tc4-mod1}.
 \begin{cmhtcbraster}[raster column skip=.1\linewidth]
  \cmhlistingsfromfile{demonstrations/tw-tc4.tex}{\texttt{tw-tc4.tex}}{lst:tw-tc4}
  \cmhlistingsfromfile{demonstrations/tw-tc4-mod1.tex}{\texttt{tw-tc4-mod1.tex}}{lst:tw-tc4-mod1}
 \end{cmhtcbraster}
 \end{example}

 \begin{example}
 The file in \cref{lst:tw-tc5} contains multiple trailing comments.

 Running the command

 \begin{commandshell}
latexindent.pl -m tw-tc5.tex -l textwrap1A.yaml -o=+-mod1 
\end{commandshell}

 gives the output in \cref{lst:tw-tc5-mod1}.
 \begin{cmhtcbraster}[raster column skip=.1\linewidth]
  \cmhlistingsfromfile[showspaces=true]{demonstrations/tw-tc5.tex}{\texttt{tw-tc5.tex}}{lst:tw-tc5}
  \cmhlistingsfromfile[showspaces=true]{demonstrations/tw-tc5-mod1.tex}{\texttt{tw-tc5-mod1.tex}}{lst:tw-tc5-mod1}
 \end{cmhtcbraster}
 The space at the end of the text block has been preserved.
 \end{example}

 \begin{example}
 The file in \cref{lst:tw-tc6} contains multiple trailing comments.

 Running the command

 \begin{commandshell}
latexindent.pl -m tw-tc6.tex -l textwrap1A.yaml -o=+-mod1 
\end{commandshell}

 gives the output in \cref{lst:tw-tc6-mod1}.
 \begin{cmhtcbraster}[raster column skip=.1\linewidth]
  \cmhlistingsfromfile[showspaces=true]{demonstrations/tw-tc6.tex}{\texttt{tw-tc6.tex}}{lst:tw-tc6}
  \cmhlistingsfromfile[showspaces=true]{demonstrations/tw-tc6-mod1.tex}{\texttt{tw-tc6-mod1.tex}}{lst:tw-tc6-mod1}
 \end{cmhtcbraster}
 The space at the end of the text block has been preserved.
 \end{example}

\subsubsection{Text wrap: when before/after}\label{subsubsec:tw:before:after}
 The text wrapping routine operates, by default, \texttt{before} the
 \announce*{2023-01-01}{text wrap: before/after details} code blocks have been found, but
 this can be changed to \texttt{after}:
 \begin{itemize}
  \item \texttt{before} means it is likely that the columns of wrapped text may \emph{exceed} the
        value specified in \texttt{columns};
  \item \texttt{after} means it columns of wrapped text should \emph{not} exceed the value
        specified in \texttt{columns}.
 \end{itemize}
 We demonstrate this in the following examples. See also
 \cref{subsubsec:ospl:before:after}.

 \begin{example}
 Let's begin with the file in \cref{lst:textwrap8}.

 \cmhlistingsfromfile*{demonstrations/textwrap8.tex}{\texttt{textwrap8.tex}}{lst:textwrap8}

 Using the settings given in \cref{lst:tw-before1} and running the command

 \begin{commandshell}
latexindent.pl textwrap8.tex -o=+-mod1.tex -l=tw-before1.yaml -m
   \end{commandshell}

 gives the output given in \cref{lst:textwrap8-mod1}.

 \begin{cmhtcbraster}[raster columns=2,
   raster left skip=-1.5cm,
   raster right skip=-0cm,
   raster column skip=.06\linewidth]
  \cmhlistingsfromfile*{demonstrations/textwrap8-mod1.tex}{\texttt{textwrap8-mod1.tex}}{lst:textwrap8-mod1}
  \cmhlistingsfromfile*[style=yaml-LST]{demonstrations/tw-before1.yaml}[MLB-TCB,width=0.4\linewidth]{\texttt{tw-before1.yaml}}{lst:tw-before1}
 \end{cmhtcbraster}

 We note that, in \cref{lst:textwrap8-mod1}, that the wrapped text has \emph{exceeded}
 the specified value of \texttt{columns} (35) given in \cref{lst:tw-before1}. We can
 affect this by changing \texttt{when}; we explore this next.
 \end{example}

 \begin{example}
 We continue working with \cref{lst:textwrap8}.

 Using the settings given in \cref{lst:tw-after1} and running the command

 \begin{commandshell}
latexindent.pl textwrap8.tex -o=+-mod2.tex -l=tw-after1.yaml -m
   \end{commandshell}

 gives the output given in \cref{lst:textwrap8-mod2}.

 \begin{cmhtcbraster}[raster columns=2,
   raster left skip=-1.5cm,
   raster right skip=-0cm,
   raster column skip=.06\linewidth]
  \cmhlistingsfromfile*{demonstrations/textwrap8-mod2.tex}{\texttt{textwrap8-mod2.tex}}{lst:textwrap8-mod2}
  \cmhlistingsfromfile*[style=yaml-LST]{demonstrations/tw-after1.yaml}[MLB-TCB,width=0.4\linewidth]{\texttt{tw-after1.yaml}}{lst:tw-after1}
 \end{cmhtcbraster}

 We note that, in \cref{lst:textwrap8-mod2}, that the wrapped text has \emph{obeyed} the
 specified value of \texttt{columns} (35) given in \cref{lst:tw-after1}.
 \end{example}

\subsubsection{Text wrap: wrapping comments}\label{subsubsec:tw:comments}
 You can instruct \texttt{latexindent.pl} to apply text wrapping to comments
 \announce*{2023-01-01}{text wrap trailing comments}; we demonstrate this with examples,
 see also \cref{subsubsec:ospl:tw:comments}. \index{comments!text wrap} \index{text
 wrap!comments}

 \begin{example}
 We use the file in \cref{lst:textwrap9} which contains a trailing comment block.

 \cmhlistingsfromfile*{demonstrations/textwrap9.tex}{\texttt{textwrap9.tex}}{lst:textwrap9}

 Using the settings given in \cref{lst:wrap-comments1} and running the command

 \begin{commandshell}
latexindent.pl textwrap9.tex -o=+-mod1.tex -l=wrap-comments1.yaml -m
   \end{commandshell}

 gives the output given in \cref{lst:textwrap9-mod1}.

 \begin{cmhtcbraster}[raster columns=2,
   raster left skip=-1.5cm,
   raster right skip=-0cm,
   raster column skip=.06\linewidth]
  \cmhlistingsfromfile*{demonstrations/textwrap9-mod1.tex}{\texttt{textwrap9-mod1.tex}}{lst:textwrap9-mod1}
  \cmhlistingsfromfile*[style=yaml-LST]{demonstrations/wrap-comments1.yaml}[MLB-TCB,width=0.4\linewidth]{\texttt{wrap-comments1.yaml}}{lst:wrap-comments1}
 \end{cmhtcbraster}

 We note that, in \cref{lst:textwrap9-mod1}, that the comments have been \emph{combined
 and wrapped} because of the annotated line specified in \cref{lst:wrap-comments1}.
 \end{example}

 \begin{example}
 We use the file in \cref{lst:textwrap10} which contains a trailing comment block.

 \cmhlistingsfromfile*{demonstrations/textwrap10.tex}{\texttt{textwrap10.tex}}{lst:textwrap10}

 Using the settings given in \cref{lst:wrap-comments1:repeat} and running the command

 \begin{commandshell}
latexindent.pl textwrap10.tex -o=+-mod1.tex -l=wrap-comments1.yaml -m
   \end{commandshell}

 gives the output given in \cref{lst:textwrap10-mod1}.

 \begin{cmhtcbraster}[raster columns=2,
   raster left skip=-1.5cm,
   raster right skip=-0cm,
   raster column skip=.06\linewidth]
  \cmhlistingsfromfile*{demonstrations/textwrap10-mod1.tex}{\texttt{textwrap10-mod1.tex}}{lst:textwrap10-mod1}
  \cmhlistingsfromfile*[style=yaml-LST]{demonstrations/wrap-comments1.yaml}[MLB-TCB,width=0.4\linewidth]{\texttt{wrap-comments1.yaml}}{lst:wrap-comments1:repeat}
 \end{cmhtcbraster}

 We note that, in \cref{lst:textwrap10-mod1}, that the comments have been \emph{combined
 and wrapped} because of the annotated line specified in
 \cref{lst:wrap-comments1:repeat}, and that the space from the leading comment has not
 been inherited; we will explore this further in the next example.
 \end{example}

 \begin{example}
 We continue to use the file in \cref{lst:textwrap10}.

 Using the settings given in \cref{lst:wrap-comments2} and running the command

 \begin{commandshell}
latexindent.pl textwrap10.tex -o=+-mod2.tex -l=wrap-comments2.yaml -m
   \end{commandshell}

 gives the output given in \cref{lst:textwrap10-mod2}.

 \begin{cmhtcbraster}[raster columns=2,
   raster left skip=-1.5cm,
   raster right skip=-0cm,
   raster column skip=.06\linewidth]
  \cmhlistingsfromfile*{demonstrations/textwrap10-mod2.tex}{\texttt{textwrap10-mod2.tex}}{lst:textwrap10-mod2}
  \cmhlistingsfromfile*[style=yaml-LST]{demonstrations/wrap-comments2.yaml}[MLB-TCB,width=0.4\linewidth]{\texttt{wrap-comments2.yaml}}{lst:wrap-comments2}
 \end{cmhtcbraster}

 We note that, in \cref{lst:textwrap10-mod2}, that the comments have been \emph{combined
 and wrapped} and that the leading space has been inherited because of the annotated
 lines specified in \cref{lst:wrap-comments2}.
 \end{example}

\subsubsection{Text wrap: huge, tabstop and separator}
 The \announce{2021-07-23}*{huge:overflow is now default} default value of \texttt{huge}
 is \texttt{overflow}, which means that words will \emph{not} be broken by the text
 wrapping routine, implemented by the \texttt{Text::Wrap} \cite{textwrap}. There are
 options to change the \texttt{huge} option for the \texttt{Text::Wrap} module to either
 \texttt{wrap} or \texttt{die}. Before modifying the value of \texttt{huge}, please bear
 in mind the following warning: \index{warning!changing huge (textwrap)}%
 \begin{warning}
  \raggedright
  Changing the value of \texttt{huge} to anything other than \texttt{overflow} will slow
  down \texttt{latexindent.pl} significantly when the \texttt{-m} switch is active.

  Furthermore, changing \texttt{huge} means that you may have some words \emph{or
  commands}(!) split across lines in your .tex file, which may affect your output. I do
  not recommend changing this field.
 \end{warning}

 \begin{example}
 For example, using the settings in \cref{lst:textwrap2A-yaml,lst:textwrap2B-yaml} and
 running the commands \index{switches!-l demonstration} \index{switches!-m demonstration}
 \index{switches!-o demonstration}

 \begin{commandshell}
	 latexindent.pl -m textwrap4.tex -o=+-mod2A -l textwrap2A.yaml
	 latexindent.pl -m textwrap4.tex -o=+-mod2B -l textwrap2B.yaml
\end{commandshell}

 gives the respective output in \cref{lst:textwrap4-mod2A,lst:textwrap4-mod2B}.

 \begin{cmhtcbraster}[raster column skip=.1\linewidth]
  \cmhlistingsfromfile{demonstrations/textwrap4-mod2A.tex}{\texttt{textwrap4-mod2A.tex}}{lst:textwrap4-mod2A}
  \cmhlistingsfromfile[style=yaml-LST]{demonstrations/textwrap2A.yaml}[MLB-TCB]{\texttt{textwrap2A.yaml}}{lst:textwrap2A-yaml}

  \cmhlistingsfromfile{demonstrations/textwrap4-mod2B.tex}{\texttt{textwrap4-mod2B.tex}}{lst:textwrap4-mod2B}
  \cmhlistingsfromfile[style=yaml-LST]{demonstrations/textwrap2B.yaml}[MLB-TCB]{\texttt{textwrap2B.yaml}}{lst:textwrap2B-yaml}
 \end{cmhtcbraster}
 \end{example}

 You can also specify the \texttt{tabstop} field \announce{2020-11-06}{tabstop option for
 text wrap module} as an integer value, which is passed to the text wrap module; see
 \cite{textwrap} for details.

 \begin{example}
 Starting with the code in \cref{lst:textwrap-ts} with settings in \cref{lst:tabstop},
 and running the command \index{switches!-l demonstration} \index{switches!-m
 demonstration} \index{switches!-o demonstration}%

 \begin{commandshell}
	 latexindent.pl -m textwrap-ts.tex -o=+-mod1 -l tabstop.yaml
	 \end{commandshell}

 gives the code given in \cref{lst:textwrap-ts-mod1}.
 \begin{cmhtcbraster}[raster columns=3,
   raster left skip=-3.5cm,
   raster right skip=-2cm,
   raster column skip=.03\linewidth]
  \cmhlistingsfromfile[showtabs=true]{demonstrations/textwrap-ts.tex}{\texttt{textwrap-ts.tex}}{lst:textwrap-ts}
  \cmhlistingsfromfile[style=yaml-LST]{demonstrations/tabstop.yaml}[MLB-TCB]{\texttt{tabstop.yaml}}{lst:tabstop}
  \cmhlistingsfromfile[showtabs=true]{demonstrations/textwrap-ts-mod1.tex}{\texttt{textwrap-ts-mod1.tex}}{lst:textwrap-ts-mod1}
 \end{cmhtcbraster}
 \end{example}

 You can specify \texttt{separator}, \texttt{break} and \texttt{unexpand} options in your
 settings in analogous ways to those demonstrated in
 \cref{lst:textwrap2B-yaml,lst:tabstop}, and they will be passed to the
 \texttt{Text::Wrap} module. I have not found a useful reason to do this; see
 \cite{textwrap} for more details.
