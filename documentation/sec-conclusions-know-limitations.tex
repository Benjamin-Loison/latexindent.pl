% arara: pdflatex: { files: [latexindent]}
\section{Conclusions and known limitations}\label{sec:knownlimitations}
 There are a number of known limitations of the script, and almost certainly quite a few
 that are \emph{unknown}! The known issues include:
 \begin{description}
  \item[multicolumn alignment] when working with code blocks in which multicolumn commands
   overlap, the algorithm can fail; see \vref{lst:tabular2-mod2}.
  \item[text wrap] routine operates \emph{before} indentation occurs; this means that it is
   likely that your final, indented, text wrapped text may exceed the value of
   \texttt{columns} that you specify; see \vref{subsec:textwrapping}.
  \item[efficiency] particularly when the \texttt{-m} switch is active, as this adds many checks
   and processes. The current implementation relies upon finding and storing \emph{every}
   code block (see the discussion on \cpageref{page:phases}); I hope that, in a future
   version, only \emph{nested} code blocks will need to be stored in the `packing' phase,
   and that this will improve the efficiency of the script.
 \end{description}

 You can run \texttt{latexindent} on any file; \announce{2019-07-13}*{ability to call
 latexindent on any file} if you don't specify an extension, then the extensions that you
 specify in \lstinline[breaklines=true]!fileExtensionPreference! (see
 \vref{lst:fileExtensionPreference}) will be consulted. If you find a case in which the
 script struggles, please feel free to report it at \cite{latexindent-home}, and in the
 meantime, consider using a \texttt{noIndentBlock} (see \cpageref{lst:noIndentBlock}).%

 I hope that this script is useful to some; if you find an example where the script does
 not behave as you think it should, the best way to contact me is to report an issue on
 \cite{latexindent-home}; otherwise, feel free to find me on the
 \url{http://tex.stackexchange.com/users/6621/cmhughes}.
