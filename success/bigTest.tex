% arara: indent: {trace: true, overwrite: yes}
\documentclass[10pt,twoside]{report}
\begin{document}

% needed for the mini-tableofcontents
\dominitoc
\faketableofcontents

\fancyhf{} % delete current header and footer
\fancyhead[LE,RO]{\bfseries\thepage}
\fancyhead[LO,RE]{\tiny\rightmark}
\fancyheadoffset[LE,LO]{4cm}

\pagestyle{fancy}
%\include{coverpage}
\include{functions}
%\include{exponentialfunctions}
%\include{functions2}
%\include{logarithms}
%\include{polyrat}
%\include{ideas}

%=======================
%   BEGIN SOLUTIONS
%=======================

% change the page geometry using \newgeometry
%\cleardoublepage
\clearpage
%\setbool{@twoside}{false} 
\fancyheadoffset[RE,RO]{2cm}
\fancyheadoffset[LE,LO]{1cm}
\renewcommand{\rightmark}{Solutions to Section \thesection}
\fancyhead[LO,RE]{\rightmark}
\newgeometry{left=4cm,right=4cm,showframe=true,
	marginratio=1:1,
	top=1.5cm,bottom=1.5cm,bindingoffset=0cm}

% finish the php file
\Writetofile{crossrefsWEB}{?>}

% close the solutions files
\Closesolutionfile{shortsolutions}
\Closesolutionfile{longsolutions}
%\Closesolutionfile{hints}
\Closesolutionfile{crossrefsWEB}

% when itemized lists are used in the solutions, they
% are actually at 2nd depth because the solution environment 
% uses an \itemize environment to get the indendation correct
\setlist[itemize,2]{label=\textbullet}

% SHORT solution to problem (show only odd, even, all)
% Note: this renewenvironment needs to go here
%       so that the answers package can still
%       display correctly to the page if needed
\newbool{oddproblemnumber}
\renewenvironment{shortSoln}[1]{%
	\exploregroups % needed to ignore {}
	% before the environment starts - this is a stretchable space
	\vskip 0.1cm plus 2cm minus 0.1cm%
	\fullexpandarg % need this line so that '.' are counted
	%
	% either problems, or subproblems, e.g: 3.1 or 3.1.4 respectively
	% determine which one by counting the '.'
	\StrCount{#1}{.}[\numberofdecimals]
	%
	% find the problem number by splitting the string
	\ifnumequal{\numberofdecimals}{0}%
	{%
		% problems, such as 4, 5, 6, ...
		\def\problemnumber{#1}%
	}%
	{}%
	\ifnumequal{\numberofdecimals}{1}%
	{%
		% subproblems, such as 4.3, 1.2, 10.5
		\StrBefore{#1}{.}[\problemnumber]%
	}%
	{}%
	\ifnumequal{\numberofdecimals}{2}%
	{%
		% subproblems such as 1.3.1, 1.2.4, 7.5.6
		% note that these aren't currently used, but maybe someday
		\StrBehind{#1}{.}[\newbit]%
		\StrBefore{\newbit}{.}[\problemnumber]%
	}%
	{}%
	%
	% determine if the problem number is odd or even
	% and depending on our choices above, display or not
	\ifnumodd{\problemnumber}%
	{%
		% set a boolean that says the problem number is odd (used later)
		\setbool{oddproblemnumber}{true}%
		% display or not 
		\ifbool{showoddsolns}%
		{%
			% if we want to show the odd problems
			\ifbool{coreproblemYesNo}%
			{%  Core problem
				\expandafter\itemize[label=\llap{$\bigstar$ }\bfseries \hyperlink{prob:#1:\thechapter:\thesection}{#1}.]\item%
			}%
			{%  NOT Core problem
				\expandafter\itemize[label=\bfseries \hyperlink{prob:#1:\thechapter:\thesection}{#1}.]\item%
			}%
		}%
		{%
			% otherwise don't show them!
			\expandafter\comment%          
		}%
	}%
	{%
		% even numbered problem, set the boolean (used later)
		\setbool{oddproblemnumber}{false}%
		\ifbool{showevensolns}%
		{%
			% if we want to show the even problems
			\ifbool{coreproblemYesNo}%
			{%  Core problem
				\expandafter\itemize[label=\llap{$\bigstar$ }\itshape \hyperlink{prob:#1:\thechapter:\thesection}{#1}.]\item%
			}%
			{%  NOT Core problem
				\expandafter\itemize[label=\itshape \hyperlink{prob:#1:\thechapter:\thesection}{#1}.]\item%
			}%
		}%
		{%
			% otherwise don't show them!
			\expandafter\comment%          
		}%
	}%
}%
{%
	% after the environment finishes
	\ifbool{oddproblemnumber}%
	{%
		% odd numbered problems 
		\ifbool{showoddsolns}%
		{%
			% if we want to show the odd problems
			% then the environment is finished
			\enditemize%
		}%
		{%
			% otherwise we need to finish the comment
			\expandafter\endcomment%
		}%
	}%
	{%
		% even numbered problems
		\ifbool{showevensolns}%
		{%
			% if we want to show the even problems
			% then the environment is finished
			\enditemize%
		}%
		{%
			% otherwise we need to finish the comment
			\expandafter\endcomment%
		}%
	}%
}

% LONG solution to problem (show only odd, even, all)
% Note: this renewenvironment needs to go here
%       so that the answers package can still
%       display correctly to the page if needed
\renewenvironment{longSoln}[1]{%
	\exploregroups % needed to ignore {}
	% before the environment starts - this is a stretchable space
	\vskip 0.1cm plus 2cm minus 0.1cm%
	\fullexpandarg % need this line so that '.' are counted
	%
	% either problems, or subproblems, e.g: 3.1 or 3.1.4 respectively
	% determine which one by counting the '.'
	\StrCount{#1}{.}[\numberofdecimals]
	%
	% find the problem number by splitting the string
	\ifnumequal{\numberofdecimals}{0}%
	{%
		% problems, such as 4, 5, 6, ...
		\def\problemnumber{#1}%
	}%
	{}%
	\ifnumequal{\numberofdecimals}{1}%
	{%
		% problems, such as 4.3, 1.2, 10.5
		\StrBefore{#1}{.}[\problemnumber]%
	}%
	{}%
	\ifnumequal{\numberofdecimals}{2}%
	{%
		% subproblems such as 1.3.1, 1.2.4, 7.5.6
		\StrBehind{#1}{.}[\newbit]%
		\StrBefore{\newbit}{.}[\problemnumber]%
	}%
	{}%
	%
	% determine if the problem number is odd or even
	% and depending on our choices above, display or not
	\ifnumodd{\problemnumber}%
	{%
		% set a boolean that says the problem number is odd (used later)
		\setbool{oddproblemnumber}{true}%
		% display or not 
		\ifbool{showoddsolns}%
		{%
			% if we want to show the odd problems
			{\bfseries \hyperlink{prob:#1:\thechapter:\thesection}{#1}.}%
		}%
		{%
			% otherwise don't show them!
			\expandafter\comment%          
		}%
	}%
	{%
		% even numbered problem, set the boolean (used later)
		\setbool{oddproblemnumber}{false}%
		\ifbool{showevensolns}%
		{%
			% if we want to show the even problems
			{\itshape \hyperlink{prob:#1:\thechapter:\thesection}{#1}.}%
		}%
		{%
			% otherwise don't show them!
			\expandafter\comment%          
		}%
	}%
}%
{%
	% after the environment finishes
	\ifbool{oddproblemnumber}%
	{%
		% odd numbered problems 
		\ifbool{showoddsolns}%
		{%
			% if we want to show the odd problems
			% then the environment is finished
		}%
		{%
			% otherwise we need to finish the comment
			\expandafter\endcomment%
		}%
	}%
	{%
		% even numbered problems
		\ifbool{showevensolns}%
		{%
			% if we want to show the even problems
			% then the environment is finished
		}%
		{%
			% otherwise we need to finish the comment
			\expandafter\endcomment%
		}%
	}%
}

% renew tikzpicture environment to make it use valign=t
% on every one, which fixes vertical alignment of tikzpicture
% with the solution label: http://tex.stackexchange.com/questions/30367/aligning-enumerate-labels-to-top-of-image
\BeforeBeginEnvironment{tikzpicture}{\begin{adjustbox}{valign=t}}
\AfterEndEnvironment{tikzpicture}{\end{adjustbox}}

% do the same for the tabular environment
\BeforeBeginEnvironment{tabular}{\begin{adjustbox}{valign=t}}
\AfterEndEnvironment{tabular}{\end{adjustbox}}

% set every picture in the solutions to have \solutionfigurewidth
\pgfplotsset{
	every axis/.append style={%
		width=\solutionfigurewidth}}

% input the SHORT solutions file
\IfFileExists{shortsolutions.tex}{\input{shortsolutions.tex}}{}

\clearpage
% input the LONG solutions file
%\IfFileExists{longsolutions.tex}{\input{longsolutions.tex}}{}

\clearpage
% input the HINTS file
%\IfFileExists{hints.tex}{\input{hints.tex}}{}
%=======================
%   END SOLUTIONS
%=======================

%=======================
%   INDEX
%=======================
\printindex

\end{document}
