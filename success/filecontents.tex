% arara: indent: {overwrite: true, trace: false, localSettings: yes}

% used with localSettings.yaml as follows
%indentPreamble: 1
%indentRules:
%   @online: "\t\t\t\t"
%   #myenvironment: "\t\t"
%   myotherenvironment: "\t\t"
%   \[: "\t\t"
%   tabular: "\t\t\t"
%noAdditionalIndent:
%   @online: 0
%   myotherenvironment: 1
%   \[: 0
%   \]: 0
%   tabular: 0
%   something: 0
%   parbox: 1
%verbatimEnvironments:
%   myotherenvironment: 1
%   tabular: 0
%   someothername: 0


%        \begin{noindent}
here we are in a block
% \end{noindent}
some more
\begin{tabular}{cccc}
	1 & 2 & 3 & 4 \\
	5 &   & 6 &   \\
\end{tabular}

another test
\begin{tabular}{cccc}
	1 & 2 & 3 & 4 \\
	5 &   & 6 &   \\
\end{tabular}

\begin{something}
	\parbox{something
		else
		goes
		here
	}
	some text some text
	some text some text
	some text some text
	\[
		x^2+2x
	\]
	some text some text
	some text some text
	some text some text
	some text some text
	some text some text
\end{something}
\begin{filecontents}{mybib.bib}
	@online{strawberryperl,
		title="Strawberry Perl",
		url="http://strawberryperl.com/"}
	@online{cmhblog,
		title="A Perl script for indenting tex files",
		url="http://tex.blogoverflow.com/2012/08/a-perl-script-for-indenting-tex-files/"}
\end{filecontents}

\begin{myotherenvironment}
	some text goes here
	some text goes here
	some text goes here
	some text goes here
\end{myotherenvironment}

