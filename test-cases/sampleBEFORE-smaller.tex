% arara: indent: {overwrite: true, trace: on}
% A sample chapter file- it contains a lot of
% environments, including tabulars, align, etc
%
% Don't try and compile this file using pdflatex etc, just
% compare the *format* of it to the format of the
% sampleAFTER.tex
%
% In particular, compare the tabular and align-type
% environments before and after running the script

\section{Polynomial functions}
 \reformatstepslist{P} % the steps list should be P1, P2, \ldots
 In your previous mathematics classes you have studied \emph{linear} and
 \emph{quadratic} functions. The most general forms of these types of
 functions can be represented (respectively) by  the functions $f$
 and $g$ that have formulas
 \begin{equation}\label{poly:eq:linquad}
 	f(x)=mx+b, \qquad g(x)=ax^2+bx+c
 \end{equation}
 We know that $m$ is the slope of $f$, and that $a$ is the \emph{leading coefficient}
 of $g$. We also know that the \emph{signs} of $m$ and $a$ completely
 determine the behavior of the functions $f$ and $g$. For example, if $m>0$
 then $f$ is an \emph{increasing} function, and if $m<0$ then $f$ is
 a \emph{decreasing} function.  Similarly, if $a>0$ then $g$ is
 \emph{concave up} and if $a<0$ then $g$ is \emph{concave down}. Graphical
 representations of these statements are given in \cref{poly:fig:linquad}.

 \begin{figure}[!htb]
 	\setlength{\figurewidth}{.2\textwidth}
 	\begin{subfigure}{\figurewidth}
 		\begin{tikzpicture}
 			\begin{axis}[
 					framed,
 					xmin=-10,xmax=10,
 					ymin=-10,ymax=10,
 					width=\textwidth,
 					xtick={-11},
 					ytick={-11},
 				]
 				\addplot expression[domain=-10:8]{(x+2)};
 			\end{axis}
 		\end{tikzpicture}
 		\caption{$m>0$}
 	\end{subfigure}
 	\hfill
 	\begin{subfigure}{\figurewidth}
 		\begin{tikzpicture}
 			\begin{axis}[
 					framed,
 					xmin=-10,xmax=10,
 					ymin=-10,ymax=10,
 					width=\textwidth,
 					xtick={-11},
 					ytick={-11},
 				]
 				\addplot expression[domain=-10:8]{-(x+2)};
 			\end{axis}
 		\end{tikzpicture}
 		\caption{$m<0$}
 	\end{subfigure}
 	\hfill
 	\begin{subfigure}{\figurewidth}
 		\begin{tikzpicture}
 			\begin{axis}[
 					framed,
 					xmin=-10,xmax=10,
 					ymin=-10,ymax=10,
 					width=\textwidth,
 					xtick={-11},
 					ytick={-11},
 				]
 				\addplot expression[domain=-4:4]{(x^2-6)};
 			\end{axis}
 		\end{tikzpicture}
 		\caption{$a>0$}
 	\end{subfigure}
 	\hfill
 	\begin{subfigure}{\figurewidth}
 		\begin{tikzpicture}
 			\begin{axis}[
 					framed,
 					xmin=-10,xmax=10,
 					ymin=-10,ymax=10,
 					width=\textwidth,
 					xtick={-11},
 					ytick={-11},
 				]
 				\addplot expression[domain=-4:4]{-(x^2-6)};
 			\end{axis}
 		\end{tikzpicture}
 		\caption{$a<0$}
 	\end{subfigure}
 	\caption{Typical graphs of linear and quadratic functions.}
 	\label{poly:fig:linquad}
 \end{figure}

 Let's look a little more closely at the formulas for $f$ and $g$ in
 \cref{poly:eq:linquad}. Note that the \emph{degree}
 of $f$ is $1$ since the highest power of $x$ that is present in the
 formula for $f(x)$ is $1$. Similarly, the degree of $g$ is $2$ since
 the highest power of $x$ that is present in the formula for $g(x)$
 is $2$.

 In this section we will build upon our knowledge of these elementary
 functions. In particular, we will generalize the functions $f$ and $g$ to a function $p$ that has
 any degree that we wish.

 %===================================
 %   Author: Hughes
 %   Date:   March 2012
 %===================================
 \begin{essentialskills}
 	%===================================
 	%   Author: Hughes
 	%   Date:   March 2012
 	%===================================
 	\begin{problem}[Quadratic functions]
 	Every quadratic function has the form $y=ax^2+bx+c$; state the value
 	of $a$ for each of the following functions, and hence decide if the
 	parabola that represents the function opens upward or downward.
 	\begin{multicols}{2}
 		\begin{subproblem}
 			$F(x)=x^2+3$
 			\begin{shortsolution}
 				$a=1$; the parabola opens upward.
 			\end{shortsolution}
 		\end{subproblem}
 		\begin{subproblem}
 			$G(t)=4-5t^2$
 			\begin{shortsolution}
 				$a=-5$; the parabola opens downward.
 			\end{shortsolution}
 		\end{subproblem}
 		\begin{subproblem}
 			$H(y)=4y^2-96y+8$
 			\begin{shortsolution}
 				$a=4$; the parabola opens upward.
 			\end{shortsolution}
 		\end{subproblem}
 		\begin{subproblem}
 			$K(z)=-19z^2$
 			\begin{shortsolution}
 				$m=-19$; the parabola opens downward.
 			\end{shortsolution}
 		\end{subproblem}
 	\end{multicols}
 	Now let's generalize our findings for the most general quadratic function $g$
 	that has formula $g(x)=a_2x^2+a_1x+a_0$. Complete the following sentences.
 	\begin{subproblem}
 		When $a_2>0$, the parabola that represents $y=g(x)$ opens $\ldots$
 		\begin{shortsolution}
 			When $a_2>0$, the parabola that represents the function opens upward.
 		\end{shortsolution}
 	\end{subproblem}
 	\begin{subproblem}
 		When $a_2<0$, the parabola that represents $y=g(x)$ opens $\ldots$
 		\begin{shortsolution}
 			When $a_2<0$, the parabola that represents the function opens downward.
 		\end{shortsolution}
 	\end{subproblem}
 	\end{problem}
 \end{essentialskills}

 \subsection*{Power functions with positive exponents}
 	The study of polynomials will rely upon a good knowledge
 	of power functions| you may reasonably ask, what is a power function?
 	\begin{pccdefinition}[Power functions]
 		Power functions have the form
 		\[
 			f(x) = a_n x^n
 		\]
 		where $n$ can be any real number.

 		Note that for this section we will only be concerned with the
 		case when $n$ is a positive integer.
 	\end{pccdefinition}

 	You may find assurance in the fact that you are already very comfortable
 	with power functions that have $n=1$ (linear) and $n=2$ (quadratic). Let's
 	explore some power functions that you might not be so familiar with.
 	As you read \cref{poly:ex:oddpow,poly:ex:evenpow}, try and spot
 	as many patterns and similarities as you can.

 	%===================================
 	%   Author: Hughes
 	%   Date:   March 2012
 	%===================================
 	\begin{pccexample}[Power functions with odd positive exponents]
 		\label{poly:ex:oddpow}
 		Graph each of the following functions, state their domain, and their
 		long-run behavior as $x\rightarrow\pm\infty$
 		\[
 			f(x)=x^3,   \qquad  g(x)=x^5, \qquad h(x)=x^7
 		\]
 		\begin{pccsolution}
 			The functions $f$, $g$, and $h$ are plotted in \cref{poly:fig:oddpow}.
 			The domain of each of the functions $f$, $g$, and $h$ is $(-\infty,\infty)$. Note that
 			the long-run behavior of each of the functions is the same, and in particular
 			\begin{align*}
 				f(x)\rightarrow\infty                           & \text{ as } x\rightarrow\infty  \\
 				\mathllap{\text{and }}   f(x)\rightarrow-\infty & \text{ as } x\rightarrow-\infty
 			\end{align*}
 			The same results hold for $g$ and $h$.
 		\end{pccsolution}
 	\end{pccexample}

 	\begin{figure}[!htb]
 		\begin{minipage}{.45\textwidth}
 			\begin{tikzpicture}
 				\begin{axis}[
 						framed,
 						xmin=-1.5,xmax=1.5,
 						ymin=-5,ymax=5,
 						xtick={-1.0,-0.5,...,1.0},
 						minor ytick={-3,-1,...,3},
 						grid=both,
 						width=\textwidth,
 						legend pos=north west,
 					]
 					\addplot expression[domain=-1.5:1.5]{x^3};
 					\addplot expression[domain=-1.379:1.379]{x^5};
 					\addplot expression[domain=-1.258:1.258]{x^7};
 					\addplot[soldot]coordinates{(-1,-1)} node[axisnode,anchor=north west]{$(-1,-1)$};
 					\addplot[soldot]coordinates{(1,1)} node[axisnode,anchor=south east]{$(1,1)$};
 					\legend{$f$,$g$,$h$}
 				\end{axis}
 			\end{tikzpicture}
 			\caption{Odd power functions}
 			\label{poly:fig:oddpow}
 		\end{minipage}%
 		\hfill
 		\begin{minipage}{.45\textwidth}
 			\begin{tikzpicture}
 				\begin{axis}[
 						framed,
 						xmin=-2.5,xmax=2.5,
 						ymin=-5,ymax=5,
 						xtick={-2.0,-1.5,...,2.0},
 						minor ytick={-3,-1,...,3},
 						grid=both,
 						width=\textwidth,
 						legend pos=south east,
 					]
 					\addplot expression[domain=-2.236:2.236]{x^2};
 					\addplot expression[domain=-1.495:1.495]{x^4};
 					\addplot expression[domain=-1.307:1.307]{x^6};
 					\addplot[soldot]coordinates{(-1,1)} node[axisnode,anchor=east]{$(-1,1)$};
 					\addplot[soldot]coordinates{(1,1)} node[axisnode,anchor=west]{$(1,1)$};
 					\legend{$F$,$G$,$H$}
 				\end{axis}
 			\end{tikzpicture}
 			\caption{Even power functions}
 			\label{poly:fig:evenpow}
 		\end{minipage}%
 	\end{figure}

 	%===================================
 	%   Author: Hughes
 	%   Date:   March 2012
 	%===================================
 	\begin{pccexample}[Power functions with even positive exponents]\label{poly:ex:evenpow}%
 		Graph each of the following functions, state their domain, and their
 		long-run behavior as $x\rightarrow\pm\infty$
 		\[
 			F(x)=x^2, \qquad G(x)=x^4, \qquad H(x)=x^6
 		\]
 		\begin{pccsolution}
 			The functions $F$, $G$, and $H$ are plotted in \cref{poly:fig:evenpow}. The domain
 			of each of the functions is $(-\infty,\infty)$. Note that the long-run behavior
 			of each of the functions is the same, and in particular
 			\begin{align*}
 				F(x)\rightarrow\infty                          & \text{ as } x\rightarrow\infty  \\
 				\mathllap{\text{and }}   F(x)\rightarrow\infty & \text{ as } x\rightarrow-\infty
 			\end{align*}
 			The same result holds for $G$ and $H$.
 		\end{pccsolution}
 	\end{pccexample}

 	\begin{doyouunderstand}
 		\begin{problem}
 		Repeat \cref{poly:ex:oddpow,poly:ex:evenpow} using (respectively)
 		\begin{subproblem}
 			$f(x)=-x^3,   \qquad  g(x)=-x^5, \qquad h(x)=-x^7$
 			\begin{shortsolution}
 				The functions $f$, $g$, and $h$ have domain $(-\infty,\infty)$ and
 				are graphed below.

 				\begin{tikzpicture}
 					\begin{axis}[
 							framed,
 							xmin=-1.5,xmax=1.5,
 							ymin=-5,ymax=5,
 							xtick={-1.0,-0.5,...,0.5},
 							minor ytick={-3,-1,...,3},
 							grid=both,
 							width=\solutionfigurewidth,
 							legend pos=north east,
 						]
 						\addplot expression[domain=-1.5:1.5]{-x^3};
 						\addplot expression[domain=-1.379:1.379]{-x^5};
 						\addplot expression[domain=-1.258:1.258]{-x^7};
 						\legend{$f$,$g$,$h$}
 					\end{axis}
 				\end{tikzpicture}

 				Note that
 				\begin{align*}
 					f(x)\rightarrow-\infty                         & \text{ as } x\rightarrow\infty  \\
 					\mathllap{\text{and }}   f(x)\rightarrow\infty & \text{ as } x\rightarrow-\infty
 				\end{align*}
 				The same is true for $g$ and $h$.
 			\end{shortsolution}
 		\end{subproblem}
 		\begin{subproblem}
 			$F(x)=-x^2,   \qquad  G(x)=-x^4, \qquad H(x)=-x^6$
 			\begin{shortsolution}
 				The functions $F$, $G$, and $H$ have domain $(-\infty,\infty)$ and
 				are graphed below.

 				\begin{tikzpicture}
 					\begin{axis}[
 							framed,
 							xmin=-2.5,xmax=2.5,
 							ymin=-5,ymax=5,
 							xtick={-1.0,-0.5,...,0.5},
 							minor ytick={-3,-1,...,3},
 							grid=both,
 							width=\solutionfigurewidth,
 							legend pos=north east,
 						]
 						\addplot expression[domain=-2.236:2.236]{-x^2};
 						\addplot expression[domain=-1.495:1.495]{-x^4};
 						\addplot expression[domain=-1.307:1.307]{-x^6};
 						\legend{$F$,$G$,$H$}
 					\end{axis}
 				\end{tikzpicture}

 				Note that
 				\begin{align*}
 					F(x)\rightarrow-\infty                          & \text{ as } x\rightarrow\infty  \\
 					\mathllap{\text{and }}   F(x)\rightarrow-\infty & \text{ as } x\rightarrow-\infty
 				\end{align*}
 				The same is true for $G$ and $H$.
 			\end{shortsolution}
 		\end{subproblem}
 		\end{problem}
 	\end{doyouunderstand}

 \subsection*{Polynomial functions}
 	Now that we have a little more familiarity with power functions,
 	we can define polynomial functions. Provided that you were comfortable
 	with our opening discussion about linear and quadratic functions (see
 	$f$ and $g$ in \cref{poly:eq:linquad}) then there is every chance
 	that you'll be able to master polynomial functions as well; just remember
 	that polynomial functions are a natural generalization of linear
 	and quadratic functions. Once you've studied the examples and problems
 	in this section, you'll hopefully agree that polynomial functions
 	are remarkably predictable.

 	%===================================
 	%   Author: Hughes
 	%   Date:   May 2011
 	%===================================
 	\begin{pccdefinition}[Polynomial functions]
 		Polynomial functions have the form
 		\[
 			p(x)=a_nx^n+a_{n-1}x^{n-1}+\ldots+a_1x+a_0
 		\]
 		where $a_n$, $a_{n-1}$, $a_{n-2}$, \ldots, $a_0$ are real numbers.
 		\begin{itemize}
 			\item We call $n$ the degree of the polynomial, and require that $n$
 			      is a non-negative integer;
 			\item $a_n$, $a_{n-1}$, $a_{n-2}$, \ldots, $a_0$ are called the coefficients;
 			\item We typically write polynomial functions in descending powers of $x$.
 		\end{itemize}
 		In particular, we call $a_n$ the \emph{leading} coefficient, and $a_nx^n$ the
 		\emph{leading term}.

 		Note that if a polynomial is given in factored form, then the degree can be found
 		by counting the number of linear factors.
 	\end{pccdefinition}

 	%===================================
 	%   Author: Hughes
 	%   Date:   March 2012
 	%===================================
 	\begin{pccexample}[Polynomial or not]
 		Identify the following functions as polynomial or not; if the function
 		is a polynomial, state its degree.
 		\begin{multicols}{3}
 			\begin{enumerate}
 				\item $p(x)=x^2-3$
 				\item $q(x)=-4x^{\nicefrac{1}{2}}+10$
 				\item $r(x)=10x^5$
 				\item $s(x)=x^{-2}+x^{23}$
 				\item $f(x)=-8$
 				\item $g(x)=3^x$
 				\item $h(x)=\sqrt[3]{x^7}-x^2+x$
 				\item $k(x)=4x(x+2)(x-3)$
 				\item $j(x)=x^2(x-4)(5-x)$
 			\end{enumerate}
 		\end{multicols}
 		\begin{pccsolution}
 			\begin{enumerate}
 				\item $p$ is a polynomial, and its degree is $2$.
 				\item $q$ is \emph{not} a polynomial, because $\frac{1}{2}$ is not an integer.
 				\item $r$ is a polynomial, and its degree is $5$.
 				\item $s$ is \emph{not} a polynomial, because $-2$ is not a positive integer.
 				\item $f$ is a polynomial, and its degree is $0$.
 				\item $g$ is \emph{not} a polynomial, because the independent
 				      variable, $x$, is in the exponent.
 				\item $h$ is \emph{not} a polynomial, because $\frac{7}{3}$ is not an integer.
 				\item $k$ is a polynomial, and its degree is $3$.
 				\item $j$ is a polynomial, and its degree is $4$.
 			\end{enumerate}
 		\end{pccsolution}
 	\end{pccexample}

 	%===================================
 	%   Author: Hughes
 	%   Date:   March 2012
 	%===================================
 	\begin{pccexample}[Typical graphs]\label{poly:ex:typical}
 		\Cref{poly:fig:typical} shows graphs of some polynomial functions;
 		the ticks have deliberately been left off the axis to allow us to concentrate
 		on the features of each graph. Note in particular that:
 		\begin{itemize}
 			\item \cref{poly:fig:typical1} shows a degree-$1$ polynomial (you might also
 			      classify the function as linear) whose leading coefficient, $a_1$, is positive.
 			\item \cref{poly:fig:typical2} shows a degree-$2$ polynomial (you might also
 			      classify the function as quadratic) whose leading coefficient, $a_2$, is positive.
 			\item \cref{poly:fig:typical3} shows a degree-$3$ polynomial whose leading coefficient, $a_3$,
 			      is positive| compare its overall
 			      shape and long-run behavior to the functions described in \cref{poly:ex:oddpow}.
 			\item \cref{poly:fig:typical4} shows a degree-$4$ polynomial whose leading coefficient, $a_4$,
 			      is positive|compare its overall shape and long-run behavior to the functions described in \cref{poly:ex:evenpow}.
 			\item \cref{poly:fig:typical5} shows a degree-$5$ polynomial whose leading coefficient, $a_5$,
 			      is positive| compare its overall
 			      shape and long-run behavior to the functions described in \cref{poly:ex:oddpow}.
 		\end{itemize}
 	\end{pccexample}

 	%===================================
 	%   Author: Hughes
 	%   Date:   May 2011
 	%===================================
 	\begin{figure}[!htb]
 		\begin{widepage}
 		\setlength{\figurewidth}{\textwidth/6}
 		\begin{subfigure}{\figurewidth}
 			\begin{tikzpicture}
 				\begin{axis}[
 						framed,
 						xmin=-10,xmax=10,
 						ymin=-10,ymax=10,
 						width=\textwidth,
 						xtick={-11},
 						ytick={-11},
 					]
 					\addplot expression[domain=-10:8]{(x+2)};
 				\end{axis}
 			\end{tikzpicture}
 			\caption{$a_1>0$}
 			\label{poly:fig:typical1}
 		\end{subfigure}
 		\hfill
 		\begin{subfigure}{\figurewidth}
 			\begin{tikzpicture}
 				\begin{axis}[
 						framed,
 						xmin=-10,xmax=10,
 						ymin=-10,ymax=10,
 						width=\textwidth,
 						xtick={-11},
 						ytick={-11},
 					]
 					\addplot expression[domain=-4:4]{(x^2-6)};
 				\end{axis}
 			\end{tikzpicture}
 			\caption{$a_2>0$}
 			\label{poly:fig:typical2}
 		\end{subfigure}
 		\hfill
 		\begin{subfigure}{\figurewidth}
 			\begin{tikzpicture}
 				\begin{axis}[
 						framed,
 						xmin=-10,xmax=10,
 						ymin=-10,ymax=10,
 						width=\textwidth,
 						xtick={-11},
 						ytick={-11},
 					]
 					\addplot expression[domain=-7.5:7.5]{0.05*(x+6)*x*(x-6)};
 				\end{axis}
 			\end{tikzpicture}
 			\caption{$a_3>0$}
 			\label{poly:fig:typical3}
 		\end{subfigure}
 		\hfill
 		\begin{subfigure}{\figurewidth}
 			\begin{tikzpicture}
 				\begin{axis}[
 						framed,
 						xmin=-10,xmax=10,
 						ymin=-10,ymax=10,
 						width=\textwidth,
 						xtick={-11},
 						ytick={-11},
 					]
 					\addplot expression[domain=-2.35:5.35,samples=100]{0.2*(x-5)*x*(x-3)*(x+2)};
 				\end{axis}
 			\end{tikzpicture}
 			\caption{$a_4>0$}
 			\label{poly:fig:typical4}
 		\end{subfigure}
 		\hfill
 		\begin{subfigure}{\figurewidth}
 			\begin{tikzpicture}
 				\begin{axis}[
 						framed,
 						xmin=-10,xmax=10,
 						ymin=-10,ymax=10,
 						width=\textwidth,
 						xtick={-11},
 						ytick={-11},
 					]
 					\addplot expression[domain=-5.5:6.3,samples=100]{0.01*(x+2)*x*(x-3)*(x+5)*(x-6)};
 				\end{axis}
 			\end{tikzpicture}
 			\caption{$a_5>0$}
 			\label{poly:fig:typical5}
 		\end{subfigure}
 		\end{widepage}
 		\caption{Graphs to illustrate typical curves of polynomial functions.}
 		\label{poly:fig:typical}
 	\end{figure}

 	%===================================
 	%   Author: Hughes
 	%   Date:   March 2012
 	%===================================
 	\begin{doyouunderstand}
 		\begin{problem}
 		Use \cref{poly:ex:typical} and \cref{poly:fig:typical} to help you sketch
 		the graphs of polynomial functions that have negative leading coefficients| note
 		that there are many ways to do this! The intention with this problem
 		is to use your knowledge of transformations- in particular, \emph{reflections}-
 		to guide you.
 		\begin{shortsolution}
 			$a_1<0$:

 			\begin{tikzpicture}
 				\begin{axis}[
 						framed,
 						xmin=-10,xmax=10,
 						ymin=-10,ymax=10,
 						width=\solutionfigurewidth,
 						xtick={-11},
 						ytick={-11},
 					]
 					\addplot expression[domain=-10:8]{-(x+2)};
 				\end{axis}
 			\end{tikzpicture}

 			$a_2<0$

 			\begin{tikzpicture}
 				\begin{axis}[
 						framed,
 						xmin=-10,xmax=10,
 						ymin=-10,ymax=10,
 						width=\solutionfigurewidth,
 						xtick={-11},
 						ytick={-11},
 					]
 					\addplot expression[domain=-4:4]{-(x^2-6)};
 				\end{axis}
 			\end{tikzpicture}

 			$a_3<0$

 			\begin{tikzpicture}
 				\begin{axis}[
 						framed,
 						xmin=-10,xmax=10,
 						ymin=-10,ymax=10,
 						width=\solutionfigurewidth,
 						xtick={-11},
 						ytick={-11},
 					]
 					\addplot expression[domain=-7.5:7.5]{-0.05*(x+6)*x*(x-6)};
 				\end{axis}
 			\end{tikzpicture}

 			$a_4<0$

 			\begin{tikzpicture}
 				\begin{axis}[
 						framed,
 						xmin=-10,xmax=10,
 						ymin=-10,ymax=10,
 						width=\solutionfigurewidth,
 						xtick={-11},
 						ytick={-11},
 					]
 					\addplot expression[domain=-2.35:5.35,samples=100]{-0.2*(x-5)*x*(x-3)*(x+2)};
 				\end{axis}
 			\end{tikzpicture}

 			$a_5<0$

 			\begin{tikzpicture}
 				\begin{axis}[
 						framed,
 						xmin=-10,xmax=10,
 						ymin=-10,ymax=10,
 						width=\solutionfigurewidth,
 						xtick={-11},
 						ytick={-11},
 					]
 					\addplot expression[domain=-5.5:6.3,samples=100]{-0.01*(x+2)*x*(x-3)*(x+5)*(x-6)};
 				\end{axis}
 			\end{tikzpicture}
 		\end{shortsolution}
 		\end{problem}
 	\end{doyouunderstand}

 	\fixthis{poly: Need a more basic example here- it can have a similar
 		format to the multiple zeros example, but just keep it simple; it should
 	be halfway between the 2 examples surrounding it}

 	%===================================
 	%   Author: Hughes
 	%   Date:   May 2011
 	%===================================
 	\begin{pccexample}[Multiple zeros]
 		Consider the polynomial functions $p$, $q$, and $r$ which are
 		graphed in \cref{poly:fig:moremultiple}.
 		The formulas for $p$, $q$, and $r$ are as follows
 		\begin{align*}
 			p(x) & =(x-3)^2(x+4)^2       \\
 			q(x) & =x(x+2)^2(x-1)^2(x-3) \\
 			r(x) & =x(x-3)^3(x+1)^2
 		\end{align*}
 		Find the degree of $p$, $q$, and $r$, and decide if the functions bounce off or cut
 		through the horizontal axis at each of their zeros.
 		\begin{pccsolution}
 			The degree of $p$ is 4. Referring to \cref{poly:fig:bouncep},
 			the curve bounces off the horizontal axis at both zeros, $3$ and $4$.

 			The degree of $q$ is 6. Referring to \cref{poly:fig:bounceq},
 			the curve bounces off the horizontal axis at $-2$ and $1$, and cuts
 			through the horizontal axis at $0$ and $3$.

 			The degree of $r$ is 6. Referring to \cref{poly:fig:bouncer},
 			the curve bounces off the horizontal axis at $-1$, and cuts through
 			the horizontal axis at $0$ and at $3$, although is flattened immediately to the left and right of $3$.
 		\end{pccsolution}
 	\end{pccexample}

 	\setlength{\figurewidth}{0.25\textwidth}
 	\begin{figure}[!htb]
 		\begin{subfigure}{\figurewidth}
 			\begin{tikzpicture}
 				\begin{axis}[
 						xmin=-6,xmax=5,
 						ymin=-30,ymax=200,
 						xtick={-4,-2,...,4},
 						width=\textwidth,
 					]
 					\addplot expression[domain=-5.63733:4.63733,samples=50]{(x-3)^2*(x+4)^2};
 					\addplot[soldot]coordinates{(3,0)(-4,0)};
 				\end{axis}
 			\end{tikzpicture}
 			\caption{$y=p(x)$}
 			\label{poly:fig:bouncep}
 		\end{subfigure}
 		\hfill
 		\begin{subfigure}{\figurewidth}
 			\begin{tikzpicture}
 				\begin{axis}[
 						xmin=-3,xmax=4,
 						xtick={-2,...,3},
 						ymin=-60,ymax=40,
 						width=\textwidth,
 					]
 					\addplot+[samples=50] expression[domain=-2.49011:3.11054]{x*(x+2)^2*(x-1)^2*(x-3)};
 					\addplot[soldot]coordinates{(-2,0)(0,0)(1,0)(3,0)};
 				\end{axis}
 			\end{tikzpicture}
 			\caption{$y=q(x)$}
 			\label{poly:fig:bounceq}
 		\end{subfigure}
 		\hfill
 		\begin{subfigure}{\figurewidth}
 			\begin{tikzpicture}
 				\begin{axis}[
 						xmin=-2,xmax=4,
 						xtick={-1,...,3},
 						ymin=-40,ymax=40,
 						width=\textwidth,
 					]
 					\addplot expression[domain=-1.53024:3.77464,samples=50]{x*(x-3)^3*(x+1)^2};
 					\addplot[soldot]coordinates{(-1,0)(0,0)(3,0)};
 				\end{axis}
 			\end{tikzpicture}
 			\caption{$y=r(x)$}
 			\label{poly:fig:bouncer}
 		\end{subfigure}
 		\caption{}
 		\label{poly:fig:moremultiple}
 	\end{figure}

 	\begin{pccdefinition}[Multiple zeros]\label{poly:def:multzero}
 		Let $p$ be a polynomial that has a repeated linear factor $(x-a)^n$. Then we say
 		that $p$ has a multiple zero at $a$ of multiplicity $n$ and
 		\begin{itemize}
 			\item if the factor $(x-a)$ is repeated an even number of times, the graph of $y=p(x)$ does not
 			      cross the $x$ axis at $a$, but `bounces' off the horizontal axis at $a$.
 			\item if the factor $(x-a)$ is repeated an odd number of times, the graph of $y=p(x)$ crosses the
 			      horizontal axis at $a$, but it looks `flattened' there
 		\end{itemize}
 		If $n=1$, then we say that $p$ has a \emph{simple} zero at $a$.
 	\end{pccdefinition}

 	%===================================
 	%   Author: Hughes
 	%   Date:   May 2011
 	%===================================
 	\begin{pccexample}[Find a formula]
 		Find formulas for the polynomial functions, $p$ and $q$, graphed in \cref{poly:fig:findformulademoboth}.
 		\begin{figure}[!htb]
 			\begin{subfigure}{.45\textwidth}
 				\begin{tikzpicture}
 					\begin{axis}[framed,
 							xmin=-5,xmax=5,
 							ymin=-10,ymax=10,
 							xtick={-4,-2,...,4},
 							minor xtick={-3,-1,...,3},
 							ytick={-8,-6,...,8},
 							width=\textwidth,
 						grid=both]
 						\addplot expression[domain=-3.25842:2.25842,samples=50]{-x*(x-2)*(x+3)*(x+1)};
 						\addplot[soldot]coordinates{(1,8)}node[axisnode,inner sep=.35cm,anchor=west]{$(1,8)$};
 						\addplot[soldot]coordinates{(-3,0)(-1,0)(0,0)(2,0)};
 					\end{axis}
 				\end{tikzpicture}
 				\caption{$p$}
 				\label{poly:fig:findformulademo}
 			\end{subfigure}
 			\hfill
 			\begin{subfigure}{.45\textwidth}
 				\begin{tikzpicture}
 					\begin{axis}[framed,
 							xmin=-5,xmax=5,
 							ymin=-10,ymax=10,
 							xtick={-4,-2,...,4},
 							minor xtick={-3,-1,...,3},
 							ytick={-8,-6,...,8},
 							width=\textwidth,
 						grid=both]
 						\addplot expression[domain=-4.33:4.08152]{-.25*(x+2)^2*(x-3)};
 						\addplot[soldot]coordinates{(2,4)}node[axisnode,anchor=south west]{$(2,4)$};
 						\addplot[soldot]coordinates{(-2,0)(3,0)};
 					\end{axis}
 				\end{tikzpicture}
 				\caption{$q$}
 				\label{poly:fig:findformulademo1}
 			\end{subfigure}
 			\caption{}
 			\label{poly:fig:findformulademoboth}
 		\end{figure}
 		\begin{pccsolution}
 			\begin{enumerate}
 				\item We begin by noting that the horizontal intercepts of $p$ are $(-3,0)$, $(-1,0)$, $(0,0)$ and $(2,0)$.
 				      We also note that each zero is simple (multiplicity $1$).
 				      If we assume that $p$ has no other zeros, then we can start by writing
 				      \begin{align*}
 				      	p(x) & =(x+3)(x+1)(x-0)(x-2) \\
 				      	     & =x(x+3)(x+1)(x-2)     \\
 				      \end{align*}
 				      According to \cref{poly:fig:findformulademo}, the point $(1,8)$ lies
 				      on the curve $y=p(x)$.
 				      Let's check if the formula we have written satisfies this requirement
 				      \begin{align*}
 				      	p(1) & = (1)(4)(2)(-1) \\
 				      	     & = -8
 				      \end{align*}
 				      which is clearly not correct| it is close though. We can correct this by
 				      multiplying $p$ by a constant $k$; so let's assume that
 				      \[
 				      	p(x)=kx(x+3)(x+1)(x-2)
 				      \]
 				      Then $p(1)=-8k$, and if this is to equal $8$, then $k=-1$. Therefore
 				      the formula for $p(x)$ is
 				      \[
 				      	p(x)=-x(x+3)(x+1)(x-2)
 				      \]
 				\item The function $q$ has a zero at $-2$ of multiplicity $2$, and zero of
 				      multiplicity $1$ at $3$ (so $3$ is a simple zero of $q$); we can therefore assume that $q$ has the form
 				      \[
 				      	q(x)=k(x+2)^2(x-3)
 				      \]
 				      where $k$ is some real number. In order to find $k$, we use the given ordered pair, $(2,4)$, and
 				      evaluate $p(2)$
 				      \begin{align*}
 				      	p(2) & =k(4)^2(-1) \\
 				      	     & =-16k
 				      \end{align*}
 				      We solve the equation $4=-8k$ and obtain $k=-\frac{1}{4}$ and conclude that the
 				      formula for $q(x)$ is
 				      \[
 				      	q(x)=-\frac{1}{4}(x+2)^2(x-3)
 				      \]
 			\end{enumerate}
 		\end{pccsolution}
 	\end{pccexample}


 	\fixthis{Chris: need sketching polynomial problems}
 	\begin{pccspecialcomment}[Steps to follow when sketching polynomial functions]
 		\begin{steps}
 			\item \label{poly:step:first} Determine the degree of the polynomial,
 			its leading term and leading coefficient, and hence determine
 			the long-run behavior of the polynomial| does it behave like $\pm x^2$ or $\pm x^3$
 			as $x\rightarrow\pm\infty$?
 			\item Determine the zeros and their multiplicity. Mark all zeros
 			and the vertical intercept on the graph using solid circles $\bullet$.
 			\item \label{poly:step:last}  Deduce the overall shape of the curve, and sketch it. If there isn't
 			enough information from the previous steps, then construct a table of values.
 		\end{steps}
 		Remember that until we have the tools of calculus, we won't be able to
 		find the exact coordinates of local minimums, local maximums, and points
 		of inflection.
 	\end{pccspecialcomment}
 	Before we demonstrate some examples, it is important to remember the following:
 	\begin{itemize}
 		\item our sketches will give a good representation of the overall
 		      shape of the graph, but until we have the tools of calculus (from MTH 251)
 		      we can not find local minimums, local maximums, and inflection points algebraically. This
 		      means that we will make our best guess as to where these points are.
 		\item we will not concern ourselves too much with the vertical scale (because of
 		      our previous point)| we will, however, mark the vertical intercept (assuming there is one),
 		      and any horizontal asymptotes.
 	\end{itemize}
 	%===================================
 	%   Author: Hughes
 	%   Date:   May 2012
 	%===================================
 	\begin{pccexample}\label{poly:ex:simplecubic}
 		Use \crefrange{poly:step:first}{poly:step:last} to sketch a graph of the function $p$
 		that has formula
 		\[
 			p(x)=\frac{1}{2}(x-4)(x-1)(x+3)
 		\]
 		\begin{pccsolution}
 			\begin{steps}
 				\item $p$ has degree $3$. The leading term of $p$ is $\frac{1}{2}x^3$, so the leading coefficient of $p$
 				is $\frac{1}{2}$. The long-run behavior of $p$ is therefore similar to that of $x^3$.
 				\item The zeros of $p$ are $-3$, $1$, and $4$; each zero is simple (i.e, it has multiplicity $1$).
 				This means that the curve of $p$ cuts the horizontal axis at each zero. The vertical
 				intercept of $p$ is $(0,6)$.
 				\item We draw the details we have obtained so far on \cref{poly:fig:simplecubicp1}. Given
 				that the curve of $p$ looks like the curve of $x^3$ in the long-run, we are able to complete a sketch of the
 				graph of $p$ in \cref{poly:fig:simplecubicp2}.

 				Note that we can not find the coordinates of the local minimums, local maximums, and inflection
 				points| for the moment we make reasonable guesses as to where these points are (you'll find how
 				to do this in calculus).
 			\end{steps}

 			\begin{figure}[!htbp]
 				\begin{subfigure}{.45\textwidth}
 					\begin{tikzpicture}
 						\begin{axis}[
 								xmin=-10,xmax=10,
 								ymin=-10,ymax=15,
 								xtick={-8,-6,...,8},
 								ytick={-5,5},
 								width=\textwidth,
 							]
 							\addplot[soldot] coordinates{(-3,0)(1,0)(4,0)(0,6)}node[axisnode,anchor=south west]{$(0,6)$};
 						\end{axis}
 					\end{tikzpicture}
 					\caption{}
 					\label{poly:fig:simplecubicp1}
 				\end{subfigure}%
 				\hfill
 				\begin{subfigure}{.45\textwidth}
 					\begin{tikzpicture}
 						\begin{axis}[
 								xmin=-10,xmax=10,
 								ymin=-10,ymax=15,
 								xtick={-8,-6,...,8},
 								ytick={-5,5},
 								width=\textwidth,
 							]
 							\addplot[soldot] coordinates{(-3,0)(1,0)(4,0)(0,6)}node[axisnode,anchor=south west]{$(0,6)$};
 							\addplot[pccplot] expression[domain=-3.57675:4.95392,samples=100]{.5*(x-4)*(x-1)*(x+3)};
 						\end{axis}
 					\end{tikzpicture}
 					\caption{}
 					\label{poly:fig:simplecubicp2}
 				\end{subfigure}%
 				\caption{$y=\dfrac{1}{2}(x-4)(x-1)(x+3)$}
 				\label{poly:fig:simplecubic}
 			\end{figure}
 		\end{pccsolution}
 	\end{pccexample}

 	%===================================
 	%   Author: Hughes
 	%   Date:   May 2012
 	%===================================
 	\begin{pccexample}\label{poly:ex:degree5}
 		Use \crefrange{poly:step:first}{poly:step:last} to sketch a graph of the function $q$
 		that has formula
 		\[
 			q(x)=\frac{1}{200}(x+7)^2(2-x)(x-6)^2
 		\]
 		\begin{pccsolution}
 			\begin{steps}
 				\item $q$ has degree $4$. The leading term of $q$ is
 				\[
 					-\frac{1}{200}x^5
 				\]
 				so the leading coefficient of $q$ is $-\frac{1}{200}$. The long-run behavior of $q$
 				is therefore similar to that of $-x^5$.
 				\item The zeros of $q$ are $-7$ (multiplicity 2), $2$ (simple), and $6$ (multiplicity $2$).
 				The curve of $q$ bounces off the horizontal axis at the zeros with multiplicity $2$ and
 				cuts the horizontal axis at the simple zeros. The vertical intercept of $q$ is $\left( 0,\frac{441}{25} \right)$.
 				\item We mark the details we have found so far on \cref{poly:fig:degree5p1}. Given that
 				the curve of $q$ looks like the curve of $-x^5$ in the long-run, we can complete \cref{poly:fig:degree5p2}.
 			\end{steps}

 			\begin{figure}[!htbp]
 				\begin{subfigure}{.45\textwidth}
 					\begin{tikzpicture}
 						\begin{axis}[
 								xmin=-10,xmax=10,
 								ymin=-10,ymax=40,
 								xtick={-8,-6,...,8},
 								ytick={-5,0,...,35},
 								width=\textwidth,
 							]
 							\addplot[soldot] coordinates{(-7,0)(2,0)(6,0)(0,441/25)}node[axisnode,anchor=south west]{$\left( 0, \frac{441}{25} \right)$};
 						\end{axis}
 					\end{tikzpicture}
 					\caption{}
 					\label{poly:fig:degree5p1}
 				\end{subfigure}%
 				\hfill
 				\begin{subfigure}{.45\textwidth}
 					\begin{tikzpicture}
 						\begin{axis}[
 								xmin=-10,xmax=10,
 								ymin=-10,ymax=40,
 								xtick={-8,-6,...,8},
 								ytick={-5,0,...,35},
 								width=\textwidth,
 							]
 							\addplot[soldot] coordinates{(-7,0)(2,0)(6,0)(0,441/25)}node[axisnode,anchor=south west]{$\left( 0, \frac{441}{25} \right)$};
 							\addplot[pccplot] expression[domain=-8.83223:7.34784,samples=50]{1/200*(x+7)^2*(2-x)*(x-6)^2};
 						\end{axis}
 					\end{tikzpicture}
 					\caption{}
 					\label{poly:fig:degree5p2}
 				\end{subfigure}%
 				\caption{$y=\dfrac{1}{200}(x+7)^2(2-x)(x-6)^2$}
 				\label{poly:fig:degree5}
 			\end{figure}
 		\end{pccsolution}
 	\end{pccexample}

 	%===================================
 	%   Author: Hughes
 	%   Date:   May 2012
 	%===================================
 	\begin{pccexample}
 		Use \crefrange{poly:step:first}{poly:step:last} to sketch a graph of the function $r$
 		that has formula
 		\[
 			r(x)=\frac{1}{100}x^3(x+4)(x-4)(x-6)
 		\]
 		\begin{pccsolution}
 			\begin{steps}
 				\item $r$ has degree $6$. The leading term of $r$ is
 				\[
 					\frac{1}{100}x^6
 				\]
 				so the leading coefficient of $r$ is $\frac{1}{100}$. The long-run behavior of $r$
 				is therefore similar to that of $x^6$.
 				\item The zeros of $r$ are $-4$ (simple), $0$ (multiplicity $3$), $4$ (simple),
 				and $6$ (simple). The vertical intercept of $r$ is $(0,0)$. The curve of $r$
 				cuts the horizontal axis at the simple zeros, and goes through the axis
 				at $(0,0)$, but does so in a flattened way.
 				\item We mark the zeros and vertical intercept on \cref{poly:fig:degree6p1}. Given that
 				the curve of $r$ looks like the curve of $x^6$ in the long-run, we complete the graph
 				of $r$ in \cref{poly:fig:degree6p2}.
 			\end{steps}

 			\begin{figure}[!htbp]
 				\begin{subfigure}{.45\textwidth}
 					\begin{tikzpicture}
 						\begin{axis}[
 								xmin=-5,xmax=10,
 								ymin=-20,ymax=10,
 								xtick={-4,-2,...,8},
 								ytick={-15,-10,...,5},
 								width=\textwidth,
 							]
 							\addplot[soldot] coordinates{(-4,0)(0,0)(4,0)(6,0)};
 						\end{axis}
 					\end{tikzpicture}
 					\caption{}
 					\label{poly:fig:degree6p1}
 				\end{subfigure}%
 				\hfill
 				\begin{subfigure}{.45\textwidth}
 					\begin{tikzpicture}
 						\begin{axis}[
 								xmin=-5,xmax=10,
 								ymin=-20,ymax=10,
 								xtick={-4,-2,...,8},
 								ytick={-15,-10,...,5},
 								width=\textwidth,
 							]
 							\addplot[soldot] coordinates{(-4,0)(0,0)(4,0)(6,0)};
 							\addplot[pccplot] expression[domain=-4.16652:6.18911,samples=100]{1/100*(x+4)*x^3*(x-4)*(x-6)};
 						\end{axis}
 					\end{tikzpicture}
 					\caption{}
 					\label{poly:fig:degree6p2}
 				\end{subfigure}%
 				\caption{$y=\dfrac{1}{100}(x+4)x^3(x-4)(x-6)$}
 			\end{figure}
 		\end{pccsolution}
 	\end{pccexample}

 	%===================================
 	%   Author: Hughes
 	%   Date:   March 2012
 	%===================================
 	\begin{pccexample}[An open-topped box]
 		A cardboard company makes open-topped boxes for their clients. The specifications
 		dictate that the box must have a square base, and that it must be open-topped.
 		The company uses sheets of cardboard that are $\unit[1200]{cm^2}$. Assuming that
 		the base of each box has side $x$ (measured in cm), it can be shown that the volume of each box, $V(x)$,
 		has formula
 		\[
 			V(x)=\frac{x}{4}(1200-x^2)
 		\]
 		Find the dimensions of the box that maximize the volume.
 		\begin{pccsolution}
 			We graph $y=V(x)$ in \cref{poly:fig:opentoppedbox}. Note that because
 			$x$ represents the length of a side, and $V(x)$ represents the volume
 			of the box, we necessarily require both values to be positive; we illustrate
 			the part of the curve that applies to this problem using a solid line.

 			\begin{figure}[!htb]
 				\centering
 				\begin{tikzpicture}
 					\begin{axis}[framed,
 							xmin=-50,xmax=50,
 							ymin=-5000,ymax=5000,
 							xtick={-40,-30,...,40},
 							minor xtick={-45,-35,...,45},
 							minor ytick={-3000,-1000,1000,3000},
 							width=.75\textwidth,
 							height=.5\textwidth,
 						grid=both]
 						\addplot[pccplot,dashed,<-] expression[domain=-40:0,samples=50]{x/4*(1200-x^2)};
 						\addplot[pccplot,-] expression[domain=0:34.64,samples=50]{x/4*(1200-x^2)};
 						\addplot[pccplot,dashed,->] expression[domain=34.64:40,samples=50]{x/4*(1200-x^2)};
 						\addplot[soldot] coordinates{(20,4000)};
 					\end{axis}
 				\end{tikzpicture}
 				\caption{$y=V(x)$}
 				\label{poly:fig:opentoppedbox}
 			\end{figure}

 			According to \cref{poly:fig:opentoppedbox}, the maximum volume of such a box is
 			approximately $\unit[4000]{cm^2}$, and we achieve it using a base of length
 			approximately $\unit[20]{cm}$. Since the base is square and each sheet of cardboard
 			is $\unit[1200]{cm^2}$, we conclude that the dimensions of each box are $\unit[20]{cm}\times\unit[20]{cm}\times\unit[30]{cm}$.
 		\end{pccsolution}
 	\end{pccexample}

 \subsection*{Complex zeros}
 	There has been a pattern to all of the examples that we have seen so far|
 	the degree of the polynomial has dictated the number of \emph{real} zeros that the
 	polynomial has. For example, the function $p$ in \cref{poly:ex:simplecubic}
 	has degree $3$, and $p$ has $3$ real zeros; the function $q$ in \cref{poly:ex:degree5}
 	has degree $5$ and $q$ has $5$ real zeros.

 	You may wonder if this result can be generalized| does every polynomial that
 	has degree $n$ have $n$ real zeros? Before we tackle the general result,
 	let's consider an example that may help motivate it.
 	%===================================
 	%   Author: Hughes
 	%   Date:   June 2012
 	%===================================
 	\begin{pccexample}\label{poly:ex:complx}
 		Consider the polynomial function $c$ that has formula
 		\[
 			c(x)=x(x^2+1)
 		\]
 		It is clear that $c$ has degree $3$, and that $c$ has a (simple) zero at $0$. Does
 		$c$ have any other zeros, i.e, can we find any values of $x$ that satisfy the equation
 		\begin{equation}\label{poly:eq:complx}
 			x^2+1=0
 		\end{equation}
 		The solutions to \cref{poly:eq:complx} are $\pm i$.

 		We conclude that $c$ has $3$ zeros: $0$ and $\pm i$; we note that \emph{not
 		all of them are real}.
 	\end{pccexample}
 	\Cref{poly:ex:complx} shows that not every degree-$3$ polynomial has $3$
 	\emph{real} zeros; however, if we are prepared to venture into the complex numbers,
 	then we can state the following theorem.
 	%===================================
 	%   Author: Hughes
 	%   Date:   June 2012
 	%===================================
 	\begin{pccspecialcomment}[The fundamental theorem of algebra]
 		Every polynomial function of degree $n$ has $n$ roots, some of which may
 		be complex, and some may be repeated.
 	\end{pccspecialcomment}
 	\fixthis{Fundamental theorem of algebra: is this wording ok? do we want
 	it as a theorem?}
 	%===================================
 	%   Author: Hughes
 	%   Date:   June 2012
 	%===================================
 	\begin{pccexample}
 		Find all the zeros of the polynomial function $p$ that has formula
 		\[
 			p(x)=x^4-2x^3+5x^2
 		\]
 		\begin{pccsolution}
 			We begin by factoring $p$
 			\begin{align*}
 				p(x) & =x^4-2x^3+5x^2 \\
 				     & =x^2(x^2-2x+5)
 			\end{align*}
 			We note that $0$ is a zero of $p$ with multiplicity $2$. The other zeros of $p$
 			can be found by solving the equation
 			\[
 				x^2-2x+5=0
 			\]
 			This equation can not be factored, so we use the quadratic formula
 			\begin{align*}
 				x & =\frac{2\pm\sqrt{(-2)^2}-20}{2(1)} \\
 				  & =\frac{2\pm\sqrt{-16}}{2}          \\
 				  & =1\pm 2i
 			\end{align*}
 			We conclude that $p$ has $4$ zeros: $0$ (multiplicity $2$), and $1\pm 2i$ (simple).
 		\end{pccsolution}
 	\end{pccexample}
 	%===================================
 	%   Author: Hughes
 	%   Date:   June 2012
 	%===================================
 	\begin{pccexample}
 		Find a polynomial that has zeros at $2\pm i\sqrt{2}$.
 		\begin{pccsolution}
 			We know that the zeros of a polynomial can be found by analyzing the linear
 			factors. We are given the zeros, and have to work backwards to find the
 			linear factors.

 			We begin by assuming that $p$ has the form
 			\begin{align*}
 				p(x) & =(x-(2-i\sqrt{2}))(x-(2+i\sqrt{2}))                           \\
 				     & =x^2-x(2+i\sqrt{2})-x(2-i\sqrt{2})+(2-i\sqrt{2})(2+i\sqrt{2}) \\
 				     & =x^2-4x+(4-2i^2)                                              \\
 				     & =x^2-4x+6
 			\end{align*}
 			We conclude that a possible formula for a polynomial function, $p$,
 			that has zeros at $2\pm i\sqrt{2}$ is
 			\[
 				p(x)=x^2-4x+6
 			\]
 			Note that we could multiply $p$ by any real number and still ensure
 			that $p$ has the same zeros.
 		\end{pccsolution}
 	\end{pccexample}
 	\investigation*{}
 	%===================================
 	%   Author: Hughes
 	%   Date:   May 2011
 	%===================================
 	\begin{problem}[Find a formula from a graph]
 	For each of the polynomials in \cref{poly:fig:findformula}
 	\begin{enumerate}
 		\item count the number of times the curve turns round, and cuts/bounces off the $x$ axis;
 		\item approximate the degree of the polynomial;
 		\item use your information to find the linear factors of each polynomial, and therefore write a possible formula for each;
 		\item make sure your polynomial goes through the given ordered pair.
 	\end{enumerate}
 	\begin{shortsolution}
 		\Vref{poly:fig:findformdeg2}:
 		\begin{enumerate}
 			\item the curve turns round once;
 			\item the degree could be 2;
 			\item based on the zeros, the linear factors are $(x+5)$ and $(x-3)$; since the
 			      graph opens downwards, we will assume the leading coefficient is negative: $p(x)=-k(x+5)(x-3)$;
 			\item $p$ goes through $(2,2)$, so we need to solve $2=-k(7)(-1)$ and therefore $k=\nicefrac{2}{7}$, so
 			      \[
 			      	p(x)=-\frac{2}{7}(x+5)(x-3)
 			      \]
 		\end{enumerate}
 		\Vref{poly:fig:findformdeg3}:
 		\begin{enumerate}
 			\item the curve turns around twice;
 			\item the degree could be 3;
 			\item based on the zeros, the linear factors are $(x+2)^2$, and $(x-1)$;
 			      based on the behavior of $p$, we assume that the leading coefficient is positive, and try $p(x)=k(x+2)^2(x-1)$;
 			\item $p$ goes through $(0,-2)$, so we need to solve $-2=k(4)(-1)$ and therefore $k=\nicefrac{1}{2}$, so
 			      \[
 			      	p(x)=\frac{1}{2}(x+2)^2(x-1)
 			      \]
 		\end{enumerate}
 		\Vref{poly:fig:findformdeg5}:
 		\begin{enumerate}
 			\item the curve turns around 4 times;
 			\item the degree could be 5;
 			\item based on the zeros, the linear factors are $(x+5)^2$, $(x+1)$, $(x-2)$, $(x-3)$;
 			      based on the behavior of $p$, we assume that the leading coefficient is positive, and try $p(x)=k(x+5)^2(x+1)(x-2)(x-3)$;
 			\item $p$ goes through $(-3,-50)$, so we need to solve $-50=k(64)(-2)(-5)(-6)$ and therefore $k=\nicefrac{5}{384}$, so
 			      \[
 			      	p(x)=\frac{5}{384}(x+5)^2(x+1)(x-2)(x-3)
 			      \]
 		\end{enumerate}
 	\end{shortsolution}
 	\end{problem}


 	\begin{figure}[!htb]
 		\setlength{\figurewidth}{0.3\textwidth}
 		\begin{subfigure}{\figurewidth}
 			\begin{tikzpicture}
 				\begin{axis}[
 						xmin=-5,xmax=5,
 						ymin=-2,ymax=5,
 						width=\textwidth,
 					]
 					\addplot expression[domain=-4.5:3.75]{-1/3*(x+4)*(x-3)};
 					\addplot[soldot] coordinates{(-4,0)(3,0)(2,2)} node[axisnode,above right]{$(2,2)$};
 				\end{axis}
 			\end{tikzpicture}
 			\caption{}
 			\label{poly:fig:findformdeg2}
 		\end{subfigure}
 		\hfill
 		\begin{subfigure}{\figurewidth}
 			\begin{tikzpicture}
 				\begin{axis}[
 						xmin=-3,xmax=2,
 						ymin=-2,ymax=4,
 						xtick={-2,...,1},
 						width=\textwidth,
 					]
 					\addplot expression[domain=-2.95:1.75]{1/3*(x+2)^2*(x-1)};
 					\addplot[soldot]coordinates{(-2,0)(1,0)(0,-1.33)}node[axisnode,anchor=north west]{$(0,-2)$};
 				\end{axis}
 			\end{tikzpicture}
 			\caption{}
 			\label{poly:fig:findformdeg3}
 		\end{subfigure}
 		\hfill
 		\begin{subfigure}{\figurewidth}
 			\begin{tikzpicture}
 				\begin{axis}[
 						xmin=-5,xmax=5,
 						ymin=-100,ymax=150,
 						width=\textwidth,
 					]
 					\addplot expression[domain=-4.5:3.4,samples=50]{(x+4)^2*(x+1)*(x-2)*(x-3)};
 					\addplot[soldot]coordinates{(-4,0)(-1,0)(2,0)(3,0)(-3,-60)}node[axisnode,anchor=north]{$(-3,-50)$};
 				\end{axis}
 			\end{tikzpicture}
 			\caption{}
 			\label{poly:fig:findformdeg5}
 		\end{subfigure}
 		\caption{}
 		\label{poly:fig:findformula}
 	\end{figure}




 	\begin{exercises}
 	%===================================
 	%   Author: Hughes
 	%   Date:   March 2012
 	%===================================
 	\begin{problem}[Prerequisite classifacation skills]
 	Decide if each of the following functions are linear or quadratic.
 	\begin{multicols}{3}
 		\begin{subproblem}
 			$f(x)=2x+3$
 			\begin{shortsolution}
 				$f$ is linear.
 			\end{shortsolution}
 		\end{subproblem}
 		\begin{subproblem}
 			$g(x)=10-7x$
 			\begin{shortsolution}
 				$g$ is linear
 			\end{shortsolution}
 		\end{subproblem}
 		\begin{subproblem}
 			$h(x)=-x^2+3x-9$
 			\begin{shortsolution}
 				$h$ is quadratic.
 			\end{shortsolution}
 		\end{subproblem}
 		\begin{subproblem}
 			$k(x)=-17$
 			\begin{shortsolution}
 				$k$ is linear.
 			\end{shortsolution}
 		\end{subproblem}
 		\begin{subproblem}
 			$l(x)=-82x^2-4$
 			\begin{shortsolution}
 				$l$ is quadratic
 			\end{shortsolution}
 		\end{subproblem}
 		\begin{subproblem}
 			$m(x)=6^2x-8$
 			\begin{shortsolution}
 				$m$ is linear.
 			\end{shortsolution}
 		\end{subproblem}
 	\end{multicols}
 	\end{problem}
 	%===================================
 	%   Author: Hughes
 	%   Date:   March 2012
 	%===================================
 	\begin{problem}[Prerequisite slope identification]
 	State the slope of each of the following linear functions, and
 	hence decide if each function is increasing or decreasing.
 	\begin{multicols}{4}
 		\begin{subproblem}
 			$\alpha(x)=4x+1$
 			\begin{shortsolution}
 				$m=4$; $\alpha$ is increasing.
 			\end{shortsolution}
 		\end{subproblem}
 		\begin{subproblem}
 			$\beta(x)=-9x$
 			\begin{shortsolution}
 				$m=-9$; $\beta$ is decreasing.
 			\end{shortsolution}
 		\end{subproblem}
 		\begin{subproblem}
 			$\gamma(t)=18t+100$
 			\begin{shortsolution}
 				$m=18$; $\gamma$ is increasing.
 			\end{shortsolution}
 		\end{subproblem}
 		\begin{subproblem}
 			$\delta(y)=23-y$
 			\begin{shortsolution}
 				$m=-1$; $\delta$ is decreasing.
 			\end{shortsolution}
 		\end{subproblem}
 	\end{multicols}
 	Now let's generalize our findings for the most general linear function $f$
 	that has formula $f(x)=mx+b$. Complete the following sentences.
 	\begin{subproblem}
 		When $m>0$, the function $f$ is $\ldots$
 		\begin{shortsolution}
 			When $m>0$, the function $f$ is $\ldots$  \emph{increasing}.
 		\end{shortsolution}
 	\end{subproblem}
 	\begin{subproblem}
 		When $m<0$, the function $f$ is $\ldots$
 		\begin{shortsolution}
 			When $m<0$, the function $f$ is $\ldots$  \emph{decreasing}.
 		\end{shortsolution}
 	\end{subproblem}
 	\end{problem}
 	%===================================
 	%   Author: Hughes
 	%   Date:   May 2011
 	%===================================
 	\begin{problem}[Polynomial or not?]
 	Identify whether each of the following functions is a polynomial or not.
 	If the function is a polynomial, state its degree.
 	\begin{multicols}{3}
 		\begin{subproblem}
 			$p(x)=2x+1$
 			\begin{shortsolution}
 				$p$ is a polynomial (you might also describe $p$ as linear). The degree of $p$ is 1.
 			\end{shortsolution}
 		\end{subproblem}
 		\begin{subproblem}
 			$p(x)=7x^2+4x$
 			\begin{shortsolution}
 				$p$ is a polynomial (you might also describe $p$ as quadratic). The degree of $p$ is 2.
 			\end{shortsolution}
 		\end{subproblem}
 		\begin{subproblem}
 			$p(x)=\sqrt{x}+2x+1$
 			\begin{shortsolution}
 				$p$ is not a polynomial; we require the powers of $x$ to be integer values.
 			\end{shortsolution}
 		\end{subproblem}
 		\begin{subproblem}
 			$p(x)=2^x-45$
 			\begin{shortsolution}
 				$p$ is not a polynomial; the $2^x$ term is exponential.
 			\end{shortsolution}
 		\end{subproblem}
 		\begin{subproblem}
 			$p(x)=6x^4-5x^3+9$
 			\begin{shortsolution}
 				$p$ is a polynomial, and the degree of $p$ is $6$.
 			\end{shortsolution}
 		\end{subproblem}
 		\begin{subproblem}
 			$p(x)=-5x^{17}+9x+2$
 			\begin{shortsolution}
 				$p$ is a polynomial, and the degree of $p$ is 17.
 			\end{shortsolution}
 		\end{subproblem}
 		\begin{subproblem}
 			$p(x)=4x(x+7)^2(x-3)^3$
 			\begin{shortsolution}
 				$p$ is a polynomial, and the degree of $p$ is $6$.
 			\end{shortsolution}
 		\end{subproblem}
 		\begin{subproblem}
 			$p(x)=4x^{-5}-x^2+x$
 			\begin{shortsolution}
 				$p$ is not a polynomial because $-5$ is not a positive integer.
 			\end{shortsolution}
 		\end{subproblem}
 		\begin{subproblem}
 			$p(x)=-x^6(x^2+1)(x^3-2)$
 			\begin{shortsolution}
 				$p$ is a polynomial, and the degree of $p$ is $11$.
 			\end{shortsolution}
 		\end{subproblem}
 	\end{multicols}
 	\end{problem}
 	%===================================
 	%   Author: Hughes
 	%   Date:   May 2011
 	%===================================
 	\begin{problem}[Polynomial graphs]
 	Three polynomial functions $p$, $m$, and $n$ are shown in \crefrange{poly:fig:functionp}{poly:fig:functionn}.
 	The functions have the following formulas
 	\begin{align*}
 		p(x) & = (x-1)(x+2)(x-3)           \\
 		m(x) & = -(x-1)(x+2)(x-3)          \\
 		n(x) & = (x-1)(x+2)(x-3)(x+1)(x+4)
 	\end{align*}
 	Note that for our present purposes we are not concerned with the vertical scale of the graphs.
 	\begin{subproblem}
 		Identify both on the graph {\em and} algebraically, the zeros of each polynomial.
 		\begin{shortsolution}
 			$y=p(x)$ is shown below.

 			\begin{tikzpicture}
 				\begin{axis}[
 						xmin=-5,xmax=5,
 						ymin=-10,ymax=10,
 						width=\solutionfigurewidth,
 					]
 					\addplot expression[domain=-2.5:3.5,samples=50]{(x-1)*(x+2)*(x-3)};
 					\addplot[soldot] coordinates{(-2,0)(1,0)(3,0)};
 				\end{axis}
 			\end{tikzpicture}

 			$y=m(x)$ is shown below.

 			\begin{tikzpicture}
 				\begin{axis}[
 						xmin=-5,xmax=5,
 						ymin=-10,ymax=10,
 						width=\solutionfigurewidth,
 					]
 					\addplot expression[domain=-2.5:3.5,samples=50]{-1*(x-1)*(x+2)*(x-3)};
 					\addplot[soldot] coordinates{(-2,0)(1,0)(3,0)};
 				\end{axis}
 			\end{tikzpicture}

 			$y=n(x)$ is shown below.

 			\begin{tikzpicture}
 				\begin{axis}[
 						xmin=-5,xmax=5,
 						ymin=-90,ymax=70,
 						width=\solutionfigurewidth,
 					]
 					\addplot expression[domain=-4.15:3.15,samples=50]{(x-1)*(x+2)*(x-3)*(x+1)*(x+4)};
 					\addplot[soldot] coordinates{(-4,0)(-2,0)(-1,0)(1,0)(3,0)};
 				\end{axis}
 			\end{tikzpicture}

 			The zeros of $p$ are $-2$, $1$, and $3$; the zeros of $m$ are $-2$, $1$, and $3$; the zeros of $n$ are
 			$-4$, $-2$, $-1$, and $3$.
 		\end{shortsolution}
 	\end{subproblem}
 	\begin{subproblem}
 		Write down the degree, how many times the curve of each function `turns around',
 		and how many zeros it has
 		\begin{shortsolution}
 			\begin{itemize}
 				\item The degree of $p$ is 3, and the curve $y=p(x)$ turns around twice.
 				\item The degree of $q$ is also 3, and the curve $y=q(x)$ turns around twice.
 				\item The degree of $n$ is $5$, and the curve $y=n(x)$ turns around 4 times.
 			\end{itemize}
 		\end{shortsolution}
 	\end{subproblem}
 	\end{problem}

 	\begin{figure}[!htb]
 		\begin{widepage}
 		\setlength{\figurewidth}{0.3\textwidth}
 		\begin{subfigure}{\figurewidth}
 			\begin{tikzpicture}
 				\begin{axis}[
 						xmin=-5,xmax=5,
 						ymin=-10,ymax=10,
 						ytick={-5,5},
 						width=\textwidth,
 					]
 					\addplot expression[domain=-2.5:3.5,samples=50]{(x-1)*(x+2)*(x-3)};
 					\addplot[soldot]coordinates{(-2,0)(1,0)(3,0)};
 				\end{axis}
 			\end{tikzpicture}
 			\caption{$y=p(x)$}
 			\label{poly:fig:functionp}
 		\end{subfigure}
 		\hfill
 		\begin{subfigure}{\figurewidth}
 			\begin{tikzpicture}
 				\begin{axis}[
 						xmin=-5,xmax=5,
 						ymin=-10,ymax=10,
 						ytick={-5,5},
 						width=\textwidth,
 					]
 					\addplot expression[domain=-2.5:3.5,samples=50]{-1*(x-1)*(x+2)*(x-3)};
 					\addplot[soldot]coordinates{(-2,0)(1,0)(3,0)};
 				\end{axis}
 			\end{tikzpicture}
 			\caption{$y=m(x)$}
 			\label{poly:fig:functionm}
 		\end{subfigure}
 		\hfill
 		\begin{subfigure}{\figurewidth}
 			\begin{tikzpicture}
 				\begin{axis}[
 						xmin=-5,xmax=5,
 						ymin=-90,ymax=70,
 						width=\textwidth,
 					]
 					\addplot expression[domain=-4.15:3.15,samples=100]{(x-1)*(x+2)*(x-3)*(x+1)*(x+4)};
 					\addplot[soldot]coordinates{(-4,0)(-2,0)(-1,0)(1,0)(3,0)};
 				\end{axis}
 			\end{tikzpicture}
 			\caption{$y=n(x)$}
 			\label{poly:fig:functionn}
 		\end{subfigure}
 		\caption{}
 		\end{widepage}
 	\end{figure}
 	%===================================
 	%   Author: Hughes
 	%   Date:   May 2011
 	%===================================
 	\begin{problem}[Horizontal intercepts]\label{poly:prob:matchpolys}%
 	State the horizontal intercepts (as ordered pairs) of the following polynomials.
 	\begin{multicols}{2}
 		\begin{subproblem}\label{poly:prob:degree5}
 			$p(x)=(x-1)(x+2)(x-3)(x+1)(x+4)$
 			\begin{shortsolution}
 				$(-4,0)$, $(-2,0)$, $(-1,0)$, $(1,0)$, $(3,0)$
 			\end{shortsolution}
 		\end{subproblem}
 		\begin{subproblem}
 			$q(x)=-(x-1)(x+2)(x-3)$
 			\begin{shortsolution}
 				$(-2,0)$, $(1,0)$, $(3,0)$
 			\end{shortsolution}
 		\end{subproblem}
 		\begin{subproblem}
 			$r(x)=(x-1)(x+2)(x-3)$
 			\begin{shortsolution}
 				$(-2,0)$, $(1,0)$, $(3,0)$
 			\end{shortsolution}
 		\end{subproblem}
 		\begin{subproblem}\label{poly:prob:degree2}
 			$s(x)=(x-2)(x+2)$
 			\begin{shortsolution}
 				$(-2,0)$, $(2,0)$
 			\end{shortsolution}
 		\end{subproblem}
 	\end{multicols}
 	\end{problem}
 	%===================================
 	%   Author: Hughes
 	%   Date:   March 2012
 	%===================================
 	\begin{problem}[Minimums, maximums, and concavity]\label{poly:prob:incdec}
 	Four polynomial functions are graphed in \cref{poly:fig:incdec}. The formulas
 	for these functions are (not respectively)
 	\begin{gather*}
 		p(x)=\frac{x^3}{6}-\frac{x^2}{4}-3x, \qquad q(x)=\frac{x^4}{20}+\frac{x^3}{15}-\frac{6}{5}x^2+1\\
 		r(x)=-\frac{x^5}{50}-\frac{x^4}{40}+\frac{2x^3}{5}+6, \qquad s(x)=-\frac{x^6}{6000}-\frac{x^5}{2500}+\frac{67x^4}{4000}+\frac{17x^3}{750}-\frac{42x^2}{125}
 	\end{gather*}
 	\begin{figure}[!htb]
 		\begin{widepage}
 		\setlength{\figurewidth}{.23\textwidth}
 		\centering
 		\begin{subfigure}{\figurewidth}
 			\begin{tikzpicture}
 				\begin{axis}[
 						framed,
 						width=\textwidth,
 						xmin=-10,xmax=10,
 						ymin=-10,ymax=10,
 						xtick={-8,-6,...,8},
 						ytick={-8,-6,...,8},
 						grid=major,
 					]
 					\addplot expression[domain=-5.28:4.68,samples=50]{-x^5/50-x^4/40+2*x^3/5+6};
 				\end{axis}
 			\end{tikzpicture}
 			\caption{}
 			\label{poly:fig:incdec3}
 		\end{subfigure}
 		\hfill
 		\begin{subfigure}{\figurewidth}
 			\begin{tikzpicture}
 				\begin{axis}[
 						framed,
 						width=\textwidth,
 						xmin=-10,xmax=10,ymin=-10,ymax=10,
 						xtick={-8,-6,...,8},
 						ytick={-8,-6,...,8},
 						grid=major,
 					]
 					\addplot expression[domain=-6.08:4.967,samples=50]{x^4/20+x^3/15-6/5*x^2+1};
 				\end{axis}
 			\end{tikzpicture}
 			\caption{}
 			\label{poly:fig:incdec2}
 		\end{subfigure}
 		\hfill
 		\begin{subfigure}{\figurewidth}
 			\begin{tikzpicture}
 				\begin{axis}[
 						framed,
 						width=\textwidth,
 						xmin=-6,xmax=8,ymin=-10,ymax=10,
 						xtick={-4,-2,...,6},
 						ytick={-8,-4,4,8},
 						minor ytick={-6,-2,...,6},
 						grid=both,
 					]
 					\addplot expression[domain=-4.818:6.081,samples=50]{x^3/6-x^2/4-3*x};
 				\end{axis}
 			\end{tikzpicture}
 			\caption{}
 			\label{poly:fig:incdec1}
 		\end{subfigure}
 		\hfill
 		\begin{subfigure}{\figurewidth}
 			\begin{tikzpicture}
 				\begin{axis}[
 						framed,
 						width=\textwidth,
 						xmin=-10,xmax=10,ymin=-10,ymax=10,
 						xtick={-8,-4,4,8},
 						ytick={-8,-4,4,8},
 						minor xtick={-6,-2,...,6},
 						minor ytick={-6,-2,...,6},
 						grid=both,
 					]
 					\addplot expression[domain=-9.77:8.866,samples=50]{-x^6/6000-x^5/2500+67*x^4/4000+17/750*x^3-42/125*x^2};
 				\end{axis}
 			\end{tikzpicture}
 			\caption{}
 			\label{poly:fig:incdec4}
 		\end{subfigure}
 		\caption{Graphs for \cref{poly:prob:incdec}.}
 		\label{poly:fig:incdec}
 		\end{widepage}
 	\end{figure}
 	\begin{subproblem}
 		Match each of the formulas with one of the given graphs.
 		\begin{shortsolution}
 			\begin{itemize}
 				\item $p$ is graphed in \vref{poly:fig:incdec1};
 				\item $q$ is graphed in \vref{poly:fig:incdec2};
 				\item $r$ is graphed in \vref{poly:fig:incdec3};
 				\item $s$ is graphed in \vref{poly:fig:incdec4}.
 			\end{itemize}
 		\end{shortsolution}
 	\end{subproblem}
 	\begin{subproblem}
 		Approximate the zeros of each function using the appropriate graph.
 		\begin{shortsolution}
 			\begin{itemize}
 				\item $p$ has simple zeros at about $-3.8$, $0$, and $5$.
 				\item $q$ has simple zeros at about $-5.9$, $-1$, $1$, and $4$.
 				\item $r$ has simple zeros at about $-5$, $-2.9$, and $4.1$.
 				\item $s$ has simple zeros at about $-9$, $-6$, $4.2$, $8.1$, and a zero of multiplicity $2$ at $0$.
 			\end{itemize}
 		\end{shortsolution}
 	\end{subproblem}
 	\begin{subproblem}
 		Approximate the local maximums and minimums of each of the functions.
 		\begin{shortsolution}
 			\begin{itemize}
 				\item $p$ has a local maximum of approximately $3.9$ at $-2$, and a local minimum of approximately $-6.5$ at $3$.
 				\item $q$ has a local minimum of approximately $-10$ at $-4$, and $-4$ at $3$; $q$ has a local maximum of approximately $1$ at $0$.
 				\item $r$ has a local minimum of approximately $-5.5$ at $-4$, and a local maximum of approximately $10$ at $3$.
 				\item $s$ has a local maximum of approximately $5$ at $-8$, $0$ at $0$, and $5$ at  $7$; $s$ has local minimums
 				      of approximately $-3$ at $-4$, and $-1$ at $3$.
 			\end{itemize}
 		\end{shortsolution}
 	\end{subproblem}
 	\begin{subproblem}
 		Approximate the global maximums and minimums of each of the functions.
 		\begin{shortsolution}
 			\begin{itemize}
 				\item $p$ does not have a global maximum, nor a global minimum.
 				\item $q$ has a global minimum of approximately $-10$; it does not have a global maximum.
 				\item $r$ does not have a global maximum, nor a global minimum.
 				\item $s$ has a global maximum of approximately $5$; it does not have a global minimum.
 			\end{itemize}
 		\end{shortsolution}
 	\end{subproblem}
 	\begin{subproblem}
 		Approximate the intervals on which each function is increasing and decreasing.
 		\begin{shortsolution}
 			\begin{itemize}
 				\item $p$ is increasing on $(-\infty,-2)\cup (3,\infty)$, and decreasing on $(-2,3)$.
 				\item $q$ is increasing on $(-4,0)\cup (3,\infty)$, and decreasing on $(-\infty,-4)\cup (0,3)$.
 				\item $r$ is increasing on $(-4,3)$, and decreasing on $(-\infty,-4)\cup (3,\infty)$.
 				\item $s$ is increasing on $(-\infty,-8)\cup (-4,0)\cup (3,5)$, and decreasing on $(-8,-4)\cup (0,3)\cup (5,\infty)$.
 			\end{itemize}
 		\end{shortsolution}
 	\end{subproblem}
 	\begin{subproblem}
 		Approximate the intervals on which each function is concave up and concave down.
 		\begin{shortsolution}
 			\begin{itemize}
 				\item $p$ is concave up on  $(1,\infty)$, and concave down on  $(-\infty,1)$.
 				\item $q$ is concave up on $(-\infty,-1)\cup (1,\infty)$, and concave down on $(-1,1)$.
 				\item $r$ is concave up on $(-\infty,-3)\cup (0,2)$, and concave down on $(-3,0)\cup (2,\infty)$.
 				\item $s$ is concave up on $(-6,-2)\cup (2,5)$, and concave down on $(-\infty,-6)\cup (-2,2)\cup (5,\infty)$.
 			\end{itemize}
 		\end{shortsolution}
 	\end{subproblem}
 	\begin{subproblem}
 		The degree of $q$ is $5$. Assuming that all of the real zeros of $q$ are
 		shown in its graph, how many complex zeros does $q$ have?
 		\begin{shortsolution}
 			\Vref{poly:fig:incdec2} shows that $q$ has $3$ real zeros
 			since the curve of $q$ cuts the horizontal axis $3$ times.
 			Since $q$ has degree $5$, $q$ must have $2$ complex zeros.
 		\end{shortsolution}
 	\end{subproblem}
 	\end{problem}

 	%===================================
 	%   Author: Hughes
 	%   Date:   May 2011
 	%===================================
 	\begin{problem}[Long-run behaviour of polynomials]
 	Describe the long-run behavior of each of polynomial functions in
 	\crefrange{poly:prob:degree5}{poly:prob:degree2}.
 	\begin{shortsolution}
 		$\dd\lim_{x\rightarrow-\infty}p(x)=-\infty$,
 		$\dd\lim_{x\rightarrow\infty}p(x)=\infty$,
 		$\dd\lim_{x\rightarrow-\infty}q(x)=\infty$,
 		$\dd\lim_{x\rightarrow\infty}q(x)=-\infty$,
 		$\dd\lim_{x\rightarrow-\infty}r(x)=-\infty$,
 		$\dd\lim_{x\rightarrow\infty}r(x)=\infty$,
 		$\dd\lim_{x\rightarrow-\infty}s(x)=\infty$,
 		$\dd\lim_{x\rightarrow\infty}s(x)=\infty$,
 	\end{shortsolution}
 	\end{problem}

 	%===================================
 	%   Author: Hughes
 	%   Date:   May 2011
 	%===================================
 	\begin{problem}[True of false?]
 	Let $p$ be a polynomial function.
 	Label each of the following statements as true (T) or false (F); if they are false,
 	provide an example that supports your answer.
 	\begin{subproblem}
 		If $p$ has degree $3$, then $p$ has $3$ distinct zeros.
 		\begin{shortsolution}
 			False. Consider $p(x)=x^2(x+1)$ which has only 2 distinct zeros.
 		\end{shortsolution}
 	\end{subproblem}
 	\begin{subproblem}
 		If $p$ has degree $4$, then $\dd\lim_{x\rightarrow-\infty}p(x)=\infty$ and $\dd\lim_{x\rightarrow\infty}p(x)=\infty$.
 		\begin{shortsolution}
 			False. Consider $p(x)=-x^4$.
 		\end{shortsolution}
 	\end{subproblem}
 	\begin{subproblem}
 		If $p$ has even degree, then it is possible that $p$ can have no real zeros.
 		\begin{shortsolution}
 			True.
 		\end{shortsolution}
 	\end{subproblem}
 	\begin{subproblem}
 		If $p$ has odd degree, then it is possible that $p$ can have no real zeros.
 		\begin{shortsolution}
 			False. All odd degree polynomials will cut the horizontal axis at least once.
 		\end{shortsolution}
 	\end{subproblem}
 	\end{problem}
 	%===================================
 	%   Author: Hughes
 	%   Date:   May 2011
 	%===================================
 	\begin{problem}[Find a formula from a description]
 	In each of the following problems, give a possible formula for a polynomial
 	function that has the specified properties.
 	\begin{subproblem}
 		Degree 2 and has zeros at $4$ and $5$.
 		\begin{shortsolution}
 			Possible option: $p(x)=(x-4)(x-5)$. Note we could multiply $p$ by any real number, and still meet the requirements.
 		\end{shortsolution}
 	\end{subproblem}
 	\begin{subproblem}
 		Degree 3 and has zeros at $4$,$5$ and $-3$.
 		\begin{shortsolution}
 			Possible option: $p(x)=(x-4)(x-5)(x+3)$. Note we could multiply $p$ by any real number, and still meet the requirements.
 		\end{shortsolution}
 	\end{subproblem}
 	\begin{subproblem}
 		Degree 4 and has zeros at $0$, $4$, $5$, $-3$.
 		\begin{shortsolution}
 			Possible option: $p(x)=x(x-4)(x-5)(x+3)$. Note we could multiply $p$ by any real number, and still meet the requirements.
 		\end{shortsolution}
 	\end{subproblem}
 	\begin{subproblem}
 		Degree 4, with zeros that make the graph cut at $2$, $-5$, and a zero that makes the graph touch at $-2$;
 		\begin{shortsolution}
 			Possible option: $p(x)=(x-2)(x+5)(x+2)^2$. Note we could multiply $p$ by any real number, and still meet the requirements.
 		\end{shortsolution}
 	\end{subproblem}
 	\begin{subproblem}
 		Degree 3, with only one zero at $-1$.
 		\begin{shortsolution}
 			Possible option: $p(x)=(x+1)^3$. Note we could multiply $p$ by any real number, and still meet the requirements.
 		\end{shortsolution}
 	\end{subproblem}
 	\end{problem}
 	%===================================
 	%   Author: Hughes
 	%   Date:   June 2012
 	%===================================
 	\begin{problem}[\Cref{poly:step:last}]
 	\pccname{Saheed} is graphing a polynomial function, $p$.
 	He is following \crefrange{poly:step:first}{poly:step:last} and has so far
 	marked the zeros of $p$ on \cref{poly:fig:optionsp1}. Saheed tells you that
 	$p$ has degree $3$, but does \emph{not} say if the leading coefficient
 	of $p$ is positive or negative.
 	\begin{figure}[!htbp]
 		\begin{widepage}
 		\begin{subfigure}{.45\textwidth}
 			\begin{tikzpicture}
 				\begin{axis}[
 						xmin=-10,xmax=10,
 						ymin=-10,ymax=10,
 						xtick={-8,-6,...,8},
 						ytick={-15},
 						width=\textwidth,
 						height=.5\textwidth,
 					]
 					\addplot[soldot] coordinates{(-5,0)(2,0)(6,0)};
 				\end{axis}
 			\end{tikzpicture}
 			\caption{}
 			\label{poly:fig:optionsp1}
 		\end{subfigure}%
 		\hfill
 		\begin{subfigure}{.45\textwidth}
 			\begin{tikzpicture}
 				\begin{axis}[
 						xmin=-10,xmax=10,
 						ymin=-10,ymax=10,
 						xtick={-8,-6,...,8},
 						ytick={-15},
 						width=\textwidth,
 						height=.5\textwidth,
 					]
 					\addplot[soldot] coordinates{(-5,0)(6,0)};
 				\end{axis}
 			\end{tikzpicture}
 			\caption{}
 			\label{poly:fig:optionsp2}
 		\end{subfigure}%
 		\caption{}
 		\end{widepage}
 	\end{figure}
 	\begin{subproblem}
 		Use the information in \cref{poly:fig:optionsp1} to help sketch $p$, assuming that the leading coefficient
 		is positive.
 		\begin{shortsolution}
 			Assuming that $a_3>0$:

 			\begin{tikzpicture}
 				\begin{axis}[
 						xmin=-10,xmax=10,
 						ymin=-10,ymax=10,
 						xtick={-8,-6,...,8},
 						ytick={-15},
 						width=\solutionfigurewidth,
 					]
 					\addplot expression[domain=-6.78179:8.35598,samples=50]{1/20*(x+5)*(x-2)*(x-6)};
 					\addplot[soldot] coordinates{(-5,0)(2,0)(6,0)};
 				\end{axis}
 			\end{tikzpicture}
 		\end{shortsolution}
 	\end{subproblem}
 	\begin{subproblem}
 		Use the information in \cref{poly:fig:optionsp1} to help sketch $p$, assuming that the leading coefficient
 		is negative.
 		\begin{shortsolution}
 			Assuming that $a_3<0$:

 			\begin{tikzpicture}
 				\begin{axis}[
 						xmin=-10,xmax=10,
 						ymin=-10,ymax=10,
 						xtick={-8,-6,...,8},
 						ytick={-15},
 						width=\solutionfigurewidth,
 					]
 					\addplot expression[domain=-6.78179:8.35598,samples=50]{-1/20*(x+5)*(x-2)*(x-6)};
 					\addplot[soldot] coordinates{(-5,0)(2,0)(6,0)};
 				\end{axis}
 			\end{tikzpicture}
 		\end{shortsolution}
 	\end{subproblem}
 	Saheed now turns his attention to another polynomial function, $q$. He finds
 	the zeros of $q$ (there are only $2$) and marks them on \cref{poly:fig:optionsp2}.
 	Saheed knows that $q$ has degree $3$, but doesn't know if the leading
 	coefficient is positive or negative.
 	\begin{subproblem}
 		Use the information in \cref{poly:fig:optionsp2} to help sketch $q$, assuming that the leading
 		coefficient of $q$ is positive. Hint: only one of the zeros is simple.
 		\begin{shortsolution}
 			Assuming that $a_4>0$ there are $2$ different options:

 			\begin{tikzpicture}
 				\begin{axis}[
 						xmin=-10,xmax=10,
 						ymin=-10,ymax=10,
 						xtick={-8,-6,...,8},
 						ytick={-15},
 						width=\solutionfigurewidth,
 					]
 					\addplot expression[domain=-8.68983:7.31809,samples=50]{1/20*(x+5)^2*(x-6)};
 					\addplot expression[domain=-6.31809:9.68893,samples=50]{1/20*(x+5)*(x-6)^2};
 					\addplot[soldot] coordinates{(-5,0)(6,0)};
 				\end{axis}
 			\end{tikzpicture}
 		\end{shortsolution}
 	\end{subproblem}
 	\begin{subproblem}
 		Use the information in \cref{poly:fig:optionsp2} to help sketch $q$, assuming that the leading
 		coefficient of $q$ is negative.
 		\begin{shortsolution}
 			Assuming that $a_4<0$ there are $2$ different options:

 			\begin{tikzpicture}
 				\begin{axis}[
 						xmin=-10,xmax=10,
 						ymin=-10,ymax=10,
 						xtick={-8,-6,...,8},
 						ytick={-15},
 						width=\solutionfigurewidth,
 					]
 					\addplot expression[domain=-8.68983:7.31809,samples=50]{-1/20*(x+5)^2*(x-6)};
 					\addplot expression[domain=-6.31809:9.68893,samples=50]{-1/20*(x+5)*(x-6)^2};
 					\addplot[soldot] coordinates{(-5,0)(6,0)};
 				\end{axis}
 			\end{tikzpicture}
 		\end{shortsolution}
 	\end{subproblem}
 	\end{problem}
