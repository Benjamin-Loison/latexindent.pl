% arara: pdflatex: {shell: yes, files: [latexindent]}
% arara: pdflatex: {shell: yes, files: [latexindent]}
\subsection{\texttt{noAdditionalIndent} and \texttt{indentRules}}
	\label{sec:noadd-indent-rules}
	\texttt{latexindent.pl} operates on files by looking for code blocks, as detailed in \vref{subsubsec:code-blocks}; for each type of code block  in \vref{tab:code-blocks} (which we will call a \emph{$\langle$thing$\rangle$} in what follows) it searches YAML fields for information in the following order:
	\begin{enumerate}
		\item \texttt{noAdditionalIndent} for the \emph{name} of the current \emph{$\langle$thing$\rangle$};
		\item \texttt{indentRules} for the \emph{name} of the current \emph{$\langle$thing$\rangle$};
		\item \texttt{noAdditionalIndentGlobal} for the \emph{type} of the current \emph{$\langle$thing$\rangle$};
		\item \texttt{indentRulesGlobal} for the \emph{type} of the current \emph{$\langle$thing$\rangle$}.
	\end{enumerate}

	Using the above list, the first piece of information to be found will be used; failing that, the value of \texttt{defaultIndent} is used.
	If information is found in multiple fields, the first one according to the list above will be used; for example, if information is present in both \texttt{indentRules} and in \texttt{noAdditionalIndentGlobal}, then the information from \texttt{indentRules} takes priority.

	We now present details for the different type of code blocks known to \texttt{latexindent.pl}, as detailed in \vref{tab:code-blocks}; for reference, there follows a list of the code blocks covered.

	\startcontents[noAdditionalIndent]
	\printcontents[noAdditionalIndent]{}{0}{}
