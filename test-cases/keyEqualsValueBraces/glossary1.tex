% http://tex.stackexchange.com/questions/342544/is-there-a-program-for-managing-glossary-tags
\newglossaryentry{Perl}{name=\texttt{Perl},
	sort=Perl, % need a sort key because name contains a command
	description=A scripting language}

\newglossaryentry{glossary}{name=glossary,
	description={\nopostdesc},
	plural={glossaries}}

\newglossaryentry{glossarycol}{
	description={collection of glosses},
	sort={2},
	parent={glossary}}

\newglossaryentry{glossarylist}{
	description={list of technical words},
	sort={1},
	parent={glossary}}

\newglossaryentry{pagelist}{name=page list,
	% description value has to be enclosed in braces
	% because it contains commas
	description={a list of individual pages or page ranges
			(e.g.\ 1,2,4,7-9)}}

\newglossaryentry{mtrx}{name=matrix,
	description={rectangular array of quantities},
	% plural is not simply obtained by appending an s, so specify
	plural=matrices}

% entry with a paragraph break in the description

\newglossaryentry{par}{name=paragraph,
	description={distinct section of piece of
			writing.\glspar Beginning on new, usually indented, line}}

% entry with two types of plural. Set the plural form to the
% form most likely to be used. If you want to use a different
% plural, you will need to explicity specify it in \glslink
\newglossaryentry{cow}{name=cow,
	% this isn't necessary, as this form (appending an s) is
	% the default
	plural=cows,
	% description:
	description={(\emph{pl.}\ cows, \emph{archaic} kine) an adult
			female of any bovine animal}}

\newglossaryentry{bravo}{name={bravo},
	description={\nopostdesc}}

\newglossaryentry{bravo1}{description={cry of approval (pl.\ bravos)},
	sort={1},
	plural={bravos},
	parent=bravo}

\newglossaryentry{bravo2}{description={hired ruffian or killer (pl.\ bravoes)},
	sort={2},
	plural={bravoes},
	parent=bravo}

\newglossaryentry{seal}{%
	name=seal,%
	description={sea mammal with flippers that eats fish}
}

\newglossaryentry{sealion}{%
	name={sea lion},%
	description={large seal}%
}

\newglossaryentry{M}{name={$M$},
	sort=M,
	description={mass},
	symbol=kg}

\newglossaryentry{svm}{
	% how the entry name should appear in the glossary
	name={Support vector machine (SVM)},
	% how the description should appear in the glossary
	description={Statistical pattern recognition
			technique~\cite{svm}},
	% how the entry should appear in the document text
	text={svm},
	% how the entry should appear the first time it is
	% used in the document text
	first={support vector machine (svm)}}

\newglossaryentry{ksvm}{
	name={Kernel support vector machine (KSVM)},
	description={Statistical pattern recognition technique
			using the ``kernel trick'' (see also SVM)},
	text={ksvm},
	first={kernel support vector machine}}

\newglossaryentry{ident}{name=identity matrix,
	description=diagonal matrix with 1s along the leading diagonal,
	plural=identity matrices}


% These are special characters or protected characters. glossaries knows how to handle these.
\newglossaryentry{quote}{name={"},
	description={the double quote symbol}}

\newglossaryentry{at}{name={@},
	description={the ``at'' symbol}}

\newglossaryentry{excl}{name={!},
	description={the exclamation mark symbol}}

\newglossaryentry{bar}{name={\ensuremath{|}},
	description={the vertical bar symbol}}

\newglossaryentry{hash}{name={\#},
	description={the hash symbol}}

\newglossaryentry{emigre}{%
	name={{é}migré},
	description={person who has emigrated to another country,
			especially for political reasons}
}

\newglossaryentry{not:set}{type=notation, % glossary type
	name={$\mathcal{S}$},
	description={A set},
	sort={S}}

%If one wants to use \gls call in a formula, he'd used the \ensuremath command
\newglossaryentry{Gamma}{name=\ensuremath{\Gamma(z)},
	description=Gamma function,
	sort=Gamma}

\newglossaryentry{Phi}{name={\ensuremath{\Phi(\alpha,\gamma;z)}},
	description=confluent hypergeometric function,sort=Pagz}

\newglossaryentry{knu}{name=\ensuremath{k_\nu(x)},
	description=Bateman's function,sort=kv}

%%%%%%%%%%%%%%%%%%%%%%%%%%%%%%%%%%%%%%%%%%%%%%%%%%%%%%%%%%%%%%%
%%% typical acronym definitions

%the typical definition of an acronym is this, that is not quite similar to the name={field}... This could pose a challange for tweaking a program such as jabref...

\newacronym{svm1}% label
{svm1}% abbreviation
{support vector machine one}% long form

\newacronym{laser}{laser}{light amplification by stimulated
	emission of radiation}
