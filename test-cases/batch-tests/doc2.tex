% arara: pdflatex: { files: [latexindent]}
\section{How to use the script}

\subsection{From the command line}\label{sec:commandline}

	You can call the \texttt{-o} switch with the name of the output file \emph{without} an
	extension; in \announce{2017-06-25}{upgrade to -o switch} this case,
	\texttt{latexindent.pl} will use the extension from the original file. For example, the
	following two calls to \texttt{latexindent.pl} are equivalent:%
	\begin{commandshell}
latexindent.pl myfile.tex -o=output
latexindent.pl myfile.tex -o=output.tex
\end{commandshell}

	You can call the \texttt{-o} switch using a \texttt{++} symbol at the end of the name
	\announce{2017-06-25}{++ in o switch} of your output file; this tells
	\texttt{latexindent.pl} to search successively for the name of your output file
	concatenated with $0, 1, \ldots$ while the name of the output file exists. For example,%
	\begin{commandshell}
latexindent.pl myfile.tex -o=output++
\end{commandshell}
	tells \texttt{latexindent.pl} to output to \texttt{output0.tex}, but if it exists then
	output to \texttt{output1.tex}, and so on.

	\texttt{latexindent.pl} will always load \texttt{defaultSettings.yaml} (rhymes with
	camel) and if it is called with the \texttt{-l} switch and it finds
	\texttt{localSettings.yaml} in the same directory as \texttt{myfile.tex}, then, if not
	found, it looks for \texttt{localSettings.yaml} (and friends, see
	\vref{sec:localsettings}) in the current working directory, then
	these%
	\announce{2021-03-14}*{-l switch: localSettings and
		friends} settings will be added to the indentation scheme. Information will be given in
	\texttt{indent.log} on the success or failure of loading \texttt{localSettings.yaml}.
