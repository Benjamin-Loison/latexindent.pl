\section*{Executive Summary}
Sonifications are non-verbal representation of plots or graphs.
This report details the results of an eSTEeM-funded project to investigate the efficacy of sonifications when presented to participants in study-like activities.
We are grateful to eSTEeM for their support and funding throughout the project.

\tableofcontents
\section{Introduction}
The depiction of numerical data using graphs and charts play a vital part in many STEM modules.
As Tufte says in a key text about the design of plots and charts ``at their best graphics are instruments for reasoning about quantitative measurement'' \cite{Tufte1983}.
In this report we will focus on static images, such as graphs and plots in printed materials.
Dynamic images in which the user can change features of the diagram or graph are not considered.
In order to meet the OU's mission of being open to [all] people, such plots and graphs need to be accessible to \emph{all} students, as some students may otherwise be disadvantaged in their study.

The Equality Act 2010 \cite{ehrcequalityact} requires universities to avoid discrimination against people with protected characteristics, including disability, and to do so by making `reasonable adjustments'.
The Equality and Human Rights Commission offers guidance for Higher Education providers \cite{ehrcprovidersguidance}.
The Act created the Public Sector Equality Duty \cite{ehrcpublicsector}, which requires universities to promote equality of opportunity by removing disadvantage and meet the needs of protected groups.
In the context of The Open University, this means that the authors of module materials should ensure that plots and charts (or alternate versions of them) are accessible to all students with visual impairments, including those students with no vision at all.

Individuals who are blind have often had limited access to mathematics and science \cite{advisorycommission}, especially in distance learning courses \cite{educause}; in part this is because of the highly visual nature of the representations of numerical relationships.
Methods commonly used to accommodate learners who are blind or have low vision include: use of sighted assistants who can describe graphics verbally; provision of text-based descriptions of graphics that can be read with text-to-speech applications (for example JAWS \cite{jaws}, Dolphin \cite{dolphin}); accessed as Braille, either as hard copy or via refreshable display, or through provision of tactile graphics for visual representations.

Desirable features of an accessible graph include the following \cite{Summers2012}: \begin{itemize} \item Perceptual precision: the representation allows the user to interpret the plot with an appropriate amount of detail.
	\item First-hand access: the representation allows the user to directly interpret the data and is not reliant on subject interpretation by others (bias).
	\item Works on affordable, mainstream hardware.
	\item Born-accessible: the creator of the plot would not have to put extra effort into creating the accessible version.
\end{itemize}

\begin{figure}[!htb]
	\centering
	\figureDescription{Screenshot of participant S7 interacting with the ferris wheel example.}
	\includegraphics[width=.5\textwidth]{p7-ferris-wheel}
	\caption{Visualisation of the sonification from S7.}
	\label{fig:partipants-ferris-wheel}
\end{figure}
