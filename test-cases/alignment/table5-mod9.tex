\documentclass{article}

\begin{document}

\begin{tabular}{M|MMMMMMMMM}
	   &A_1&A_2&A_3&A_4&A_5&A_6&A_7&A_8&A_9\\\hline
	A_1&0  &   &   &   &   &   &   &   &   \\
	A_2&   &0  &   &   &   &   &   &   &   \\
	A_3&   &   &0  &   &   &   &   &   &   \\
	A_4&   &   &   &0  &   &   &   &   &   \\
	A_5&   &   &   &   &0  &   &   &   &   \\
	A_6&   &   &   &   &   &0  &   &   &   \\
	A_7&   &   &   &   &   &   &0  &   &   \\
	A_8&   &   &   &   &   &   &   &0  &   \\
	A_9&   &   &   &   &   &   &   &   &0  \\
\end{tabular}
\begin{tabularx}{M|MMMMMMMMM}
	    & A_1 & A_2 & A_3 & A_4 & A_5 & A_6 & A_7 & A_8 & A_9 \\\hline
	A_1 & 0   &     &     &     &     &     &     &     &     \\
	A_2 &     & 0   &     &     &     &     &     &     &     \\
	A_3 &     &     & 0   &     &     &     &     &     &     \\
	A_4 &     &     &     & 0   &     &     &     &     &     \\
	A_5 &     &     &     &     & 0   &     &     &     &     \\
	A_6 &     &     &     &     &     & 0   &     &     &     \\
	A_7 &     &     &     &     &     &     & 0   &     &     \\
	A_8 &     &     &     &     &     &     &     & 0   &     \\
	A_9 &     &     &     &     &     &     &     &     & 0   \\
\end{tabularx}
\begin{align*}
	CCI_n          & = \frac{p_n-SMA(p_n)}{0.015 \cdot \sigma(p_n)}                                                                            \\
	\textrm{wobei} & n = \textrm{Perioden, i.\,d.\,R. 20};\p_n = \textrm{Typischer Preis/Kurs};\ SMA(p_n) = \textrm{SMA der typischen Preise}; \\
	               & \sigma(p_n) = \textrm{Standardabweichung}                                                                                 \\
\end{align*}
\end{document}

