% arara: pdflatex
% !arara: indent: {overwrite: yes, trace: yes, localsettings: yes}
\documentclass[11pt]{article}
\usepackage[left=4.5cm,right=2.5cm,showframe=false,
top=2cm,bottom=1.5cm]{geometry}               
\usepackage{parskip}
\usepackage{booktabs}
\usepackage{listings}
\usepackage{titlesec}
\usepackage{xcolor}
%\usepackage{kpfonts}
\usepackage{hyperref}
\usepackage{cleveref}

% tex.stackexchange.com/questions/93944/using-lstinline-inside-a-item/
% allow \lstinline inside an \item
% \begin{noindent}
\makeatletter
\let\orig@item\item

\def\item{%
	\@ifnextchar{[}%
		{\lstinline@item}%
		{\orig@item}%
	}
	
	\begingroup
\catcode`\]=\active
\gdef\lstinline@item[{%
		\setbox0\hbox\bgroup
	\catcode`\]=\active
\let]\lstinline@item@end
}
\endgroup

\def\lstinline@item@end{%
	\egroup
	\orig@item[\usebox0]%
}

\makeatother
% \end{noindent}

\lstset{%
	basicstyle=\small\ttfamily,language={[LaTeX]TeX},   
	numbers=left,
	numberstyle=\ttfamily\small,
	breaklines=true,frame=single,framexleftmargin=8mm, xleftmargin=8mm,
	prebreak = \raisebox{0ex}[0ex][0ex]{\ensuremath{\hookrightarrow}},
	backgroundcolor=\color{green!5},frameround=fttt,
	rulecolor=\color{red},
	keywordstyle=\color[rgb]{0,0,1},                    % keywords
	commentstyle=\color{blue},    % comments
	%columns=fullflexible   
}%
% stolen from arara.sty http://mirrors.med.harvard.edu/ctan/support/arara/doc/arara.sty
\lstnewenvironment{yaml}[1][]{\lstset{%
	basicstyle=\ttfamily,
	numbers=left,
	xleftmargin=1.5em,
	breaklines=true,
	numberstyle=\ttfamily\small,
	columns=flexible,
	mathescape=false,
	#1,
	}}{}
\newcommand{\fixthis}[1]
{%
	\marginpar{\huge \color{red} \framebox{FIX}}%
	\typeout{FIXTHIS: p\thepage : #1^^J}%
}
% custom section
\titleformat{\section}
{\normalfont\Large\bfseries}
{\llap{\thesection\hskip.5cm}}
{0pt}
{}
% custom subsection
\titleformat{\subsection}
{\normalfont\bfseries}
{\llap{\thesubsection\hskip.5cm}}
{0pt}
{}

\titlespacing\section{0pt}{12pt plus 4pt minus 2pt}{-5pt plus 2pt minus 2pt}
\titlespacing\subsection{0pt}{12pt plus 4pt minus 2pt}{-6pt plus 2pt minus 2pt}
\titlespacing\subsubsection{0pt}{12pt plus 4pt minus 2pt}{-6pt plus 2pt minus 2pt}


% cleveref settings
\crefname{table}{Table}{Tables}
\Crefname{table}{Table}{Tables}
\crefname{section}{Section}{Sections}
\Crefname{section}{Section}{Sections}
\crefname{lstlisting}{Listing}{Listings}
\Crefname{lstlisting}{Listing}{Listings}

\begin{document}

\section{How to use the script}
There are two ways to use \lstinline!indent.plx!: from the command line, 
and using \lstinline!arara!.  We will discuss how to change the settings and behaviour 
of the script in \cref{sec:defuseloc}.

\subsection{From the command line}\label{sec:commandline}
\lstinline!indent.plx! has a number of different switches/flags/options, which 
can be combined in any way that you like. \lstinline!indent.plx! 
produces a \lstinline!.log! file, \lstinline!indent.log! every time it
is run. There is a base of information that is written to \lstinline!indent.log!,
but other additional information will be written depending 
on which of the following options are used.
\begin{itemize}
	\item[] \lstinline!indent.plx myfile.tex!
	
	This will simply output to your terminal; \lstinline!myfile.tex! will
	not be changed in any way using this command. 
	\item[\lstinline!-w!] \lstinline!indent.plx -w myfile.tex!
	
	This \emph{will} overwrite \lstinline!myfile.tex!, but it will
	make a copy of \lstinline!myfile.tex! first. You can control the name of 
	the extension (default is \lstinline!.bak!), and how many different backups are made-- 
	more on this in \cref{sec:defuseloc}; see \lstinline!backupExtension! and \lstinline!onlyOneBackUp!.
	\item[\lstinline!-o!] \lstinline!indent.plx -o myfile.tex outputfile.tex!
	
	This will indent \lstinline!myfile.tex! and output it to \lstinline!outputfile.tex!, 
	overwriting it (\lstinline!outputfile.tex!) if it already exists. Note that if \lstinline!indent.plx! is called with both
	the \lstinline!-w! and \lstinline!-o! switches, then \lstinline!-w! will
	be ignored and \lstinline!-o! will take priority (this seems safer than the 
	other way round).
	
	Note that using \lstinline!-o! is equivalent to using \lstinline!indent.plx myfile.tex > outputfile.tex!
	\item[\lstinline!-s!] \lstinline!indent.plx -s myfile.tex!
	
	Silent mode: no output will be given to the terminal.
	\item[\lstinline!-t!] \lstinline!indent.plx -t myfile.tex!
	
	Tracing mode: verbose output will be given to \lstinline!indent.log!. This 
	is useful if \lstinline!indent.plx! has made a mistake and you're
	trying to find out where and why.
	\item[\lstinline!-l!] \lstinline!indent.plx -l myfile.tex!
	
	Local settings: you might like to read \cref{sec:defuseloc} before 
	using this switch. \lstinline!indent.plx! will always load \lstinline!defaultSettings.yaml!
	and if it is called with the \lstinline!-l! switch and it finds \lstinline!localSettings.yaml! 
	in the same directory as \lstinline!myfile.tex! then these settings will be 
	added to the indentation scheme.
\end{itemize}
\subsection{From arara}
Using \lstinline!indent.plx! from the command line is fine for some folks, but
others may find it easier to use from \lstinline!arara!. \lstinline!indent.plx!
ships with an \lstinline!arara! rule-- you can either copy it to the directory of
your other \lstinline!arara! rules, or otherwise add the \lstinline!indent.plx!
directory to your \lstinline!araraconfig.yaml! file.

Once you have told \lstinline!arara! where to find your \lstinline!indent! rule, 
you can use it any of the following ways (or combinations thereof). 

\begin{lstlisting}[caption={\lstinline!arara! samples},escapeinside={(*@}{@*)}]
%(*@@*) arara: indent
%(*@@*) arara: indent: {overwrite: yes}
%(*@@*) arara: indent: {output: myfile.tex}
%(*@@*) arara: indent: {silent: yes}
%(*@@*) arara: indent: {trace: yes}
%(*@@*) arara: indent: {localsettings: yes}
\documentclass{article}
...
\end{lstlisting}

Hopefully the use of these rules is fairly self-explanatory, but for completeness
\cref{tab:orbsandswitches} shows the relationship between \lstinline!arara! orb-tags and the 
switches given in \cref{sec:commandline}.

\begin{table}[!ht]
	\centering
	\caption{Orb tags and corresponding switches}
	\label{tab:orbsandswitches}
	\begin{tabular}{lc}
		\toprule
		\lstinline!arara! orb tags & switch         \\
		\midrule
		\lstinline!overwrite!      & \lstinline!-w! \\
		\lstinline!output!         & \lstinline!-o! \\
		\lstinline!silent!         & \lstinline!-s! \\
		\lstinline!trace!          & \lstinline!-t! \\
		\lstinline!localsettings!  & \lstinline!-l! \\
		\bottomrule
	\end{tabular}
\end{table}
\section{default, user, and local settings}\label{sec:defuseloc}
\lstinline!indent.plx! loads its settings from \lstinline!defaultSettings.yaml! 
(rhymes with camel). The idea is to separate the behaviour of the script 
from the internal working-- this is very similar to the way that we separate content
from form when writing our documents in \LaTeX.


\subsection{\lstinline!defaultSettings.yaml!}
If you look in \lstinline!defaultSettings.yaml! you'll find the switches 
that govern the behaviour of \lstinline!indent.plx!. The code is commented, 
but here is a description of what each switch is designed to do. The default 
value is given in each case.

You can certainly feel free to edit \lstinline!defaultSettings.yaml!, but 
this is not ideal as it may be overwritten when you update your distribution--
all of your hard work tweaking the script would be undone! Don't worry, 
there's a solution-- feel free to peek ahead to \cref{sec:indentconfig} if you like.
\begin{itemize}
	\item[\lstinline!defaultindent!] \lstinline!"\t"!
	
	This is the default indentation (\lstinline!\t! means a tab) used in the absence of other details 
	for the command or environment we are working with-- see \lstinline!indentrules!
	for more details (\cpageref{page:indentrules}).
	\item[\lstinline!backupExtension!] \lstinline!.bak!
	
	If you call \lstinline!indent.plx! with the \lstinline!-w! switch (to overwrite
	\lstinline!myfile.tex!) then it will create a backup file before doing 
	any indentation: \lstinline!myfile.bak0! 
	
	By default, every time you call \lstinline!indent.plx! after this with 
	the \lstinline!-w! switch it will create \lstinline!myfile.bak1!, \lstinline!myfile.bak2!, 
	etc.
	\item[\lstinline!onlyOneBackUp!] \lstinline!0!
	
	If you don't want a backup for everytime that you call \lstinline!indent.plx! (so 
	you don't want \lstinline!myfile.bak1!, \lstinline!myfile.bak2!, etc) and you simply
	want \lstinline!myfile.bak! (or whatever you chose \lstinline!backupExtension! to be)
	then change \lstinline!onlyOneBackUp! to \lstinline!1!.
	\item[\lstinline!indentPreamble!] \lstinline!0!
	
	The preamble of a document can sometimes contain some trickier code 
	for \lstinline!indent.plx! to work with. By default, \lstinline!indent.plx!
	won't try to operate on the preamble, but if you'd like it to try then 
	change \lstinline!indentPreamble! to \lstinline!1!.
	\item[\lstinline!alwaysLookforSplitBraces!] \lstinline!1!
	
	This switch tells \lstinline!indent.plx! to look for commands that 
	can split \emph{braces} across lines, such as \lstinline!parbox!, \lstinline!tikzset!, etc. In older
	versions of \lstinline!indent.plx! you had to specify each one in \lstinline!checkunmatched!-- this 
	clearly became tedious, hence the introduction of \lstinline!alwaysLookforSplitBraces!. 
	
	\emph{As long as you leave this switch on (set to 1) you don't need to specify which 
	commands can split braces across lines}.
	\item[\lstinline!alwaysLookforSplitBrackets!] \lstinline!1!
	
	This switch tells \lstinline!indent.plx! to look for commands that 
	can split \emph{brackets} across lines, such as \lstinline!psSolid!, \lstinline!pgfplotstabletypeset!, 
	etc. In older versions of \lstinline!indent.plx! you had to specify each one in \lstinline!checkunmatchedbracket!-- 
	this clearly became tedious, hence the introduction of \lstinline!alwaysLookforSplitBraces!. 
	
	\emph{As long as you leave this switch on (set to 1) you don't need to specify which 
	commands can split brackets across lines}.
	
	\item[\lstinline!lookforaligndelims!] This is the first example of a field
	in \lstinline!defaultSettings.yaml! that has more than one line; \cref{lst:aligndelims}
	shows more details.
	
	\begin{yaml}[caption={\lstinline!lookforaligndelims!},label={lst:aligndelims}]
lookforaligndelims:
   tabular: 1
   align: 1
   align*: 1
   alignat: 1
   alignat*: 1
   cases: 1
   dcases: 1
   aligned: 1
   pmatrix: 1
   listabla: 1
	\end{yaml}
	
	You can populate this field with any other environments that you have that contain \lstinline!&!. 
	If you change your mind, just turn them off by setting them to \lstinline!0! instead.
	
	\item[\lstinline!verbatimEnvironments!] A field that contains a list of environments
	that you would like left completely alone-- no indentation will be done
	to environments that you have specified in this field-- see \cref{lst:verbatimEnvironments}.
	
	\begin{yaml}[caption={\lstinline!verbatimEnvironments!},label={lst:verbatimEnvironments}]
verbatimEnvironments:
    verbatim: 1
    lstlisting: 1
	\end{yaml}
	
	\item[\lstinline!noindentblock!] If you have a block of code that you don't 
	want \lstinline!indent.plx! to touch (even if it is \emph{not} a verbatim-like
	environment) then you can wrap it in an environment from \lstinline!noindentblock!
	
	\begin{yaml}[caption={\lstinline!noindentblock!},label={lst:noindentblock}]
noindentblock:
    noindent: 1
    cmhtest: 1
	\end{yaml}
	
	Of course, you don't want to have to specify these as null environments
	in your code, so you use them with a comment symbol, \lstinline!%!, as demonstrated in \cref{lst:noindentblockdemo}
	\begin{lstlisting}[caption={\lstinline!noindentblock! demonstration},label={lst:noindentblockdemo},escapeinside={(*@}{@*)}]
%(*@@*) \begin{noindent} 
        this code
                won't be touched
        by \lstinline!indent.plx!
%(*@@*) \end{noindent} 
	    \end{lstlisting}
	
	\item[\lstinline!noAdditionalIndent!] If you would prefer some of your
	environments or commands not to receive any additional indent, then 
	populate \lstinline!noAdditionalIndent!; see \cref{lst:noAdditionalIndent}. 
	Note that these environments will still receive the \emph{current} level
	of indentation unless they belong to \lstinline!verbatimEnvironments!, or \lstinline!noindentblock!.
	
	\begin{yaml}[caption={\lstinline!noAdditionalIndent!},label={lst:noAdditionalIndent}]
noAdditionalIndent:
    document: 1
    pccexample: 1
    pccdefinition: 1
    problem: 1
    exercises: 1
    pccsolution: 1
    foreach: 0
    widepage: 1
    comment: 1
    \[: 0
    frame: 0
	  \end{yaml}
	
	\item[\lstinline!indentrules!] If\label{page:indentrules} you would prefer to specify 
	individual rules for certain environments or commands, just
	populate \lstinline!indentrules!; see \cref{lst:indentrules}
	
	\begin{yaml}[caption={\lstinline!indentrules!},label={lst:indentrules}]
indentrules:
   myenvironment: "\t\t"
   anotherenvironment: "\t\t\t\t"
   \[: "\t"
	\end{yaml}
	
	%# *** NOTE ***
	%# If you have specified alwaysLookforSplitBraces: 1
	%# and alwaysLookforSplitBrackets: 1 then you don't need
	%# to worry about completing
	%#
	%#       checkunmatched
	%#       checkunmatchedELSE
	%#       checkunmatchedbracket
	%#
	%# in other words, you don't really need to edit anything 
	%# below this line- it used to be necessary for older 
	%# versions of the script, but not anymore :)
	%#***      ***
	%
	%# commands that might split {} across lines
	%# such as \parbox, \marginpar, etc
	%checkunmatched:
	%    parbox: 1
	%    vbox: 1
	%
	%# very similar to %checkunmatched except these 
	%# commands might have an else construct
	%checkunmatchedELSE:
	%    pgfkeysifdefined: 1
	%    DTLforeach: 1
	%    ifthenelse: 1
	%
	%# commands that might split []  across lines
	%# such as \pgfplotstablecreatecol, etc
	%checkunmatchedbracket:
	%    pgfplotstablecreatecol: 1
	%    pgfplotstablesave: 1
	%    pgfplotstabletypeset: 1
	%    mycommand: 1
	%    psSolid: 1
\end{itemize}


\subsection{\lstinline!indentconfig.yaml!}\label{sec:indentconfig}
A better way to change the settings is to set up your own settings file, 
\lstinline!mysettings.yaml! (or any name you like, provided it ends with \lstinline!.yaml!). 
The only thing you have to do is tell \lstinline!indent.plx! where to find it. 

\lstinline!indent.plx! will always check your home directory for \lstinline!indentconfig.yaml!, 
which is a plain text file you can create that contains the \emph{absolute}
paths for any settings files that you wish \lstinline!indent.plx! to load.

\begin{yaml}[caption={\lstinline!indentconfig.yaml!}]
# Paths to user settings for indent.plx
#
# Note that the settings will be read in the order you 
# specify here- each successive settings file will overwrite
# the variables that you specify

paths:
- /home/cmhughes/Documents/yamlfiles/mysettings.yaml
- /home/cmhughes/folder/othersettings.yaml
- /some/other/Desktop/othersettings.yaml
\end{yaml}
\fixthis{need Windows setup}

\subsection{Settings load order}
\lstinline!indent.plx! loads the settings files in the following order:
\begin{enumerate}
	\item \lstinline!defaultSettings.yaml! (always loaded)
	\item \lstinline!mySettings.yaml! (and any other files specified in \lstinline!indentconfig.yaml!)
	\item \lstinline!localSettings.yaml! (if found in same directory as \lstinline!myfile.tex! and called
	with \lstinline!-l! switch)
\end{enumerate}

\section{Before and after}
should this be motivation?
\end{document}
